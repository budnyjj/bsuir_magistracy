\subsection{Core RCU API Usage} \label{sec:api_usage}
The core RCU API is quite small and consists of only five primitives:
\co{rcu_read_lock()}, \co{rcu_read_unlock()}, \co{synchronize_rcu()},
\co{rcu_assign_pointer()}, and \co{rcu_dereference()}~\cite{McKenneyOSR08}.

An RCU read-side critical section begins with \co{rcu_read_lock()}
and ends with a corresponding \co{rcu_read_unlock()}.
When nested, they are flattened into one large critical section.
Within a critical section, it is illegal to block, but
preemption is legal in a preemptible kernel.
RCU-protected data accessed by a 
read-side critical section will not be reclaimed until after that
critical section completes.

The function \co{synchronize_rcu()} marks the end of the updater code and 
the beginning of the reclaimer code. It blocks until all pre-existing RCU
read-side critical sections have completed. Note that \co{synchronize_rcu()} 
does not necessarily wait for critical sections that begin after
it does.

%\comment{Lihao: add example, e.g. http://paulmck.livejournal.com/39343.html}

\begin{figure}[tbp]
\centering
\footnotesize
\begin{verbatim}
               int x = 0;
               int y = 0;
               int r1, r2;
               
               void rcu_reader(void) {
                 rcu_read_lock();
                 r1 = x; 
                 r2 = y; 
                 rcu_read_unlock();
               }
               
               void rcu_updater(void) {
                 x = 1; 
                 synchronize_rcu();
                 y = 1; 
               }

               ...

               // after both rcu_reader() 
               // and rcu_updater() return
               assert(r2 == 0 || r1 == 1);
\end{verbatim}
\caption{Verifying RCU Grace Periods}
\label{fig:verify_rcu_gp}
\end{figure}

Consider the example given in
Figure~\ref{fig:verify_rcu_gp}.
If the RCU read-side critical section in function \co{rcu_reader()} begins
before \co{synchronize_rcu()} in \co{rcu_updater()} is called, then it must 
finish before \co{synchronize_rcu()} returns, so that the value of
\co{r2} must be 0. If it ends after \co{synchronize_rcu()} returns, then 
the value of \co{r1} must be 1.

Finally, to assign a new value to an RCU-protected pointer, RCU updaters use 
\co{rcu_assign_pointer()}, which returns the new value. RCU readers 
can use \co{rcu_dereference()} to fetch an RCU-protected pointer, 
which can then be safely dereferenced. 
%Note that this API does not actually dereference the pointer.
%Instead, it only fetches the pointer for later dereferencing.
The returned value is only valid within the enclosing RCU read-side critical section.
% Lihao: include this in PhD thesis and the technical report
The \co{rcu_assign_pointer()} and \co{rcu_dereference()} functions work
together to ensure that if a given reader dereferences an RCU-protected pointer to 
a just-inserted object, the dereference operation will return valid data rather
than pre-initialization garbage.
