%\documentclass{acm_proc_article-sp}
%\documentclass{sig-alternate}
\documentclass[preprint, 10pt, numbers]{sigplanconf}

\usepackage{cmap}                    % Улучшенный поиск русских слов в полученном pdf-файле
\usepackage[T1,T2A]{fontenc}	     % Поддержка русских букв
\usepackage[utf8x]{inputenc}	     % Кодировка utf8
\usepackage[english, russian]{babel} % Языки: русский, английский
\usepackage{pscyr}				     % Красивые русские шрифты
\renewcommand{\rmdefault}{ftm}       % Включаем Times New Roman

\usepackage{tikz}         % for graphics
\usepackage{caption}
\usepackage{pgfplots}
\usepackage{amssymb}      % some useful maths symbols
\usepackage{pifont}       % to get dingbats
\usepackage{url}          % for typesetting URLs
\usepackage{fancyvrb}     % for code snippets
\usepackage{listings}
\usepackage{paralist}
\usepackage{color}                % only for autor notes to each other
%\def\comment#1{{\color{red}#1}}  % display comments/notes.
%\def\comment#1{{\color{red}}}    % hide comments/notes.

\lstset{language=C, basicstyle=\ttfamily, xleftmargin=2pc}
\lstset{
  literate={\_}{}{0\discretionary{\_}{}{\_}}%
}
\newcommand{\co}[1]{\lstinline[breaklines=true,breakatwhitespace=true]{#1}}
\newcommand{\mkcol}[1]{\textcolor{red}{\textbf{#1}}}

\begin{document}

% ---------------------------------------------------------------------
% Title and authors,  adherence to SIGS style
% ---------------------------------------------------------------------
\title{Верификация иерархического механизма Read-Copy~Update ядра Linux}
\date{}

% sigplan template
%\authorinfo{\textit{Authors and affiliations omitted, for double-blind review.}}

\authorinfo{Lihao Liang}
           {University of Oxford}
           {lihao.liang@cs.ox.ac.uk}

\authorinfo{Paul E. McKenney}
           {Linux Technology Center, IBM}
           {paulmck@linux.vnet.ibm.com}

\authorinfo{Daniel Kroening}
           {University of Oxford}
           {daniel.kroening@cs.ox.ac.uk}

\authorinfo{Tom Melham}
           {University of Oxford}
           {tom.melham@cs.ox.ac.uk}

% sig-alternate template
%\numberofauthors{4}
%\author{
%%\textit{Authors and affiliations omitted, for double-blind review.}
%%
%  Lihao Liang \\
%  \affaddr{University of Oxford} \\
%  %\email{lihao.liang@cs.ox.ac.uk}
%  \alignauthor
%%
%  Paul E. McKenney \\
%  \affaddr{Linux Technology Center, IBM} \\
%  %\email{paulmck@linux.vnet.ibm.com}
%  \alignauthor
%%
%  Daniel Kroening \\
%  \affaddr{University of Oxford} \\
%  %\email{daniel.kroening@cs.ox.ac.uk}
%  \alignauthor
%%
%  Tom Melham \\
%  \affaddr{University of Oxford} \\
%  %\email{tom.melham@cs.ox.ac.uk}
%}

% ---------------------------------------------------------------------
% Start of text.
% ---------------------------------------------------------------------


\maketitle

\begin{abstract}
Read-Copy Update (RCU) --- это высокопроизводительный масштабируемый
механизм синхронизации ядра операционной системы Linux,
который позволяет выполнять нетребовательные к ресурсам запросы на
чтение данных вместе с запросами на их изменение.
Реализация качественного RCU для многоядерных систем является
весьма сложной задачей. Учитывая распространенность Linux,
даже самая редкая ошибка исходного кода реализации будет проявлятся недопустимо часто.
В связи с этим, строгая валидация сложных сценариев поведения RCU является
критически важной. В связи с тем, что исчерпывающее тестирование данного
механизма невозможно из-за экпоненциального роста числа сценариев тестирования,
имеет смысл использовать метод формальной верификации.

Следует отметить, что прошлые попытки верификации RCU были направлены
либо на более простые реализации, либо использовали языки моделирования,
что требует процесса ручного перевода исходного текста ядра,
который также подвержен ошибкам.
Кроме этого, подобный перевод придется выполнять слишком часто, поскольку
в реализацию RCU Linux регулярно вносятся правки.
В этой статье мы опишем реализацию Tree RCU в ядре Linux,
затем рассмотрим подход к построению модели верификации напрямую из
исходного кода реализации и использованию верификатора CBMC для проверки ее инвариантов.
По нашим сведениям, это первая попытка верификации существенной части
исходного кода RCU и важный шаг на пути интеграции процедуры формальной верификации
в набор регрессионных тестов ядра Linux.
\end{abstract}

\category{}{D.2.4}{Программное обеспечение/Верификация программ}[Верификация моделей]
\category{}{D.1.3}{Многопоточное программирование}[Параллельное программирование]

\keywords
Верификация программного обеспечения, Параллельные вычисления, Read-Copy Update, ядро Linux

% Include the sections of the paper.

\section*{ВВЕДЕНИЕ}
\addcontentsline{toc}{section}{Введение}

Математический аппарат нечеткой логики находит широкое применение
в различных отраслях экономики: металлургия, автомобилестроение,
логистика, медицина и др.~\cite{terano92}.
Основными прикладынми задачами, решаемыми с помощью нечеткого логического вывода,
являются задачи оценивания и управления в условиях неполноты или
недостаточной точности входных данных.
К достоинствам данного подхода обычно причисляют простоту реализации,
низкую вычислительную сложность, высокую эффективность работы.
Основным его недостатком является необходимость в
привлечении эксперта, способного составить исходный набор правил системы вывода.

Еще одним направлением развития искусственного интеллекта являются нейронные сети.
В их работе можно выделить два этапа:
\begin{enumerate}
\item Обучение, в ходе которого производится настройка сети на основании
  учебных примеров исходных данных, носящих репрезентативный характер.
\item Нормальный режим работы, в ходе которого решается поставленная задача
  классификации, прогнозирования и~т.~п. на основании реальных исходных данных.
\end{enumerate}

Нейронные сети позволяют получать ответы приемлемой точности в задачах,
точное решение которых является невозможным или нецелесообразным.
Кроме этого, при их использовании отпадает необходимость в услугах эксперта,
поскольку её настройка производится на этапе обучения автоматически.

К недостаткам нейронных сетей обычно причисляют отсутствие
детерминированной зависимости между её топологией,
разрешающей способностью (числом искусственных нейронов),
размером обучающей выборки и качеством решения задачи.
Складывается впечатление, что необходимость эмпирического подбора параметров сети
для того, чтобы она наилучшим образом подходила для каждой конкретной задачи,
сдерживает темпы роста использования данного подхода для
решения широкого прикладных задач в различных отраслях народного хозяйства.

В данной работе рассматривается гибридный подход,
основанный на идеях нечеткого логического вывода и
искусственных нейронных сетей.
Целью работы является изложение принципа работы \emph{ANFIS} ---
адаптивной сети на основе системы нечеткого
вывода~\cite{Jang93anfis:adaptive-network-based}.
Изложение основных понятий, необходимых для понимания работы ANFIS,
производится на основании~\cite{kruglov2001}.
                 % Introduction
\section{Background}

\subsection{Что такое RCU?}

Read-copy update (RCU) --- это механизм синхронизации,
часто используемый взамен блокировок чтения-записи.
RCU позволяет потокам-читателям выполняться одновременно
с потоками-писателями, избегая использования блокировок
чтения за счет управления жизненными циклами
множества версий целевого объекта.
В частности, данный механизм следит, чтобы объект,
к которому обращается поток-читатель,
не был удален в течение некоторого периода
после его изменения потоком-писателем.
Суть метода состоит в том, чтобы разделить процесс обновления
объекта на фазу удаления и освобождения, между которыми находится
некоторый промежуток времени --- \emph{grace-период}~\cite{McKenneyRCU98}.
В ходе фазы удаления выполнятеся удаление ссылок на объекты,
доступных для потоков-читателей, сопровождающееся, возможно,
заменой их новыми версиями.

Современные процессоры гарантируют, что операции чтения
одиночных выравненных указателей являются атомарными,
поэтому потоки-читатели могут получить доступ исключительно
к старой либо новой версии объекта чтения.
Atomic-write semantics позволяет выполнять атомарные вставки,
удаления и замены в связанных структурах данных.
Это, в свою очередь, позволяет потокам-читателям отказаться
от использования <<дорогих>> атомарных операций,
избавиться от барьеров памяти и связанных с ними промахов кэша.
Действительно, в наиболее оптимизированных конфигурациях Linux RCU,
потоки-читатели могут выполнять точно такую же последовательность
инструкций, какая использовалась бы в их однопоточной реализации,
что обеспечивает их отличную производительность и масштабируемость.

Как показано на рисунке~\ref{fig:rcu_concepts}, \emph{grace}-периоды
в действительности нужны только тем потокам-читателям,
у которыз момент чтения накладывается на фазу удаления.
Те из них, которые выполняются после удаления, не могут удерживать ссылки
на удаленные объекты и поэтому не могут быть заблокированы
в ходе фазы освобождения.

\begin{figure}[tbp]
\centering
\includegraphics[scale=0.35]{rcu_concepts.pdf}
\caption{Поведение RCU}
\label{fig:rcu_concepts}
\end{figure}

\subsection{Программный интерфейс RCU} \label{sec:api_usage}
Программный интерфейс RCU достаточно невелик и состоит всего из пяти операций:
\co{rcu_read_lock()}, \co{rcu_read_unlock()}, \co{synchronize_rcu()},
\co{rcu_assign_pointer()}, and \co{rcu_dereference()}~\cite{McKenneyOSR08}.

Критическая секция RCU-читателя начинается с \co{rcu_read_lock()}
и заканчивается соответсвующим \co{rcu_read_unlock()}.
Вложенные критические секции чтения объединяются.
Внутри критической секции запрещается блокирование данного потока.
Данные, чтение которых осуществляется доступ внутри критической секции RCU,
будут доступны до её окончания.

Функция \co{synchronize_rcu()} соответсвует окончанию выполнения кода,
обновляющего значение объекта, тем самым сигнализируя о начале фазы освобождения.
Она блокирует поток-писатель до тех пор, пока все потоки-читатели
не выйдут из своих критических RCU-секций.
Отметим, что \co{synchronize_rcu()} не ожидает окончания
критических секций, вход в которые был осуществлен позже её вызова.

%\comment{Lihao: add example, e.g. http://paulmck.livejournal.com/39343.html}

\begin{figure}[tbp]
\centering
\footnotesize
\begin{verbatim}
               int x = 0;
               int y = 0;
               int r1, r2;

               void rcu_reader(void) {
                 rcu_read_lock();
                 r1 = x;
                 r2 = y;
                 rcu_read_unlock();
               }

               void rcu_updater(void) {
                 x = 1;
                 synchronize_rcu();
                 y = 1;
               }

               ...

               // after both rcu_reader()
               // and rcu_updater() return
               assert(r2 == 0 || r1 == 1);
\end{verbatim}
\caption{Verifying RCU Grace Periods}
\label{fig:verify_rcu_gp}
\end{figure}

Рассмотрим пример, приведенный на рисунке~\ref{fig:verify_rcu_gp}.
Если вход в критическую секцию чтения функции \co{rcu_reader()} выполнится до
вызова \co{synchronize_rcu()} в \co{rcu_updater()}, то выход из ней должен быть
совершен до возврата из \co{synchronize_rcu()}, чтобы значение переменной
\co{r2} было равно 0. Если же вход в неё произойдет после возврата из
\co{synchronize_rcu()}, то значение \co{r1} будет равным 1.

Наконец, для присвоения нового значения указателю, защищенному RCU,
потоки-писатели должны использовать \co{rcu_assign_pointer()},
которая возвращает новое значение.
RCU-читатели могут использовать \co{rcu_dereference()} для чтения
указателя, защищенного RCU, который впоследствии может быть безопасно разыменован.
%Note that this API does not actually dereference the pointer.
%Instead, it only fetches the pointer for later dereferencing.
Возвращаемое ею значение является корректным лишь внутри критической секции.
% Lihao: include this in PhD thesis and the technical report
Функции \co{rcu_assign_pointer()} и \co{rcu_dereference()} используются в паре
для того, чтобы убедиться, что если данный поток-читатель разыменовывает
защищенный указатель на только что вставленный объект, операция разыменования
вернет корректное значение, а не недоинициализированный мусор.

            % Background
\section{Реализация Tree RCU}\label{sec:tree_rcu}

Основное преимущество механизма RCU заключается в том, что он позволяет
ожидать выхода весьма большого числа потоков-читателей из своих
критических секций без необходимости учета каждого из них:
в ядрах с non-preemptible реализацией многопоточности их число
ограничено количеством ядер процессора,
в ядрах с preemptible реализацией --- неограниченно вовсе.
Несмотря на то, что примитивы чтения RCU обладают замечательными
показателями производительности и масштабируемости,
примитивы записи должны оттягивать фазу освобождения до тех пор,
пока все потоки-читатели не выйдут из своих критических секций,
за счет блокирования или регистрации callback'ов, которые должны быть
вызваны по истечении grace-периода.
Производительность и масштабируемость RCU определяются
эффективностью механизмов обнаружения окончания grace-периода.
Например, простейшая реализация RCU может требовать,
чтобы каждое ядро процессора использовало глобальную блокировку
для каждого grace-периода, но этот подход существенно снизит
производительность и масштабируемость.
На реальных системах, имеющих тысячи процессоров и управляемых Linux,
данный подход неприменим. Этот факт послужил причиной создания Tree RCU.

\subsection{Обзор}

Будем рассматривать <<стандартный>> программный интерфейс RCU
в комбинации с non-preemptible версией ядра Linux,
концентрируясь в основном на примитивах
\co{rcu_read_lock()}, \co{rcu_read_unlock()} и \co{synchronize_rcu()}.
%
%\comment{Lihao: it seems the Tree implementation of Classic RCU also
%implements other three flavors: RCU-sched, RCU-bh, and RCU-preempt;
%and Tiny RCU implements RCU-sched and RCU-bh. Am I right?
%Conceptually, what is the relationship between flavors Classic RCU and
%RCU-sched/RCU-bh? I shall discuss their relationship here otherwise
%readers may get confused when we discuss different flavors in the
%Tree RCU implementation in later sections.}
%\comment{Paul:
%	Yes, Tree RCU implements RCU-sched, RCU-bh, and RCU-preempt,
%	but only when \co{CONFIG_PREEMPT=y}.
%	If \co{CONFIG_PREEMPT=n}, then RCU-preempt is mapped into
%	RCU-sched.
%	Because Tiny RCU is requires \co{CONFIG_PREEMPT=n}, it behaves
%	the same as does Tree RCU when \co{CONFIG_PREEMPT=n},
%	implementing RCU-sched and RCU-bh, and mapping RCU-preempt into
%	RCU-sched.
%	For RCU-preempt, any location outside of an RCU read-side
%	critical section is a quiescent state.
%	For RCU-sched, context switch, idle, userspace,
%	\co{cond_resched_rcu_qs()}, and offline are all quiescent
%	states.
%	For RCU-bh, any location where bottom-half execution is enabled
%	is a quiescent state.
%	Use RCU-sched when you need updaters to wait on hardware interrupt
%	handlers (device drivers) or preempt-disable regions (tracing).
%	Use RCU-bh when networking denial-of-service attacks are a potential
%	issue.}
%
% In a non-pre\-empt\-ible kernel, Tiny and Tree RCU use the same
% \co{rcu_read_lock()} and \co{rcu_read_unlock()} implementation.
% Tiny RCU's \co{synchronize_rcu()} implementation is trivial,
% while preemptible and non-pre\-emptible Tree RCU largely share a rather
% elaborate implementation.
%
Основная идея заключается в том, что примитивы чтения RCU являются
частью ядра и поэтому в его non-preemptible конфигурациях не блокируются.
Поэтому каждый раз, когда ядро процессора простаивает в состоянии бездействия
или блокируется в процессе выполнения пользовательских программ,
все критические секции чтения RCU, запущенные ранее на этом ядре,
оказываются завершенными.
Поэтому каждое из этих состояний называется \emph{устойчивым состоянием}.
Каждый переход через устойчивое состояние сигнализирует об
окончании соответствующего grace-периода.
Основная сложность заключается в том, чтобы определить момент,
когда все необходимые устойчивые состояния были пройлены для данного
grace-периода, сохранив при этом высокую производительность и масштабируемость.

Например, использование единой структуры данных для регистрации устойчивых
состояний каждого ядра приводит к неприемлемо частому использованию
блокировок на крупных системах, что в свою очередь приводит к снижению
производительности.
Для решения этой проблемы в Tree RCU используется иерархическая организация
структур данных, каждый узел которой предназначен для учета устойчивых
состояний отдельного ядра и предоставляет свою информацию более высоким уровням.
По достижении корня дерева grace-период заканчивается и
информация о нем распространяется по всем узлам-потомкам.
Вскоре после того, как узлы получают данную информацию,
происходит возврат из \co{synchronize_rcu()}.

В оставшейся части данного раздела мы рассмотрим реализацию Tree RCU
в non-preemptible конфигурации ядра Linux версии 4.3.6.
Вначале мы вкратце опишем реализацию примитивов чтения и записи,
затем опишем иерархическую структуру данных, используемую для эффективного
учета устойчивых состояни1, и, наконец, рассмотрим, как RCU использует
эту структуру данных для фиксации устойчивых состояний и grace-периодов
без учета отдельных потоков-читателей.

%\subsection{Read-Side Primitives} \label{sec:read_api_impl}
\subsection{Read/Write-Side Primitives} \label{sec:api_impl}
% Change in recent kernels.
В non-preemptible версии ядра любая область его исходного кода,
которая не использует добровольных блокировок, является неявной
критической секцией чтения RCU. В связи с этим, реализации
\co{rcu_read_lock()} и \co{rcu_read_unlock()} не должны выполнять
никакой работы. Действительно, в production сборках ядра
с выключенным режимом отладки, эта пара примитивов является
пустышками.

В общем случае, когда используется несколько вычислительных ядер процессора,
примитив записи \co{synchronize_rcu()} вызывает \co{wait_rcu_gp()},
которая является внутренней функцией, использующей механизм callback'ов
для отложенного вызова \co{wakeme_after_rcu()} по окончании некоторого grace-периода.
Как подсказывает название, данная функция повторно предназначена для повторного вызова
\co{wait_rcu_gp()}, которая на этот раз ничего не делает,
тем самым позволяя \co{synchronize_rcu()} вернуть управление в вызывающий поток.

%\comment{Lihao: comment out the following preemptible RCU contents if we need space.}
%In a preemptible kernel, \co{synchronize_rcu()} is implemented in
%\co{kernel/rcu/tree_plugin.h}. It first checks whether the variable
%\co{rcu_scheduler_active} is zero. If so, the system is so early in boot
%that there is only one non-preemptible task, again meaning that grace
%periods complete instantaneously, allowing an immediate return.
%Otherwise, if the grace period should be expedited,
%\co{synchronize_rcu_expedited()} is invoked.
%Otherwise, it passes \co{call_rcu()} to \co{wait_rcu_gp()}, which
%registers callback \co{wakeme_after_rcu()}, similar to
%the non-preemptible kernel discussed above.
%%\comment{Lihao: the source code comments state that \co{rcu_scheduler_active = 0}
%%allows RCU to optimize \co{synchronize_sched()} to a simple \co{barrier()}.
%%Where is the code that does this?}
%%\comment{Paul: The comment is incorrect.
%%The \co{synchronize_sched()} function instead checks the number of
%%online CPUs.
%%I have queued a patch with your
%%Reported-by changing the comment's \co{synchronize_sched()} to
%%\co{synchronize_rcu()}.}
%
%RCU's callback handling and grace period detection are explained in Sections
%\ref{sec:rcu_data} and \ref{sec:grace_period}, respectively.

% Lihao: understand how call_rcu_sched works and understand the differences from
% the Tiny RCU version which only calls the kernel function cond_resched()
%
% Lihao: In a preemptible kernel, the implementation of \co{synchronize_rcu()}
% http://lxr.free-electrons.com/source/kernel/rcu/tree_plugin.h#L539 and
% understand the differences between preemptible and non-preemptible versions

\subsection{Data Structures of Tree RCU} \label{sec:data_structure}
% In Section~\ref{sec:rcu_softirq}, we discuss how RCU's softirq handlers walk up the 
% tree hierarchy of the \co{rcu_node} data structure. In ths section, we explain 
% in detail how this data structure is implemented in \co{kernel/rcu/tree.h} and used 
% in Tree RCU.

\begin{figure}[tbp]
\centering
\includegraphics[scale=0.2]{tree_rcu_hierarchy.pdf}
\caption{Tree RCU Hierarchy}
\label{fig:tree_rcu_hierarchy}
\end{figure}

RCU's global state is recorded in the \co{rcu_state} structure, which consists of 
a tree of \co{rcu_node} structures with a child count of up to 64
(32 in a 32-bit system). Every leaf node can have at most 64 
\co{rcu_data} structures (again 32 on a 32-bit system), each representing
a single CPU, as illustrated in
Figure~\ref{fig:tree_rcu_hierarchy}.
%
Each \co{rcu_data} structure records its CPU's quiescent states, and
the \co{rcu_node} tree propagates these states up to the root, and then
propagates grace-period information back down to the leaves.
%
Quiescent-state information does not propagate upwards from a given node
until a quiescent state has been reported by each CPU covered by the subtree
headed by that node.
This propagation scheme dramatically reduces the lock contention experienced
by the upper levels of the tree.
%
% Lihao: include this in PhD thesis and the technical report
For example, consider a default \co{rcu_node} tree for a 4,096-CPU system,
which will have have 256 leaf nodes, four internal nodes, and one root node.
During a given grace period, each CPU will report its quiescent states
to its leaf node, but there will only be 16 CPUs contending for each of
those 256 leaf nodes.
Only 256 of the CPUs will report quiescent states to the internal nodes,
with only 64 CPUs contending for each of the four internal nodes.
Only four CPUs will report quiescent states to the root node, resulting
in extremely low contention on the root node's lock, so that contention
on any given \co{rcu_node} structure is sharply bounded even in very
large configurations.
%
The current RCU implementation in the Linux kernel supports up to a
four-level tree, and thus in total $64^4 = 16,777,216$ CPUs in a 64
bit machine.\footnote{
	Four-level trees are only used in stress testing,
	but three-level trees are used in production by 4096-CPU systems.}

\subsubsection{\co{rcu_state} Structure}

\begin{figure}[tbp]
\centering
\includegraphics[scale=0.9]{rcu_node_array.pdf}
\caption{Array Representation for a Tree of \co{rcu_node} Structures}
\label{fig:rcu_node_array}
\end{figure}

Each flavor of RCU has its own global \co{rcu_state} structure. 
%For example, \co{rcu_state} pointers \co{rcu_sched_state}, 
%\co{rcu_bh_state} and \co{rcu_state_p} are used by RCU-sched, RCU-bh 
%and RCU-preempt, respectively. 
The \co{rcu_state} structure includes
a array of \co{rcu_node} structures organized as a tree
\co{struct rcu_node node[NUM_RCU_NODES]}, with
\co{rcu_data} structures connected to the leaves.
Given this organization, a breadth-first traversal is 
simply a linear scan of the array.
%\comment{Lihao: we may also remove the following sentence and the figure 
%as it's a bit too technical and the level array is never refered again in the paper}
% Lihao: include this in the technical report
Another array \co{struct rcu_node} \co{*level[NUM_RCU_LVLS]} 
is used to point to the left-most node at each level of the tree,
as shown in Figure~\ref{fig:rcu_node_array}.

The \co{rcu_state} structure uses \co{unsigned long} fields \co{->gpnum}
and \co{->completed} to track RCU's grace periods.
The \co{->gpnum} field records the most recently started grace period,
whereas \co{->completed} records the most recently ended grace period.
If the two numbers are equal, then corresponding flavor of RCU is idle.
If \co{gpnum} is one greater than \co{completed}, then RCU is in the
middle of a grace period.
All other combinations are invalid.
% Which of course means that we could instead use a single counter with an
% odd/even scheme to track grace periods.

%(Lihao: ignore variables used to force quiescent states by force_quiescent_state())

\subsubsection{\co{rcu_node} Structure}
\label{sec:rcu_node}
The tree of \co{rcu_node} structures records and 
propagates quiescent-state information from the leaves to the root,
and also propagates grace-period information from the root to the leaves. 
%
The \co{rcu_node} structure has a spinlock \co{->lock} to protect its fields.
The \co{->parent} field references the parent \co{rcu_node} structure,
and is \co{NULL} for the root.
The \co{->level} field indicates the level in the tree, counting from zero
at the root.
The \co{->grpmask} field identifies this node's bit in the
\co{->qsmask} field of its parent.
The \co{->grplo} and \co{->grphi} fields indicates the lowest and highest 
numbered CPU that are covered by this \co{rcu_node} structure, respectively.

The \co{->qsmask} field indicates which of this node's children
still need to report quiescent states for the current grace period.
%
As with \co{rcu_state}, the \co{rcu_node} structure has \co{->gpnum} 
and \co{->completed} fields that have values identical to those of the
enclosing \co{rcu_state} structure, except at the beginnings and ends
of grace periods when the new values are propagated down the tree.
Each of these fields can be smaller than 
its \co{rcu_state} counterpart by at most one.

%\comment{Lihao: comment out the following preemptible RCU contents if we need space.}
%In a preemptible kernel, tasks can be preempted during RCU read-side
%critical sections.
%When an RCU read-side critical section is preempted,
%the preempted task's \co{task_struct} is enqueued onto the \co{->blkd_tasks}
%list in the leaf \co{rcu_node} structure covering the task's CPU.
%That task will remove itself once it reaches the RCU read-side critical
%section's outermost \co{rcu_read_unlock()},
%%
%When the \co{->gp_tasks} pointer is non-\co{NULL}, it references the first
%task blocking the current grace period.
%When a task referenced by \co{gp_tasks} points is removed 
%from \co{blkd_tasks}, the pointer will be advanced to the next task on the list,
%or is set to \co{NULL} if there are no more tasks.
%Note that
%tasks blocking the current grace period are queued in the reverse time order.
%Thus, if a task is blocking a grace period, 
%all subsequent tasks on the list are blocking the same grace period.
% Lihao: how the tasks are dequeued is described in quiescent state detection
% Lihao: we ignore expedited grace period for now
% Lihao: we don't model priority boosting

\subsubsection{\co{rcu_data} structure} \label{sec:rcu_data}
The \co{rcu_data} structure detects quiescent states and handles RCU
callbacks for the corresponding CPU.
The structure is accessed primarily from the corresponding CPU,
thus avoiding synchronization overhead.
As with the \co{rcu_state} structure, different flavors of RCU maintain 
their own per-CPU \co{rcu_data} structures. %For instance, RCU-sched's 
%\co{rcu_sched_state}, RCU-bh's \co{rcu_bh_state} and RCU-preempt's 
%\co{rcu_state_p} structures have \co{rcu_data} structures \co{rcu_sched_data}, 
%\co{rcu_bh_data}, and \co{rcu_data_p}, respectively.
%
The \co{->cpu} field identifies the corresponding CPU, the \co{->rsp}
field references the corresponding \co{rcu_state} structure, and the
\co{->mynode} field references the corresponding leaf \co{rcu_node}
structure.
The \co{->grpmask} field identifies this \co{rcu_data} structure's bit
in the \co{->qsmask} field of its leaf \co{rcu_node} structure.

The \co{rcu_data} structure's \co{->qs_pending} field indicates that RCU
needs a quiescent state from the corresponding CPU, and the
\co{->passed_quiesce} indicates that the CPU has already passed through
a quiescent state.
%
The \co{rcu_data} also has \co{->gpnum} and \co{->completed} fields,
which can lag arbitrarily behind their counterparts in
the \co{rcu_state} and \co{rcu_node} structures on idle CPUs.
However, on the non-idle CPUs that are the focus of this paper,
they can lag at most one grace period behind their leaf \co{rcu_node} 
counterparts.

The \co{rcu_state} structure's \co{->gpnum} and \co{->completed} fields
represent the most current values, and are tracked closely by those of
the \co{rcu_node} structure, which allows the \co{->gpnum} and
\co{->completed} fields in the \co{rcu_data} structures to be
are compared against their counterparts in the corresponding leaf \co{rcu_node}
to detect a new grace period. 
This scheme allows CPUs to detect beginnings and ends of grace periods without
incurring lock- or memory-contention penalties.
%
The \co{rcu_data} structure manages RCU callbacks using a 
four-segment list~\cite{LaiJiangshan2008NewClassicAlgorithm}.

% Lihao: but we need to carefully manage the numbers of each node as the consequences of
% using a quiescent state in a wrong grace period can be quite serious.
% Paul: Indeed!  And the grace-period initialization (rcu_gp_init()) and
% cleanup (rcu_gp_cleanup()) code first updates the rcu_state structure and
% then the rcu_node structures in breadth-first order to avoid such
% consequences.  In addition, cleanup propagates ->completed completely
% and only then is ->gpnum propagated for the new grace period.  Attempting
% to "optimize" this to propagate ->completed and ->gpnum changes in one
% pass results in nasty race conditions caused by different CPUs believing
% that different active grace periods are in effect.  Very low probability,
% but -very- nasty.

% Lihao: we don't model dyntick-idle handling
% Lihao: include this in the technical report
\begin{figure}[tbp]
\centering
\includegraphics[scale=0.25]{rcu_data_callbacks.pdf}
\caption{Очередь callback'ов в \co{rcu_data}}
\label{fig:rcu_data_callbacks}
\end{figure}

\subsubsection{Callback'и RCU}
Структура \co{rcu_data} управляет RCU callback'ами, используя указатель
\co{->nxtlist}, указывающим на начало списка, и массив \co{->nxttail[]}
указателей-на-конец, формирующий четырехсегментный список
callback'ов~\cite{LaiJiangshan2008NewClassicAlgorithm},
где каждый элемент массива \co{->nxttail[]} указывает на конец
соответсвующего сегмента, как показано на рисунке~\ref{fig:rcu_data_callbacks}.
Сегмент, оканчивающийся на \co{->nxttail[RCU_DONE_TAIL]}
(<<\co{RCU_DONE_TAIL} сегмент>>), содержит готовые к вызову callback'и,
связанные с предыдущим grace-периодом.
Сегменты \co{RCU_WAIT_TAIL} и \co{RCU_NEXT_READY_TAIL}
содержат callback'и, ожидающие окончания текущего и следующего
grace-периодов, соответственно.
Наконец, сегмент \co{RCU_NEXT_TAIL} содержит callback'и,
которые еще не были связаны с каким-либо grace-периодом.
Поле \co{->qlen} выполняет учет общего числа callback'ов,
а \co{->blimit} определяет максимальное число callback'ов,
которые могут быть вызваны в данный момент времени, тем самым
ограничивая размер окна времени, используемого для их вызова,
на длинных списках callback'ов.\footnote{
  Вычислительные окружения реального времени, требующие выполнения
  более строгих ограничений времени вызова, должны использовать
  callback offloading, который находится вне контекста рассмотрения данной статьи.}

Возвращаясь к рисунку~\ref{fig:rcu_data_callbacks} отметим,
что элемент массива \co{->nxttail[RCU_DONE_TAIL]} указывает на \co{->nxtlist},
что означает, что в данный момент ни один из callback'ов не готов к вызову.
Элемент \co{->nxttail[RCU_WAIT_TAIL]} указывает на \co{->next}-указатель
второго callback'а, что означает, что callback'и CB~1 и CB~2 ожидают окончания данного
grace-периода.
Элемент \co{->nxttail[RCU_NEXT_READY_TAIL]} указывает на этот же
\co{->next}-указатель, что означает, что список не содержит callback'ов,
связанных со следующим grace-периодом.
Наконец, callback'и, расположенные между \co{->nxttail[RCU_NEXT_READY_TAIL]} и
\co{->nxttail[RCU_NEXT_TAIL]} элементами (CB~3 и CB~4),
еще не связаны ни с каким grace-периодом.
Элемент \co{->nxttail[RCU_NEXT_TAIL]} всегда указывает либо на последний callback,
либо, если весь список пустой, на \co{->nxtlist}.

Cache locality достигается за счет вызова callback'ов на тех вычислительных ядрах,
которые их зарегистрировали.
Например, примитив записи RCU \co{synchronize_rcu()} добавляет
callback \co{wakeme_after_rcu()} в конец списка \co{->nxttail[RCU_NEXT_TAIL]}
на данном вычислительном ядре (раздел \ref{sec:update_api_impl}).
По окончании данного grace-периода, которому соответствует изменение значения
поля \co{->completed} структуры \co{rcu_data}, меньшего, чем соответствующее значение
структуры \co{rcu_node}, они смещаются на один сегмент списка
(с помощью \co{rcu_advance_cbs()}).
Кроме этого, ядро процессора периодически объединяет сегменты \co{RCU_NEXT_TAIL}
и \co{RCU_NEXT_READY_TAIL} путем вызова \co{rcu_accelerate_cbs()}.
В некоторых специальных случаях, ядро выполняет объединение сегментов
\co{RCU_NEXT_TAIL} и \co{RCU_WAIT_TAIL}, пропуская сегмент \co{RCU_NEXT_TAIL}.
Эта оптимизация применяется в тех случаях, когда ядро начинает новый grace-период.
Она \emph{не} используется, когда ядро обнаруживает новый grace-период,
поскольку этот период мог начаться до момента добавления callback'ов
в сегмент \co{RCU_NEXT_TAIL}.

Это особенность архитектуры неслучайна: требование обеспечения
независимости работы вычислительных ядер
(для избегания исопльзования блокировок) является более важным,
чем сокращение продолжительности grace-периодов.
В тех редких случаях, когда требуются grace-периоды должны быть
максимально короткими, требуется использовать \co{synchronize_rcu_expedited()}.
Эта функция имеет такую же семантику, как и \co{synchronize_rcu()},
но предпочитает уменьшение задержки остальным оптимизациям.

Каждый RCU callback представляет собой структуру \co{rcu_head},
имеющую поле \co{->next}, указывающее на следующий callback в списке,
и поле \co{->func}, указывающее на функцию, подлежащую вызову
по окончаниии предстоящего grace-периода.


\subsection{Quiescent State Detection} \label{sec:quiescent_state}
RCU has to wait until all pre-existing read-side critical sections have
finished before it can safely allow a grace period to end.
The performance and scalability of RCU rely on its ability to efficiently
detect quiescent states and determine whether the set of quiescent states
detected thus far allows the grace period to end.
If each CPU (or, in the case of preemptible RCU, each task)
has passed through a quiescent state, a grace period has elapsed. 

% Lihao: include this in PhD thesis; also look for 'preemptible RCU contents'
%Different flavors of RCU use different sets of quiescent states.
The non-preemptible RCU-sched flavor's quiescent states
apply to CPUs, and are user-space execution, context switch, idle, and 
offline state.
%
%RCU-bh's quiescent states are those of RCU-sched plus any execution 
%in which bottom-half (AKA softirq) is enabled, along with transitions
%from one softirq handler to another.
%%
%RCU-preempt's quiescent states are any execution outside of an
%RCU read-side critical sections.
%
Therefore, RCU-sched %and RCU-bh need only
only needs to track tasks and interrupt handlers that are actually running because
blocked and preempted tasks are always in quiescent states. Thus, RCU-sched %and RCU-bh 
needs only track CPU states.
%By contrast, RCU-preempt must track tasks states. In this section, 
%we focus on the quiescent-state detection of RCU-sched in a non-preemptible kernel.
%\comment{Lihao: comment out the following preemptible RCU contents if we need space.}
%However, there can be a great many tasks, and scanning all of them could
%result in excessive per-grace-period overheads.
%However, if a task has been preempted or has blocked outside of an RCU
%read-side critical section, its state need not be considered.
%Therefore, a given grace period need only consider tasks that have been
%preempted (or, in real-time variants of the Linux kernel, blocked)
%within an RCU read-side critical section that began before the current
%grace period did.
%These are exactly the tasks that are tracked by the \co{rcu_node} structure's
%\co{->blkd_tasks} list and \co{->gp_tasks} pointer, as discussed in
%Section~\ref{sec:rcu_node}.

\subsubsection{Scheduling-Clock Interrupt} \label{sec:timer_interrupt}
The \co{rcu_check_callbacks()} is invoked from the sched\-ul\-ing-clock interrupt
handler, which allows RCU to periodically check whether a given busy CPU
is in the user-mode or idle-loop quiescent states.
If the CPU is in one of these quiescent states, \co{rcu_check_callbacks()}
invokes \co{rcu_sched_qs()}, %and \co{rcu_bh_qs()}, 
which sets the per-CPU \co{rcu_sched_data.passed_quiesce} %and \co{rcu_bh_data.passed_quiesce}
fields to 1. %, respectively.
%In addition, when the scheduling-clock interrupt happens outside of softirq context,
%that is, outside of the softirq handler and also outside of any
%bottom-half-disable mode, then the CPU is in an RCU-bh quiescent state,
%in which case \co{rcu_check_callbacks()} will invoke \co{rcu_bh_qs()}
%to inform RCU of this quiescent state.

%\comment{Lihao: comment out the following preemptible RCU contents if we need space.}
%When RCU-preempt is present,
%\co{rcu_check_callbacks()} invokes
%\co{rcu_preempt_check_callbacks}, which
%checks for RCU-preempt quiescent states, which are in effect whenever
%the per-task \co{->rcu_read_lock_nesting} field is equal to 0.
%This condition causes \co{rcu_preempt_check_callbacks} to invoke
%\co{rcu_preempt_qs()}, which in turn records
%quiescent state for by setting the per-CPU \co{rcu_data_p->passed_quiesce} 
%field to true.
%
The \co{rcu_check_callbacks()} function invokes \co{rcu_pending()}
to determine whether a recent event or current condition means that
RCU requires attention from this CPU.
% so the next outermost \co{rcu_read_unlock()} will announce a quiescent state.
% http://lxr.free-electrons.com/source/kernel/rcu/tree_plugin.h#L485
If so, \co{rcu_check_callbacks()} invokes \co{raise_softirq()}, %with a \co{RCU_SOFTIRQ} argument, 
which will cause \co{rcu_process_callbacks()} to be invoked once the CPU 
reaches a state where it is safe to do so (roughly speaking, once the CPU 
has interrupts, preemption, and bottom halves enabled). This function is 
discussed in detail in Section \ref{sec:grace_period}.

% Paul: Yes, the order is different than the code, but it seems to flow better
% this way.  Perhaps I should re-order the code.  Except that I don't touch
% that particular piece of code without an extremely good reason.  ;-)

% Finally, if the scheduling-clock interrupt is raised in the user mode, we perform
% a context switch.
% Lihao: voluntary context switch?
% Paul: This is for Tasks RCU, which is probably out of scope.  That said,
% This function is not -performing- a voluntary context switch, but rather
% checking to see if a voluntary context switch has been performed.  In
% the context of Tasks RCU, executing in user mode is equivalent to having
% performed a voluntary context switch -- either way, whatever was happening
% in the kernel beforehand is now well and truly done.
% Lihao: understand the differences from the tiny RCU version and how the tiny RCU one 
% is related to cond_resched in synchronize_sched
% Lihao: understand how it is related to wait_rcu_gp(call_rcu_sched) in synchronize_sched()

\subsubsection{Context-Switch Handling} \label{sec:context_switch}
The context-switch quiescent state is recorded by invoking
\co{rcu_note_context_switch()} from \co{__schedule()} (and, for the
benefit of virtualization, also from \co{rcu_virt_note_context_switch()}).
% http://lxr.free-electrons.com/source/kernel/sched/core.c#L3057
%
The \co{rcu_note_context_switch()} function invokes \co{rcu_sched_qs()}
to inform RCU of the context switch, which is a quiescent state of the CPU.
%Note that although quiescent states of RCU-bh include those of RCU-sched,
%\co{rcu_note_context_switch()} does not invoke \co{rcu_bh_qs()}.
%This could in theory starve RCU-bh grace periods if a given CPU spent all
%its time in the kernel in bottom-half-disabled regions, without any
%calls to \co{schedule()}.
%No part of the kernel currently does this, but should this pattern arise,
%RCU-bh's quiescent-state recording strategy will need to be revisited.

%\comment{Lihao: comment out the following preemptible RCU contents if we need space.}
%It also invokes \co{rcu_preempt_note_context_switch()} to add the current
%task to the \co{->blkd_tasks} list of the CPU's leaf \co{rcu_node}
%structure for context switches that occur within an RCU-preempt read-side
%critical section.
%To prevent this task from being re-added while within its current
%RCU-preempt read-side critical section,
%the first \co{rcu_preempt_note_context_switch()} sets the
%\co{->rcu_read_unlock_special.b.blocked} field in the task structure.
%
%However, if current task has already reported an RCU-preempt
%quiescent state for the current grace period, and if at least one
%other task is blocking that grace period on this \co{rcu_node}
%structure,
%the task should be added to the head of the \co{->blkd_tasks} list
%in order to avoid blocking that grace period.
%In this case, the \co{->gp_tasks} field
%will be non-\co{NULL} and the current CPU's bit will already be cleared
%from the \co{->qsmask} field.
%In all other cases, the task should be added to the tail of the
%\co{->blkd_tasks} list.
%If the task is blocking the current RCU-preempt grace period and
%\co{->gp_tasks} is \co{NULL}, then this is the first task on this
%leaf \co{rcu_node} structure to block the this grace period, and
%therefore \co{->gp_tasks} is set to reference the current task.
%This approach allows RCU to easily identify which tasks are blocking
%the current grace period.
%
%The \co{rcu_preempt_note_context_switch()} function also invokes
%\co{rcu_preempt_qs()} to note a quiescent state for the current CPU.
%Nevertheless, any tasks queued on the \co{->gp_tasks} segment of
%\co{->blkd_tasks} will continue to block the grace period.
%
%All of the tasks on the \co{->blkd_tasks} list dequeue themselves
%during the outermost \co{rcu_read_unlock()}.
%This of course introduces a race condition where a task is preempted
%while executing its outermost
%\co{rcu_read_unlock()}~\cite{PaulEMcKenney2011RCU3.0trainwreck}.
%\comment{Paul: This citation isn't all that important, so feel free to remove.}
%This race is detected by having \co{rcu_read_unlock()} set the \co{task_struct}
%structure's \co{->rcu_read_lock_nesting} field to a negative value.
%When \co{rcu_preempt_note_context_switch()} sees this negative value,
%it invokes \co{rcu_read_unlock_special()} to complete the dequeuing
%of the current task from the \co{->blkd_tasks} list.
%Interrupt disabling prevents further destructive races.
%% Lihao: if show code, add the following text
%% Recall that \co{rcu_read_unlock()} sets \co{rcu_read_lock_nesting} of its  
%%\co{task_struct} structure to \co{INT_MIN} before invoking 
%%\co{rcu_read_unlock_special()} 
%%(and to 0 after \co{rcu_read_unlock_special()} returns)
%% If the {task_struct}'s \co{rcu_read_unlock_special.s} is not equal to 0,
%% we have work left to do for \co{rcu_read_unlock_special}, which is then invoked.
%% http://lxr.free-electrons.com/source/kernel/rcu/tree_plugin.h#L267 
%% Lihao: may need to explain the code of rcu_read_unlock_special() 
%% Paul: I did a minimal explanation, which can be expanded if necessary.

\subsection{Обнаружение grace-периодов} \label{sec:grace_period}
Как только каждое вычислительное ядро прошло через устойчивое состояние,
grace-период RCU закончился.
Как было рассмотрено в разделе \ref{sec:data_structure},
Tree-RCU использует иерархию структур \co{rcu_node} для
управления информацией об устойчивых состояниях и grace-периодах.
Информация об устойчивых состояниях распространяется в направлении
от терминальных узлов к корню, а информация о grace-периодах ---
от корня к терминальным узлам.
%
%The dyntick-idle mechanisms used for idle CPUs and \co{nohz_full} userspace
%execution are out of scope for this research, as are RCU CPU stall warnings. The focus is instead
Будем рассматривать процесс обнаружения grace-периодов на загруженных
вычислительных ядрах, как показано на рисунке~\ref{fig:grace_period_state_diagram}.

\begin{figure}[tb]
\centering
\includegraphics[scale=0.25]{grace_period_state_diagram.pdf}
\caption{Диаграма состояний процесса обнаружения grace-периодов}
\label{fig:grace_period_state_diagram}
\end{figure}

\subsubsection{Softirq Handler for RCU} \label{sec:rcu_softirq}
RCU's busy-CPU grace period detection relies on the
\co{RCU_SOFTIRQ} handler function \co{rcu_process_callbacks()},
which is scheduled from the scheduling-clock interrupt.
This function first calls
\co{rcu_check_quiescent_state()} to report recent quiescent states
on the current CPU.
Then \co{rcu_process_callbacks()} starts a new grace period if needed,
and finally calls \co{invoke_rcu_callbacks()} to invoke any callbacks
whose grace period has already elapsed.

Function \co{rcu_check_quiescent_state()} first invokes \co{note_gp_changes()}
to update the CPU-local \co{rcu_data} structure to record the end of
previous grace periods and the beginning of new grace periods.
%, which are detected via differences in the \co{->completed} and
%\co{->gpnum} fields, respectively.
Any new values for these fields are copied from the leaf \co{rcu_node}
structure to the \co{rcu_data} structure.
If an old grace period has ended, \co{rcu_advance_cbs()} is invoked to
advance all callbacks, otherwise, \co{rcu_accelerate_cbs()} is invoked
to assign a grace period to any recently arrived callbacks.
If a new grace period has started, \co{->passed_quiesce} is set to zero,
and if in addition RCU is waiting for a quiescent state from this CPU,
\co{->qs_pending} is set to one, so that a new quiescent state will
be detected for the new grace period.
%
% Lihao: this is one of the two places where qs_pending gets updated in Tree RCU
% \comment{(Lihao: does it mean even this softirq is invoked because of a quiescent state of this CPU
%   (\co{rdp->passed_quiesce} is set to 1 in \co{rcu_check_callbacks} so \co{rcu_pending}
%   return 1), if for some reason gpnum in \co{rcu_data} of this CPU is one lag behind its parent
%   counterpart, this CPU needs to wait for its next quiescent to commit?
%   \url{http://lxr.free-electrons.com/source/kernel/rcu/tree.c#L1747}
% )}
% Paul: Yes.  I did look into immediately detecting quiescent states for
% RCU-preempt, but it didn't seem worth the coding contortions required.

Next,
\co{rcu_check_quiescent_state()} checks whether \co{->qs_pending} indicates
that RCU needs a quiescent state from this CPU.
If so, it checks whether \co{->passed_quiesce} indicates that this
CPU has in fact passed through a quiescent state.
If so, it invokes \co{rcu_report_qs_rdp()} to report that quiescent
state up the %\co{rcu_data} and \co{rcu_node}
combining tree.

The \co{rcu_report_qs_rdp()} function first verifies that the CPU has
in fact detected a legitimate quiescent state for the current grace period,
and under the protection of the leaf \co{rcu_node} structure's \co{->lock}.
If not, it resets quiescent-state detection and returns, thus ignoring
any redundant quiescent states belonging to some earlier grace period.
Otherwise, if the \co{->qsmask} field indicates that RCU needs to report a
quiescent state from this CPU, \co{rcu_accelerate_cbs()} is invoked to assign
a grace-period number to any new callbacks, and then \co{rcu_report_qs_rnp()}
is invoked to report the quiescent state to the \co{rcu_node} combining tree.

% \comment{(Lihao: did we just check this in \co{rcu_check_quiescent_state()}?
%   \url{http://lxr.free-electrons.com/source/kernel/rcu/tree.c#L2394}
% )}
% \comment{(Lihao: but did we just update \co{rdp->gpnum = rnp->gpnum} in \co{note_gp_changes()}...?
%   are they just double-checks or something may happen in between which I miss?
% )}
% Paul:  The code could probably be simplified.  The first step is to
% add assertions to verify the suspicions, and if the assertions don't
% trigger over a period of a year or so, simplify the code.  Sometimes
% the assertions have triggered, hence the caution.  ;-)
%
% \comment{(Lihao: what are \co{rcu_qs_ctr} and \co{rcu_qs_ctr_snap} used for?
%  \url{http://lxr.free-electrons.com/source/kernel/rcu/tree.c#L2341}
% )}
% Paul: These are used by cond_resched_rcu_qs(), which records a quiescent
% state for all flavors of RCU.
%
%
% \comment{(Lihao: can we use \co{rdp->qs_pending} in the following line of code since
%  it's also get updated in \co{note_gp_changes()}, right?
%  \url{http://lxr.free-electrons.com/source/kernel/rcu/tree.c#L2357}
%)}
%
% \comment{(Lihao: comments in \url{http://lxr.free-electrons.com/source/kernel/rcu/tree.c#L2272}
%   say if this CPU is the last one to pass through a quiescent state in the current grace period,
%   \co{rcu_report_qs_rsp()} is invoked to do the clean up and let \co{rcu_start_gp()}
%   start a new grace period if one is needed.~But where is \co{rcu_start_gp()} called in
%   \co{rcu_report_qs_rsp()}?
% )}
% Paul: This is done indirectly by waking up the RCU grace-period kthread.

The \co{rcu_report_qs_rnp()} function traverses up the \co{rcu_node} tree,
at each level holding the \co{rcu_node} structure's \co{->lock}.
At any level, if the child structure's \co{->qsmask} bit is already clear,
or if the \co{->gpnum} changes, traversal stops.
Otherwise, the child structure's bit is cleared from \co{->qsmask},
after which, if \co{->qsmask} is non-zero, %or if any tasks are queued on the
%\co{->blkd_tasks} list (which applies only to RCU-preempt),
traversal stops. Otherwise, traversal proceeds on to the parent \co{rcu_node} structure.
%If there is no parent (that is, the previous \co{rcu_node} structure was the root),
%the current grace period has completed. In that case, traversal stops and
Once the root is reached, traversal stops and \co{rcu_report_qs_rsp()} is
invoked to awaken the grace-period kthread (kernel thread).
The grace-period kthread will then clean up after the now-ended grace
period, and, if needed, start a new one.

\subsubsection{Grace-Period Kernel Thread} \label{sec:rcu_gp_kthread}
The RCU grace-period kthread invokes \co{rcu_gp_kthread()}, which
contains an infinite loop that initializes, waits for, and cleans up after
each grace period.

% rcu_gp_init()
When no grace period is required, the grace-period kthread
sets its \co{rcu_state} structure's \co{->flags} field to
\co{RCU_GP_WAIT_GPS}, and then
waits within an inner infinite loop for that structure's
\co{->gp_state} field to be set.
Once set, \co{rcu_gp_kthread()} invokes \co{rcu_gp_init()} to initialize
a new grace period, which
rechecks the \co{->gp_state} field under
the root \co{rcu_node} structure's \co{->lock}.
If the field is no longer set, \co{rcu_gp_init()} returns zero.
Otherwise, it
increments \co{rsp->gpnum} by 1 to record a new grace period number.
%
Finally, it performs a breadth-first traversal of the \co{rcu_node}
structures in the combining tree.
For each \co{rcu_node} structure \co{rnp},
% drop preemptible RCU contents
%we invoke \co{rcu_preempt_check_blocked_tasks()}, which responds to
%a non-empty list of blocked tasks by setting \co{rnp->gp_tasks} to
%\co{rnp->blkd_tasks.next}, so that those tasks block the new grace period.
%
we set the \co{rnp->qsmask} to indicate which children
must report quiescent states for the new grace period (Section
\ref{sec:rcu_node}), and set \co{rnp->gpnum} and \co{rnp->completed}
to their \co{rcu_state} counterparts.
%
If the \co{rcu_node} structure \co{rnp} is the parent of the current CPU's \co{rcu_data},
we invoke \co{__note_gp_changes()} to set up the CPU-local \co{rcu_data} state.
Other CPUs will invoke \co{__note_gp_changes()} after their next
scheduling-clock interrupt. %(Section~\ref{sec:timer_interrupt}).

%Note that other CPUs will access only the leaves of the hierarchy, thus seeing that
%no grace period is in progress, at least until the corresponding leaf node has been
%initialized. In addition, we have included CPU-hotplug operations since v4.1.

% Lihao: include this in PhD thesis; also look for 'preemptible RCU contents'
% rcu_gp_fqs()
%During a grace period, the grace-period kthread periodically
%calls \co{force_qs_rnp()} to detect idle and offline CPUs.
%For each such CPU, \co{force_qs_rnp()} invokes \co{rcu_report_qs_rnp()}
%to report a quiescent state on its behalf, thus avoiding the degraded
%energy efficiency that would be incurred should RCU awaken idle CPUs.
%CPUs that fail to report quiescent states will be sent an
%inter-processor interrupt (IPI), and if that fails, warning messages
%will be emitted.

% rcu_gp_cleanup()
To clean up after a grace period, \co{rcu_gp_kthread()}
calls \co{rcu_gp_cleanup()} after setting the \co{rcu_state} field \co{rsp->gp_state}
to \co{RCU_GP_CLEANUP}. After the function returns, \co{rsp->gp_state} is set to
\co{RCU_GP_CLEANED} to record the end of the old grace period.
%
Function \co{rcu_gp_cleanup()} performs a breadth-first traversal of
\co{rcu_node} combining-tree.
It first sets each \co{rcu_node} structure's \co{->completed} field
to the \co{rcu_state} structure's \co{->gpnum} field.
It then updates the current CPU's CPU-local \co{rcu_data} structure by
calling \co{__note_gp_changes()}.
For other CPUs, the update will take place when they handle the scheduling-clock
interrupts, in a fashion similar to \co{rcu_gp_init()}.
After the traversal, it marks the completion of the grace period by setting the
\co{rcu_state} structure's \co{->completed}
field to that structure's \co{->gpnum} field, and invokes
\co{rcu_advance_cbs()} to advance callbacks.
%
Finally, if another grace period is needed,
we set \co{rsp->gp_flags} to \co{RCU_GP_FLAG_INIT}.
Then in the next iteration of the outer loop, the grace-period kthread
will initialize a new grace period as discussed above.

% Lihao: understand how nodes in the tree sync with information for each grace period

% Lihao: Tree RCU starts a new grace by calling rcu_gp_kthread_wake() that wakes up
% the rcu_gp_kthread() kernel thread which does the clean up and invokes rcu_gp_init()
% to start a new grace period

% Lihao: other places that may start a new grace period
% 1. rcu_check_quiescent_state() calls note_gp_changes() that checks
% rcu_accelerate_cbs() or rcu_advance_cbs()
% 2. rcu_report_qs_rdp() by checking rcu_accelerate_cbs()
% 3. __rcu_process_callbacks() by checking cpu_needs_another_gp and rcu_start_gp()
% which in turn calls rcu_advance_cbs() and rcu_start_gp_advanced
% 4. __call_rcu_core by checking rcu_start_gp()
% 5. force_quiescent_state()

\section{Verification Scenario}
% \comment{Lihao: alternative titles: A Running Example? Putting Things Together?}

We use the example in Figure~\ref{fig:verify_rcu_gp} to demonstrate how the
different components of Tree RCU work together to guarantee that all
pre-existing read-side critical sections finish before RCU allows a grace
period to end.  This example will drive the verification, which will check
for violations of the assertion at this end of the code.

We focus on the implementation of the non-preemptible RCU-sched flavor.  We
further assume there are only two CPUs, and that CPU~0 executes function
\co{rcu_reader()} and CPU~1 executes \co{rcu_updater()}.  When the system
boots, the Linux kernel calls \co{rcu_init()} to initialize RCU, which
includes constructing the combining tree of \co{rcu_node} and \co{rcu_data}
structures via \co{rcu_init_geometry()} and initializing the fields of the
nodes in the tree for each RCU flavor via \co{rcu_init_one()}.  In our
example it will be a one-level tree that has one \co{rcu_node} structure as
root and two children that are \co{rcu_data} structures for each CPU. 
Function \co{rcu_spawn_gp_kthread()} is also called to initialize and spawn
the RCU grace-period kthread for each RCU flavor.

Referring again to Figure~\ref{fig:verify_rcu_gp},
suppose that \co{rcu_reader()} begins
execution on CPU~0 while \co{rcu_updater()} concurrently sets \co{x} to 1
and then invokes \co{synchronize_rcu()} on CPU~1.
As discussed in Section \ref{sec:api_impl}, \co{synchronize_rcu()}
invokes \co{wait_rcu_gp()}, which in turn registers an RCU callback
that will invoke \co{wakeme_after_rcu()} some time after \co{rcu_reader()}
exits its critical section.

However, this critical-section exit has no immediate effect.
Instead, a later context switch will invoke
\co{rcu_note_context_switch()}, which in turn invokes
\co{rcu_sched_qs()}, recording the quiescent state in the
CPU's \co{rcu_sched_data} structure's \co{->passed_quiesce} field.
Later, a scheduling-clock interrupt will invoke
\co{rcu_check_callbacks()}, which calls \co{rcu_pending()} and 
notes that the \co{->passed_quiesce} field is set.
This will cause \co{rcu_pending()} to return \co{true}, which
in turn causes \co{rcu_check_callbacks()} to invoke
\co{rcu_process_callbacks()}.
In its turn, \co{rcu_process_callbacks()} will invoke
\co{raise_softirq(RCU_SOFTIRQ)}, which,
once the CPU has interrupts, preemption, and
bottom halves enabled, %(Section \ref{sec:timer_interrupt}),
calls \co{rcu_process_callbacks()}.

As discussed in Section \ref{sec:rcu_softirq}, RCU's softirq handler function \co{rcu_process_callbacks()} 
first calls \co{rcu_check_quiescent_state()} to report any recent quiescent states on the 
current CPU (CPU~0). Then it checks whether the CPU~0 has passed a quiescent state. Since 
a quiescent state has been recorded for CPU~0, \co{rcu_report_qs_rnp()} is invoked to traversal
up the combining tree. It clears the first bit of the root \co{rcu_node} structure's \co{qsmask} 
field (recall that the RCU combining tree has only one level). Since the second bit for CPU~1 has 
not been cleared, the function returns.

Since \co{synchronize_rcu()} blocks in CPU~1, it will result in a context switch. 
This triggers a sequence of events similar to that described above for
CPU~1, which results in the clearing of the
second bit of the root \co{rcu_node} structure's \co{->qs_mask} field, the value of which is now 0, indicating the end of the current grace period.
CPU~1 therefore invokes \co{rcu_report_qs_rsp()} to 
awaken the grace-period kthread, 
which will clean up the ended grace period, and, if needed, 
start a new one (Section \ref{sec:rcu_gp_kthread}).

Lastly, \co{rcu_process_callbacks()} calls \co{invoke_rcu_callbacks()} to invoke any callbacks whose
grace period has already elapsed, for example, \co{wakeme_after_rcu()},
which will allow \co{synchronize_rcu()} to return.

%\comment{Lihao: When CPU~1 is waiting for \co{synchronize_rcu()} to return, how does it reach a 
%quiescent state? Is it via a scheduling-clock interrupt? What kind of quiescent states would it be?}
%\comment{Paul: Any number of possibilities.
%	First, \co{synchronize_rcu()} blocks, which results in a context
%	switch.
%	This context switch acts as a quiescent state, and a later
%	scheduling-clock interrupt would notice this and cause
%	\co{RCU_SOFTIRQ} to run, thus reporting the queiscent state
%	to the RCU core code.
%	Second, it is possible that there was nothing else for the
%	CPU to run, so that it went idle.
%	In this case, the grace-period kthread might notice that the CPU
%	was idle before the CPU got around to reporting the context switch
%	to the RCU core code.
%	Third, the context switch might result in a task running
%	in usermode.
%	In this case, a subsequent scheduling-clock interrupt causing
%	\co{RCU_SOFTIRQ} to run might
%	report the userspace-execution quiescent state to the RCU
%	core code.
%	Fourth, this might be a \co{CONFIG_NO_HZ_FULL} kernel.
%	In that case, the RCU grace-period kthread could note
%	the userspace execution in the same way that it might note
%	the idle loop.
%	Hey, you asked!
%}
% Lihao: understand when/where in the code \co{wakeme_after_rcu()} gets moved to the head of the 
% \co{->nxttail} to be invoked

              % Tree RCU Implementation
\section{Modeling RCU for CBMC} \label{sec:model_rcu}

The C Bounded Model Checker
(CBMC)\footnote{\url{http://www.cprover.org/cbmc/}} is a program analyzer
that implements bit-precise bounded model checking for C programs~\cite{KroeningTACAS04CBMC}.
CBMC can demonstrate violation of assertions in C programs, or prove their 
safety under a given loop unwinding bound.
%
It translates an input C program into a formula, which is then passed to a
modern SAT or SMT solver together with a constraint that specifies the set
of error states.  If the solver determines the formula to be satisfiable, an
error trace giving the exact sequence of events is extracted from the
satisfying assignment.
%
Recently, support has been added for verifying concurrent programs over a
wide range of memory models, including SC, TSO, and PSO~\cite{AlglaveCAV13}.

In the remainder of this section we describe how to construct a model from
the source code of the Tree RCU implementation in the Linux kernel version 4.3.6, 
which can be verified by CBMC.  Model construction entailed stubbing
out calls to other parts of the kernel, removing irrelevant functionality
(such as idle-CPU detection), removing irrelevant data (such as statistics),
and adding preprocessor directives to conditionally inject bugs (described
in Section~\ref{sec:bug_cases}). %These changes were carried out manually,
%but could potentially be scripted, or, better yet, carried out automatically
%by CBMC.
The Linux kernel environment and the majority of these changes to the source code 
are made through macros in separate files that can be reused across different versions 
of the Tree RCU implementation. The biggest change in the source files is to use arrays 
to model per-CPU data, which could potentially be scripted.
%
The resulting model is C code with assertions that can be also run as a
user program,
%which imposes constraints discussed in the following subsections,
%Lihao: it is not clear from the subsections below what constraints are imposed by the user program
%Paul: Your deletion above makes sense!
which provides important validation of the model itself.
%
%\comment{Lihao: move to the limitation section at the end}
%We model only the fundamental components of Tree RCU. For example, CPU hotplug, 
%dyntick-idle, quiescent-state forcing, grace-period expediting, and callback handling 
%are excluded. 
%\comment{Lihao: currently there isn't any code in the paper. Shall we 
% try to add some code snippets in this section? Can't think of any good example though 
% as it seems quite obvious to me from the text. Any suggestions?}
%\comment{Lihao: add code snippets in PhD thesis}

% \comment{Lihao: shall we write more on modeling RCU for CBMC, 
% which is the most technical part of the paper and deserves more space.}
% Paul: I added a small amount, but this close to the deadline I feel the
% need to be -very- careful.

\subsubsection*{Initialization}

Our model first invokes \co{rcu_init()} which in turn invokes:
(1)~\co{rcu_init_geometry()} to compute the \co{rcu_node} tree geometry;
(2)~\co{rcu_init_one} to initialize the \co{rcu_state}
structure; (3)~\co{rcu_cpu_notify()} to initialize each CPU's
\co{rcu_data} structure.
This boot initialization tunes the data-structure configuration
to match that of the specific hardware at hand.
For example, a large-system tree might resemble
Figure~\ref{fig:tree_rcu_hierarchy}, while
a small configuration has a single \co{rcu_node} ``tree''.
The model then calls \co{rcu_spawn_gp_kthread()} 
to spawn the grace-period kthreads discussed below.

\subsubsection*{Per-CPU Variables and State}

RCU uses per-CPU data to provide cache locality and to reduce 
contention and synchronization overhead.
For example, the per-CPU structure 
\co{rcu_data} records quiescent states 
and handles RCU callbacks (Section \ref{sec:rcu_data}). 
We model this per-CPU data as an array, indexed by CPU ID.

It is also necessary to model per-CPU state, including the currently
running task and whether or not interrupts are enabled.
Identifying the running task requires a (trivial)
model of the Linux-kernel scheduler, which
uses an integer array \co{cpu_lock}, indexed by CPU ID.
Each element of this array models an exclusive lock.
When a task schedules on a given CPU, it acquires the corresponding CPU lock,
and releases it when scheduling away.
We currently do not model preemption, so need model only voluntary context
switches.

A pair of integer arrays \co{local_irq_depth} and \co{irq_lock} is used
to model CPUs enabling and disabling interrupts.
Both arrays are indexed by CPU ID, with the first recording each CPU's
interrupt-disable nesting depth and the second recording whether
or not interrupts are disabled.
% Lihao: for now comment out the following paragraph to save space
%In theory, the \co{local_irq_depth} array suffices, but in practice
%the addition of the \co{irq_lock} enables better detection of errors,
%both in the code under test and in the model itself.

\subsubsection*{Update-Side API \co{synchronize_sched()}}
Because our model omits CPU hotplug and callback handling, we cannot use
Tree RCU's normal callback mechanisms to detect the end of a grace period.
We therefore use a global variable \co{wait_rcu_gp_flag}, which is
initialized to~1 in \co{wait_rcu_gp()} before the grace period.
Because \co{wait_rcu_gp()} blocks, it can result in a context switch,
the model invokes \co{rcu_note_context_switch()}, followed by a call 
to \co{rcu_process_callbacks()} to inform RCU of the resulting
quiescent state.
When the resulting quiescent states propagate to
the root of the combining tree, the grace-period kthread is awakened.
This kthread then invokes \co{rcu_gp_cleanup()}, the modeling of which 
is described below. Then \co{rcu_gp_cleanup()} calls \co{rcu_advance_cbs()}, 
which invokes \co{pass_rcu_gp()} to clear the \co{wait_rcu_gp_flag} flag.
The \co{__CPROVER_assume(wait_rcu_gp_flag == 0)}
~in~ \co{wait_rcu_gp()} prevents CBMC from continuing execution until
\co{wait_rcu_gp_flag} is equal to~0, thus modeling the needed grace-period
wait.
%To get this upstream into mainline Linux, additional stub functions 
%will be empty by default, and then different CBMC test scenarios can
%substitute different functionality on a scenario-by-scenario basis.

\subsubsection*{Scheduling-Clock Interrupt and Context Switch} \label{sec:model_irq}

The \co{rcu_check_callbacks()} function detects
idle execution, usermode execution, and to invoke RCU core processing
in response to state changes.
Because we model neither idle nor usermode execution,
%\co{rcu_check_callbacks()} responds only to state changes.
%Also, we do not model RCU diagnostics, so 
the only state changes are quiescent states and the beginnings and ends of grace periods.
We therefore dispense with \co{rcu_check_callbacks()} (Section \ref{sec:rcu_softirq}).
Instead, we directly call \co{rcu_note_context_switch()} just after
releasing a CPU, which in turn calls \co{rcu_sched_qs()} to record the
quiescent state.
Finally, we call \co{rcu_process_callbacks()},
which notes grace-period beginnings and ends and reports quiescent states
up RCU's combining tree.
%To model \co{cond_sched()} that explicitly performs a rescheduling to reduce latency in 
%the Linux kernel, we simply call \co{rcu_note_context_switch()} to note a context switch.
%\comment{Paul: cond_resched() only causes rcu_note_context_switch() to be invoked 
%when a real reschedule actually happened. Still OK to model it with rcu_note_context_switch(),
%though. But also OK to model it as nothingness, which might reduce the verification overhead a bit}

\subsubsection*{Grace-Period Kthread}
As discussed in Section \ref{sec:rcu_gp_kthread}, \co{rcu_gp_kthread()}
invokes \co{rcu_gp_init()}, \co{rcu_gp_fqs()}, and \co{rcu_gp_cleanup()}
to initialize, wait for, and clean up after each grace period, respectively.
To reduce the size of the formula generated by CBMC, instead of spawning a
separate thread, we directly call \co{rcu_gp_init()} from
\co{rcu_spawn_gp_kthread} and \co{rcu_gp_cleanup()} from
\co{rcu_report_qs_rsp()}.
Because we model neither idle nor usermode execution, we need not
call \co{rcu_gp_fqs()}.
%It stays in an infinite loop for each task until certain conditions 
%are met, then moves to the next one. As CBMC is a bounded model checker, it unwinds 
%loops of its input program to a fixed depth before performing verification. 
%We use CBMC's built-in primitive \co{__CPROVER_assume()} to model function 
%\co{wait_event_interruptible()} in each infinite loop. During the execution of CBMC, 
%it will only perform the next task when the condition in \co{__CPROVER_assume(condition)} 
%is true, even the loop is bounded.

\subsubsection*{Kernel Spin Locks}
CBMC's
\co{__CPROVER_atomic_begin()}, \co{__CPROVER_atomic_end()}, and
\co{__CPROVER_assume()} built-in primitives are used to construct atomic test-and-set
for \co{spinlock_t} and \co{raw_spinlock_t} acquisition and
atomic reset for release.
%
We use GCC atomic builtins for user-space execution:
\co{while (__sync_lock_test_and_set(lock, 1))} acquires a
lock and \co{__sync_lock_release(lock)} releases it.

% \comment{Lihao: final version of the paper should refer the model source code online}
% Done above.
% The full source code of our RCU model can be found online at \url{}.

\subsubsection*{Limitations}
We model only the fundamental components of Tree RCU, excluding, for example,
quiescent-state forcing, grace-period expediting, and callback handling.
%
In addition, we make the assumption that all CPUs are busy executing RCU related tasks. 
As a result, we do not model the following scenarios: 1.~CPU hotplug and dyntick-idle; 
2.~Thread-migration failure modes in the Linux kernel involving per-CPU variables; 
3.~RCU priority boosting. Moreover, we model scheduling-clock interrupts as 
direct function calls, which, as discussed later, results in failures to model one of 
the bug-injection scenarios. Lastly, the test harness we use only passes through a 
single grace period, so cannot detect failures involving multiple grace periods.
              % Tree RCU Model
\section{Experiments}

In this section we discuss our experiments verifying the Linux-kernel 
Tree RCU implementation. We first describe several bug-injection
scenarios used in the experiments. Next, we report results of user-space runs
of the RCU model. Then we describe how verify our RCU model using CBMC. 
Finally, we discuss the experimental results.
%and then compare the results obtained with both approaches.
We performed our experiments on a 64-bit machine running Linux 3.19.8
with eight Intel Xeon 3.07\,GHz cores and 48\,GB of memory.

\subsection{Bug-Injection Scenarios} \label{sec:bug_cases}
Because we model non-preemptible Tree RCU, each CPU runs exactly one RCU task
as a separate thread.
Upon completion, each task increments a global counter \co{thread_cnt},
enabling the parent thread to verify the completion of all RCU tasks
using a statement \co{__CPROVER_assume(thread_cnt == 2)}.
The base case uses the example in Figure~\ref{fig:verify_rcu_gp}, including
its assertion \co{assert(r2 == 0 || r1 == 1)}.
This assertion does not hold when
RCU's fundamental safety guarantee is violated:
read-side critical sections cannot span grace
periods~\cite{DesnoyersTPDS12UserRCU}.
%
We also verify a \emph{weak form} of liveness by inserting an \co{assert(0)} 
after the \co{__CPROVER_assume(thread_cnt == 2)} statement.
This assertion cannot hold, and so it will be violated if at least one grace period completes.
Such a ``verification failure" is in fact the expected behavior for a correct RCU implementation. 
On the other hand, if the assertion is not violated, grace periods never complete, which indicates a liveness bug.

To validate our verification, we also run CBMC with the bug-injection scenarios 
described below,\footnote{Source code is available: \url{http://lxr.free-electrons.com/source/kernel/rcu/?v=4.3}}
which are simplified versions of bugs encountered in actual practice.
Bugs~2--6 are liveness checks and thus use the
aforementioned \co{assert(0)}, and the remaining scenarios are
safety checks which thus use the base-case assertion in
Figure~\ref{fig:verify_rcu_gp}.
% \comment{Lihao: we might explain each bug scenario in greater detail 
% assuming the readers aren't familiar with the code.}

\paragraph*{Bug 1}
This bug-injection scenario makes
the RCU update-side primitive \co{synchronize_rcu()} return immediately 
(line 523 in \co{tree_plugin.h}).
With this injected bug, updaters never wait for readers, which should
result in a safety violation, thus preventing
Figure~\ref{fig:verify_rcu_gp}'s assertion from holding.

\paragraph*{Bug 2}
The key idea behind this bug-injection scenario is to prevent individual
CPUs from realizing that quiescent states are needed, thus preventing
them from recording quiescent states. As a result, it prevents grace periods
from completing.
Specifically, in function \co{rcu_gp_init()}, for each \co{rcu_node}
structure in the combining tree,
we set the field \co{rnp->qsmask} to 0 instead of \co{rnp->qsmaskinit} (line 1889 in \co{tree.c}). 
Then when \co{rcu_process_callbacks()} is called, \co{rcu_check_quiescent_state()} will invoke
\co{__note_gp_changes()} that sets \co{rdp->qs_pending} to 0. Thus,  
\co{rcu_check_quiescent_state()} will return without calling \co{rcu_report_qs_rdp()},
preventing grace periods from completing.
This liveness violation should fail to trigger a violation of the end-of-execution
\co{assert(0)}.

\paragraph*{Bug 3}
This bug-injection scenario is a variation of Bug 2, in
which each CPU remains aware that quiescent states are required, but
incorrectly believes that it has already reported a quiescent state
for the current grace period.
To accomplish this,
in \co{__note_gp_changes()}, we clear \co{rnp->qsmask} by adding a statement
\co{rnp->qsmask &= ~rdp->grpmask;} in the last \co{if} code block (line 1739 in \co{tree.c}). 
Then function \co{rcu_report_qs_rnp()} never walks up the \co{rcu_node}
tree, resulting in a liveness violation as in Bug~2.

\paragraph*{Bug 4}
This bug-injection scenario is an alternative code change that gets the
same effect as does Bug~2.
For this alternative,
in \co{__note_gp_changes()}, we set the \co{rdp->qs_pending} field to 0 directly 
(line 1749 in \co{tree.c}). This is a variant of Bug 2 and thus also
a liveness violation.

\paragraph*{Bug 5}
In this bug-injection scenario, CPUs remain aware of the need for
quiescent states.
However, CPUs are prevented from recording their
quiescent states, thus preventing grace periods from ever completing.
To accomplish this, we modify
function \co{rcu_sched_qs()} to return immediately (line 246 in \co{tree.c}),
so that quiescent states are not recorded.
Grace periods therefore never complete, which constitutes a liveness
violation similar to Bug~2.

\paragraph*{Bug 6}
In this bug-injection scenario, CPUs are aware of the need for quiescent
states, and they also record them locally.
However, they are prevented from reporting them up the \co{rcu_node}
tree, which again prevents grace periods from ever completing.
This bug modifies
function \co{rcu_report_qs_rnp()} to return immediately (line 2227 in \co{tree.c}). 
This prevents RCU from walking up the \co{rcu_node} tree, thus preventing grace
periods from ending.
This is again a liveness violation similar to Bug~2.

\paragraph*{Bug 7}
Where Bug~6 prevents quiescent states from being reported up the
\co{rcu_node} tree, this bug-injection scenario causes quiescent
states to be reported up the tree prematurely, before all the
CPUs covered by a given subtree have all reported quiescent states.
To this end,
in \co{rcu_report_qs_rnp()}, we remove the \co{if}-block checking for
\co{rnp->qsmask != 0 || rcu_preempt_blocked_readers_cgp(rnp)} (line 2251 in \co{tree.c}). 
Then the tree-walking process will not stop until it reaches the root, resulting in 
too-short grace periods.
This is therefore a safety violation similar to Bug~1.

\noindent Bugs~2 and~3 would result in a too-short grace period given quiescent-state forcing, 
but such forcing falls outside the scope of this paper.


\subsection{Validating the RCU Model in User-Space}\label{sec:test_model}

\begin{table*}[tbh]
\centering
\scalebox{0.85}{%
\begin{tabular}{|l|r|r|r|c|r|c|} \hline
\multicolumn{1}{|c|}{\textbf{Scenario}} &
%\multicolumn{1}{c|}{\textbf{\#Total Runs}} &
\multicolumn{1}{c|}{\textbf{\#Successful Runs}} &
\multicolumn{1}{c|}{\textbf{\#Failing Runs}} &
\multicolumn{1}{c|}{\textbf{\#Timeouts}} &
\multicolumn{1}{c|}{\textbf{Max VM}} &
\multicolumn{1}{c|}{\textbf{Runtime}} &
\multicolumn{1}{c|}{\textbf{Result}} \\ \hline
Prove              & 1,000 (100.0\%) &     0 (0.0\%)   &     0 (0.0\%)   & 361.5\,MB & 3mins\,51s  & Safe                      \\ \hline 
Prove-GP           &     0 (0.0\%)   & 1,000 (100.0\%) &     0 (0.0\%)   & 361.5\,MB & 5mins\,9s   & End of GP Reachable       \\ \hline 
Bug 1              &   461 (46.1\%)  &   539 (53.9\%)  &     0 (0.0\%)   & 361.5\,MB & 5mins\,26s  & Assertion Violated        \\ \hline
Bug 2              &     0 (0.0\%)   &     0 (0.0\%)   & 1,000 (100.0\%) & 361.5\,MB & 16h\,40mins & End of GP Unreachable     \\ \hline
Bug 3              &     0 (0.0\%)   &     0 (0.0\%)   & 1,000 (100.0\%) & 361.5\,MB & 16h\,40mins & End of GP Unreachable     \\ \hline
Bug 4              &     0 (0.0\%)   &     0 (0.0\%)   & 1,000 (100.0\%) & 361.5\,MB & 16h\,40mins & End of GP Unreachable     \\ \hline
Bug 5              &     0 (0.0\%)   &     0 (0.0\%)   & 1,000 (100.0\%) & 361.5\,MB & 16h\,40mins & End of GP Unreachable     \\ \hline
Bug 6              &     0 (0.0\%)   &     0 (0.0\%)   & 1,000 (100.0\%) & 361.5\,MB & 16h\,40mins & End of GP Unreachable     \\ \hline
Bug 7              &     0 (0.0\%)   &     0 (0.0\%)   & 1,000 (100.0\%) & 361.5\,MB & 16h\,40mins & \mkcol{Safe (Bug Missed)} \\ \hline
Bug 7 (2 readers)  &   758 (75.8\%)  &   242 (24.2\%)  &     0 (0.0\%)   & 369.7\,MB & 4mins\,40s  & Assertion Violated        \\ \hline 
\end{tabular}
}
\caption{Experimental Results of Testing the RCU Model in User-Space}
\label{tab:results_run}
\end{table*}


%Lihao: as pointed out in Paul's recent emails, the bug-injection scenarios
%are probably a bit unfair to CBMC as they are fairly deterministic. So testing is 
%able to return the expected result for all of them. For future work, it might be 
%interesting to try more subtle ones such as removing fences, and compare CBMC 
%with RCU's regression test suite.
%To answer the question how difficult is it to detect the bugs we have
%injected using testing, we have executed our model in user space.  
To validate our RCU model before performing verification using CBMC, 
we executed it in user space. We performed 1000 runs for each scenario 
in Section~\ref{sec:bug_cases} using a 60\,s timeout to wait for the end of a 
grace period and a random delay between 0 to 1\,s in the RCU reader task.

The results are reported in Table~\ref{tab:results_run}.  Column 1 %and 2 
gives the verification scenarios. %and the total number of runs for each one of them, respectively.
Scenario Prove tests our RCU model without bug injection. Scenario Prove-GP 
tests a weak form of liveness by replacing Figure~\ref{fig:verify_rcu_gp}'s 
assertion with \co{assert(0)} as described in Section~\ref{sec:bug_cases}.
The next three columns present the number and the percentage of successful, 
failing, and timeout runs, respectively. The following two columns give the 
maximum memory consumption and the total runtime. The last column explains 
the results. 

As expected, for scenario Prove, the user program ran to completion
successfully in all runs.  For Prove-GP, it was able to detect the end of a
grace period by triggering an assertion violation in all the runs.  For
Bug~1, an assertion violation was triggered in 559 out of 1000 runs.  
%Thus, Bug~1 can be described as an ``easy'' bug.  By contrast, 
For Bugs 2--6, the user program timed out in all the runs, 
thus a grace period did not complete. %these bugs are difficult for testing.  
%\comment{Lihao: note that timeout is the expected behaviour for bugs 2--6. 
%See Section \ref{sec:bug_cases} for description of each bug-injection scenario.}
For Bug 7 with one reader thread, the testing harness failed to
trigger an assertion violation.  However, we were able to observe a failure in
242 out of 1000 runs with two reader threads.

% \comment{Lihao: shall we move the following paragraph to a more appropriate place, 
% e.g.~the end of either the CBMC or the Results and Discussion section.}
% I moved it to the end of the CBMC modeling section.  Might need to be
% summarized in results and discussion, definitely in conclusion.

%\comment{Lihao: item 9 in Paul's email}
%\comment{Lihao: there is a blog post \url{https://lkml.org/lkml/2009/9/1/229} 
%discussing to remove the timer interrupt and make the Linux kernel completely event-driven. 
%Paul, is this happening?}
%\\ \comment{Paul: Most of the motivation for that has been addressed
%by the \co{NO_HZ_FULL} work.  If it does happen, some of the work that
%RCU currently does from the scheduling-clock interrupt will need to be
%instead done by other means, for example, by hrtimer handlers.}

\subsection{Getting CBMC to work on Tree RCU} \label{sec:cbmc_on_rcu}

We have found that getting CBMC to work on our RCU model is non-trivial due
to Tree RCU's complexity combined with CBMC's bit-precise verification.  In
fact, early attempts resulted in SAT formulas that were so large that CBMC
ran out of memory.  After the optimizations described in the remainder of
this section, the largest formula contained around 90 million variables
and 450 million clauses, which enabled CBMC to run to completion.
%
% \comment{Paul: Do we have formula sizes from before this effort for
% purposes of comparison? Lihao: we do, but with different versions of the code. 
% For example, one version has 168 million variables and 738 million clauses. 
% Not sure which one to use, similar to the situation in the next comment.}

First, instead of placing the scheduling-clock interrupt in its own thread,
we invoke functions \co{rcu_note_context_switch()} and
\co{rcu_process_callbacks()} directly, as described in
Section~\ref{sec:model_irq}.  Also, we invoke
\co{__note_gp_changes()} from \co{rcu_gp_init()} to notify each CPU of a new
grace period, instead of invoking \co{rcu_process_callbacks()}.
%
%These changes reduced CBMC's common-case runtime from more than 24 hours to less than 10 hours.
% \comment{Paul: Can we provide this sort of blow-by-blow result for all steps?
% If not, should we should remove this one?
% Or give a qualitative comparision: ``Before applying this optimization,
% CBMC ran out of memory before completing the formula.'' Lihao: let's do the latter.}

Second, the support for linked lists in CBMC version 5.4 is limited,
resulting in unreachable code in CBMC's symbolic execution. Thus, we
stubbed all the list-related code in our RCU model, including those for
callback handling.

Third, CBMC's structure-pointer and array encodings result in large formulas
and long formula-generation times.  Our focus on the RCU-sched flavor
allowed us to eliminate RCU-BH's data structures and trivialize the
\co{for_each_rcu_flavor()} flavor-traversal loops.  Our focus on small
numbers of CPUs meant that RCU-sched's \co{rcu_node} tree contained only a
root node, so we also trivialized the \co{rcu_for_each_node_breadth_first()}
loops traversing this tree.

Fourth, CBMC unwinds each loop to the depth specified in its command line
option \co{--unwind}, even when the actual loop depth is smaller.  This
unnecessarily increases formula size, especially for loops containing
intricate RCU code. Since loops in our model can be decided at compile time, 
we therefore used the command line option \co{--unwindset} to specify 
unwinding depths for each individual loop.

Finally, since our test harness only requires one \co{rcu_node} structure
and two \co{rcu_data} structures, we can use 32-bit encodings for \co{int},
\co{long}, and pointers by using the command line option \co{--ILP32}.  This
reduces CBMC's formula size by half compared to the 64-bit default.


\subsection{Results and Discussion}

Table~\ref{tab:results_cbmc} presents the results of our experiments
applying CBMC version~5.4 to verify our RCU model.
Scenario Prove verifies our RCU model without
bug injection over Sequential Consistency (SC).  We also exercise the model
over the weak memory models TSO and PSO in scenarios Prove-TSO and Prove-PSO, 
respectively.  Scenario Prove-GP performs the same reachability check as in 
Section~\ref{sec:test_model} over SC.
%by replacing Figure~\ref{fig:verify_rcu_gp}'s assertion with \co{assert(0)} 
%as described in Section~\ref{sec:bug_cases}, which serves as a reachability check. 
We perform the same reachability verification over TSO
and PSO in scenarios Prove-GP-TSO and Prove-GP-PSO, respectively.  Scenarios
Bug~1--7 are the bug-injection scenarios discussed in
Section~\ref{sec:bug_cases}, and are verified over SC, TSO and PSO.  Columns
2--4 give the number of constraints (symbolic program expressions and
partial orders), variables, and clauses of the generated
formula.  The next three columns give the maximum (virtual) memory
consumption, solver runtime, and total runtime of our experiments.  The
final column gives the verification result.

% \comment{Lihao: the footnote in results.tex for bug 7 with 2 reader threads doesn't display?}
% I changed the footnote to text following the table.
% \comment{Lihao: we only use this more powerful machine to run bug 7 with 2 readers, 
% hence footnote seems more appropriate.}
% OK, I created a special within-table footnote.  ;-)

\begin{table*}[tbh]
\centering
\scalebox{0.85}{%
\begin{tabular}{|l|c|c|r|r|r|r|c|} \hline
\multicolumn{1}{|c|}{\textbf{Scenario}} &
\multicolumn{1}{c|}{\textbf{\#Constraints}} &
\multicolumn{1}{c|}{\textbf{\#Variables}} &
\multicolumn{1}{c|}{\textbf{\#Clauses}} &
\multicolumn{1}{c|}{\textbf{Max VM}} &
\multicolumn{1}{c|}{\textbf{Solver Time}} &
\multicolumn{1}{c|}{\textbf{Total Time}} &
\multicolumn{1}{c|}{\textbf{Result}} \\ \hline
Prove             &  5,279,600 & 30,085,337 & 149,758,548 & 23.27\,GB &  9h\,24mins &  9h\,36mins & Safe \\ \hline
Prove-TSO         &  5,646,959 & 42,042,386 & 210,708,442 & 34.00\,GB & 10h\,51mins & 11h\,4mins  & Safe \\ \hline
Prove-PSO         &  5,617,154 & 41,327,066 & 207,042,629 & 33.76\,GB & 11h\,23mins & 11h\,36mins & Safe \\ \hline
Prove-GP          &  5,476,540 & 30,655,428 & 152,743,545 & 23.90\,GB &  3h\,52mins &  4h\,5mins  & End of GP Reachable \\ \hline
Prove-GP-TSO      &  5,646,940 & 42,041,740 & 210,705,615 & 34.00\,GB & 13h\,1mins  & 13h\,14mins & End of GP Reachable \\ \hline
Prove-GP-PSO      &  5,617,135 & 41,326,420 & 207,039,802 & 33.76\,GB &  8h\,24mins &  8h\,37mins & End of GP Reachable \\ \hline
Bug 1             &  1,343,449 & 11,719,966 &  56,027,980 &  8.24\,GB &      31mins &      33mins & Assertion Violated \\ \hline
Bug 1-TSO         &  1,540,645 & 17,120,555 &  83,392,397 & 12.60\,GB &      53mins &      56mins & Assertion Violated \\ \hline
Bug 1-PSO         &  1,514,657 & 16,548,819 &  80,481,851 & 12.42\,GB &      46mins &      48mins & Assertion Violated \\ \hline
Bug 2             &  5,279,584 & 30,056,615 & 149,643,492 & 23.26\,GB &  4h\,25mins &  4h\,37mins & End of GP Unreachable \\ \hline
Bug 2-TSO         &  5,646,940 & 42,013,372 & 210,592,015 & 34.01\,GB &  9h\,57mins & 10h\,10mins & End of GP Unreachable \\ \hline
Bug 2-PSO         &  5,617,135 & 41,298,052 & 206,926,202 & 33.75\,GB &  8h\,51mins &  9h\,4mins & End of GP Unreachable \\ \hline
Bug 3             &  6,374,373 & 34,856,577 & 174,131,331 & 28.04\,GB &  7h\,11mins &  7h\,25mins & End of GP Unreachable \\ \hline
Bug 3-TSO         &  6,805,631 & 48,788,433 & 245,157,184 & 41.18\,GB & 19h\,40mins & 19h\,55mins & End of GP Unreachable \\ \hline
Bug 3-PSO         &  6,773,763 & 48,023,601 & 241,237,629 & 40.95\,GB & 19h\,19mins & 19h\,35mins & End of GP Unreachable \\ \hline
Bug 4             &  4,847,980 & 27,804,363 & 138,197,043 & 22.18\,GB &  4h\,3mins  &  4h\,14mins & End of GP Unreachable \\ \hline
Bug 4-TSO         &  5,170,928 & 38,480,891 & 192,605,939 & 31.49\,GB &  8h\,18mins &  8h\,30mins & End of GP Unreachable \\ \hline
Bug 4-PSO         &  5,141,123 & 37,765,571 & 188,940,126 & 31.27\,GB &  8h\,14mins &  8h\,26mins & End of GP Unreachable \\ \hline
Bug 5             &  5,161,874 & 29,510,828 & 146,787,005 & 23.02\,GB &  4h\,6mins  &  4h\,18mins & End of GP Unreachable \\ \hline
Bug 5-TSO         &  5,522,168 & 41,239,083 & 206,569,643 & 33.65\,GB &  5h\,46mins &  5h\,59mins & End of GP Unreachable \\ \hline
Bug 5-PSO         &  5,492,607 & 40,529,619 & 202,933,839 & 33.04\,GB &  5h\,42mins &  5h\,55mins & End of GP Unreachable \\ \hline
Bug 6             &  1,410,495 & 13,165,176 &  63,302,559 &  9.03\,GB &      19mins &      21mins & End of GP Unreachable \\ \hline
Bug 6-TSO         &  1,541,937 & 17,286,058 &  84,131,818 & 12.59\,GB &  1h\,32mins &  1h\,33mins & End of GP Unreachable \\ \hline
Bug 6-PSO         &  1,518,307 & 16,766,198 &  81,485,361 & 12.44\,GB &  1h\,22mins &  1h\,24mins & End of GP Unreachable \\ \hline
Bug 7             &  5,022,249 & 29,242,760 & 145,389,516 & 22.87\,GB &  8h\,48mins &  9h         & \mkcol{Safe (Bug Missed)} \\ \hline
Bug 7-TSO         &  5,201,744 & 40,139,251 & 200,857,404 & 31.93\,GB & 11h\,6mins  & 11h\,18mins & Assertion Violated \\ \hline
Bug 7-PSO         &  5,172,720 & 39,442,675 & 197,287,644 & 31.71\,GB & 11h\,32mins & 11h\,44mins & Assertion Violated \\ \hline
Bug 7 (2 readers) $^*$
                  & 15,165,557 & 71,205,400 & 359,021,922 & 59.07\,GB & 19h\,2mins  & 19h\,40mins & Assertion Violated \\ \hline
Bug 7-TSO (2 readers) $^*$
                  & 15,691,102 & 90,444,903 & 456,973,933 & 74.80\,GB & 78h\,12mins & 78h\,53mins & Assertion Violated \\ \hline
Bug 7-PSO (2 readers) $^*$
                  & 15,647,504 & 89,398,551 & 451,611,664 & 74.51\,GB & 84h\,21mins & 85h\,2mins  & \mkcol{Solver Out of Memory} \\ \hline
\end{tabular}
}
\vspace*{0.03cm} \\
\raggedright
\centering\footnotesize * This experiment was performed on a 64-bit machine running Linux 3.19.8 with twelve Intel Xeon 2.40\,GHz cores and 96\,GB of main memory
\vspace*{0.05cm}
\caption{Experimental Results of CBMC}
\label{tab:results_cbmc}
\end{table*}


Since Tree RCU's implementation in the Linux kernel is sophisticated, its test suite
is non-trivial~\cite{PaulMcKenney2005rcutorture}, comprising several thousand
lines of code.
% \comment{Paul, any reference I can cite for Tree RCU testing?} 
Therefore, it comes as little surprise that its verification is 
challenging.

In our experiments, CBMC returned all the expected results except
for Bug~7, for which it failed to report a violation of the
assertion \co{assert(r2 == 0 || r1 == 1)} with one RCU reader thread running
over SC.  This failure was due to the approximation of the
scheduling-clock interrupt by a direct function call,
as described in Section~\ref{sec:model_rcu}. 
% \comment{Lihao: should it be 'under-approximation'?  It can happen more
% frequently than the number of calls in our model.  I suggest we just use the
% word 'approximation'.}
However, CBMC did report a violation of the assertion
either when two RCU reader threads were present or when run over
TSO or PSO.
All of these cases decrease determinism,
which in turn more faithfully model non-deterministic
scheduling-clock interrupts, allowing the assertion to be violated.
% \comment{Lihao: shall we briefly explain why it's
% possible for TSO and PSO to detect bug 7 with one reader thread?}
%\comment{Lihao: briefly compare CBMC's results with the user-space program.}

CBMC took more than 9 hours to verify our model over SC (scenario Prove). 
The resulting SAT formulas have more than 5m
constraints, 30m variables and 149m clauses, and occupy
23\,GB of memory.  The formulas for scenarios Prove-TSO and
Prove-PSO are about
40\% larger than the scenario Prove.  They have more than 40m
variables and 200m clauses, and took more than 11 hours and 33\,GB
memory to solve.
Although this verification consumed considerable memory
and CPU, it verified all possible executions and
reorderings permitted by TSO and PSO,
a tiny subset of which are reached by
the \co{rcutorture} test suite.

CBMC proved that grace periods can end (i.e., \co{assert(0)} is
violated), over SC (Prove-GP), TSO (Prove-GP-TSO), and PSO
(Prove-GP-PSO).  The sizes of resulting formulas and memory consumption
are similar to those of the three Prove scenarios.
However, it took CBMC only about 4, 13, and 8.5 hours to find an
violation of \co{assert(0)} in Prove-GP, Prove-GP-TSO, and Prove-GP-PSO, respectively.

For the bug-injection scenarios described in Section~\ref{sec:bug_cases}, CBMC
was able to return the expected results in all scenarios over SC except for Bug~7,
as noted earlier.
The formula size varies from scenarios to
scenarios, with 27m--35m variables and 138m--174m
clauses.  The runtime was 4--9 hours and memory consumption exceeded
22\,GB.  The exceptions are Bugs~1 and 6, which have fewer than
14m variables and 64m clauses, and took less than 35 mins and
about 9\,GB of memory to solve.  This reduction was due to the large amount
of code removed by the bug injections in these scenarios.

% Lihao: include this in PhD thesis
\begin{figure}[tbp]
\centering
\captionsetup{justification=centering}
\begin{tikzpicture}
\scriptsize
\begin{axis}[
  ybar,
  bar width=0.15cm,
  height=4.5cm,
  width=9.5cm,
  axis lines*=left, % remove lines in the background
  ymode=log,
  %ylabel=Number of constraints,
  symbolic x coords={Prove, Prove-GP, Bug 1, Bug 2, Bug 3, 
                     Bug 4, Bug 5, Bug 6, Bug 7, Bug 7-2R,
                    },
  xtick=data,
  %nodes near coords, % numbers displayed above the bars
  %every node near coord/.append style={font=\small, rotate=90, anchor=west},
  xticklabel style={
    inner sep=0pt,
    anchor=north east,
    rotate=45
  },
  %enlargelimits=0.15,
  enlarge y limits=0.15, % space relative to the height of the plot
  enlarge x limits=0.05, % space relative to the width of the plot
  legend style={
    %anchor=north, at={(0.5, -0.9)}, % legend location
    legend pos=north west,
    legend columns=-1,
    font=\scriptsize},
]

\addplot % SC
  coordinates {(Prove, 5279600) (Prove-GP, 5476540)
               (Bug 1, 1343449) (Bug 2, 5279584) (Bug 3, 6374373)
               (Bug 4, 4847980) (Bug 5, 5161874) (Bug 6, 1410495)
               (Bug 7, 5022249) (Bug 7-2R, 15165557)
              };

\addplot % TSO 
  coordinates {(Prove, 5646959) (Prove-GP, 5646940)
               (Bug 1, 1540645) (Bug 2, 5646940) (Bug 3, 6805631)
               (Bug 4, 5170928) (Bug 5, 5522168) (Bug 6, 1541937)
               (Bug 7, 5201744) (Bug 7-2R, 15691102)
              };


\addplot % PSO
  coordinates {(Prove, 5617154) (Prove-GP, 5617135)
               (Bug 1, 1514657) (Bug 2, 5617135) (Bug 3, 6773763)
               (Bug 4, 5141123) (Bug 5, 5492607) (Bug 6, 1518307)
               (Bug 7, 5172720) (Bug 7-2R, 15647504)
              };

\legend{SC, TSO, PSO}
\end{axis}
\end{tikzpicture}

\caption{Number of Constraints in the SAT Formulas}
\label{fig:barchart_sat_constr}
\end{figure}

\begin{figure}[tbp]
\centering
\captionsetup{justification=centering}
\begin{tikzpicture}
\scriptsize
\begin{axis}[
  ybar,
  bar width=0.15cm,
  height=4.5cm,
  width=9.5cm,
  axis lines*=left, % remove lines in the background
  ymode=log,
  %ylabel=Number of variables,
  symbolic x coords={Prove, Prove-GP, Bug 1, Bug 2, Bug 3, 
                     Bug 4, Bug 5, Bug 6, Bug 7, Bug 7-2R,
                    },
  xtick=data,
  %nodes near coords, % numbers displayed above the bars
  %every node near coord/.append style={font=\small, rotate=90, anchor=west},
  xticklabel style={
    inner sep=0pt,
    anchor=north east,
    rotate=45
  },
  %enlargelimits=0.15,
  enlarge y limits=0.15, % space relative to the height of the plot
  enlarge x limits=0.05, % space relative to the width of the plot
  legend style={
    %anchor=north, at={(0.5, -0.9)}, % legend location
    legend pos=north west,
    legend columns=-1,
    font=\scriptsize},
]

\addplot % SC
  coordinates {(Prove, 30085337) (Prove-GP, 30655428)
               (Bug 1, 11719966) (Bug 2, 30056615) (Bug 3, 34856577)
               (Bug 4, 27804363) (Bug 5, 29510828) (Bug 6, 13165176)
               (Bug 7, 29242760) (Bug 7-2R, 71205400)
              };

\addplot % TSO 
  coordinates {(Prove, 42042386) (Prove-GP, 42041740)
               (Bug 1, 17120555) (Bug 2, 42013372) (Bug 3, 48788433)
               (Bug 4, 38480891) (Bug 5, 41239083) (Bug 6, 17286058)
               (Bug 7, 40139251) (Bug 7-2R, 90444903)
              };


\addplot % PSO
  coordinates {(Prove, 41327066) (Prove-GP, 41326420)
               (Bug 1, 16548819) (Bug 2, 41298052) (Bug 3, 48023601)
               (Bug 4, 37765571) (Bug 5, 40529619) (Bug 6, 16766198)
               (Bug 7, 39442675) (Bug 7-2R, 89398551)
              };

\legend{SC, TSO, PSO}
\end{axis}
\end{tikzpicture}

\caption{Number of Variables in the SAT Formulas}
\label{fig:barchart_sat_var}
\end{figure}

\begin{figure}[tbp]
\centering
\captionsetup{justification=centering}
\begin{tikzpicture}
\scriptsize
\begin{axis}[
  ybar,
  bar width=0.15cm,
  height=4.5cm,
  width=9.5cm,
  axis lines*=left, % remove lines in the background
  ymode=log,
  %ylabel=Number of clauses,
  symbolic x coords={Prove, Prove-GP, Bug 1, Bug 2, Bug 3, 
                     Bug 4, Bug 5, Bug 6, Bug 7, Bug 7-2R,
                    },
  xtick=data,
  %nodes near coords, % numbers displayed above the bars
  %every node near coord/.append style={font=\small, rotate=90, anchor=west},
  xticklabel style={
    inner sep=0pt,
    anchor=north east,
    rotate=45
  },
  %enlargelimits=0.15,
  enlarge y limits=0.15, % space relative to the height of the plot
  enlarge x limits=0.05, % space relative to the width of the plot
  legend style={
    %anchor=north, at={(0.5, -0.9)}, % legend location
    legend pos=north west,
    legend columns=-1,
    font=\scriptsize},
]

\addplot % SC
  coordinates {(Prove, 149758548) (Prove-GP, 152743545)
               (Bug 1, 56027980) (Bug 2, 149643492) (Bug 3, 174131331)
               (Bug 4, 138197043) (Bug 5, 146787005) (Bug 6, 63302559)
               (Bug 7, 145389516) (Bug 7-2R, 359021922)
              };

\addplot % TSO 
  coordinates {(Prove, 210708442) (Prove-GP, 210705615)
               (Bug 1, 83392397) (Bug 2, 210592015) (Bug 3, 245157184)
               (Bug 4, 192605939) (Bug 5, 206569643) (Bug 6, 84131818)
               (Bug 7, 200857404) (Bug 7-2R, 456973933)
              };

\addplot % PSO
  coordinates {(Prove, 207042629) (Prove-GP, 207039802)
               (Bug 1, 80481851) (Bug 2, 206926202) (Bug 3, 241237629)
               (Bug 4, 188940126) (Bug 5, 202933839) (Bug 6, 81485361)
               (Bug 7, 197287644) (Bug 7-2R, 451611664)
              };

\legend{SC, TSO, PSO}
\end{axis}
\end{tikzpicture}

\caption{Number of Clauses in the SAT Formulas}
\label{fig:barchart_sat_clause}
\end{figure}

\begin{figure}[tbp]
\centering
\captionsetup{justification=centering}
\begin{tikzpicture}
\scriptsize
\begin{axis}[
  ybar,
  bar width=0.15cm,
  height=4.5cm,
  width=9.5cm,
  axis lines*=left, % remove lines in the background
  ymode=log,
  %ylabel=Total runtime (seconds),
  symbolic x coords={Prove, Prove-GP, Bug 1, Bug 2, Bug 3, 
                     Bug 4, Bug 5, Bug 6, Bug 7, Bug 7-2R,
                    },
  xtick=data,
  %nodes near coords, % numbers displayed above the bars
  %every node near coord/.append style={font=\small, rotate=90, anchor=west},
  xticklabel style={
    inner sep=0pt,
    anchor=north east,
    rotate=45
  },
  %enlargelimits=0.15,
  enlarge y limits=0.15, % space relative to the height of the plot
  enlarge x limits=0.05, % space relative to the width of the plot
  legend style={
    %anchor=north, at={(0.5, -0.9)}, % legend location
    legend pos=north west,
    legend columns=-1,
    font=\scriptsize},
]

\addplot % SC
  coordinates {(Prove, 34570.5) (Prove-GP, 14698.4)
               (Bug 1, 2002.53) (Bug 2, 16644.8) (Bug 3, 26716.8)
               (Bug 4, 15231.3) (Bug 5, 15490.8) (Bug 6, 1255.06)
               (Bug 7, 32373.1) (Bug 7-2R, 70822.7)
              };

\addplot % TSO 
  coordinates {(Prove, 39820.1) (Prove-GP, 47643.6)
               (Bug 1, 3360.95) (Bug 2, 36608.3) (Bug 3, 71708.5)
               (Bug 4, 30599.2) (Bug 5, 21550.4) (Bug 6, 5608.83)
               (Bug 7, 40667.7) (Bug 7-2R, 283950)
              };


\addplot % PSO
  coordinates {(Prove, 41754.8) (Prove-GP, 31002.2)
               (Bug 1, 2902.64) (Bug 2, 32616.8) (Bug 3, 70477.1)
               (Bug 4, 30355) (Bug 5, 21296.9) (Bug 6, 5040.28)
               (Bug 7, 42226) (Bug 7-2R, 306133)
              };

\legend{SC, TSO, PSO}
\end{axis}
\end{tikzpicture}

\caption{Total Runtime in Seconds}
\label{fig:barchart_runtime}
\end{figure}

\begin{figure}[tbp]
\centering
\captionsetup{justification=centering}
\begin{tikzpicture}
\scriptsize
\begin{axis}[
  ybar,
  bar width=0.15cm,
  height=4.5cm,
  width=9.5cm,
  axis lines*=left, % remove lines in the background
  ymode=log,
  %ylabel=Maximum memory consumption (gigabytes),
  symbolic x coords={Prove, Prove-GP, Bug 1, Bug 2, Bug 3, 
                     Bug 4, Bug 5, Bug 6, Bug 7, Bug 7-2R,
                    },
  xtick=data,
  %nodes near coords, % numbers displayed above the bars
  %every node near coord/.append style={font=\small, rotate=90, anchor=west},
  xticklabel style={
    inner sep=0pt,
    anchor=north east,
    rotate=45
  },
  %enlargelimits=0.15,
  enlarge y limits=0.15, % space relative to the height of the plot
  enlarge x limits=0.05, % space relative to the width of the plot
  legend style={
    %anchor=north, at={(0.5, -0.9)}, % legend location
    legend pos=north west,
    legend columns=-1,
    font=\scriptsize},
]

\addplot % SC
  coordinates {(Prove, 23.27) (Prove-GP, 23.90)
               (Bug 1, 8.24) (Bug 2, 23.26) (Bug 3, 28.04)
               (Bug 4, 22.18) (Bug 5, 23.02) (Bug 6, 9.03)
               (Bug 7, 22.87) (Bug 7-2R, 59.07)
              };

\addplot % TSO 
  coordinates {(Prove, 34.00) (Prove-GP, 34.00)
               (Bug 1, 12.60) (Bug 2, 34.01) (Bug 3, 41.18)
               (Bug 4, 31.49) (Bug 5, 33.65) (Bug 6, 12.59)
               (Bug 7, 31.93) (Bug 7-2R, 74.80)
              };


\addplot % PSO
  coordinates {(Prove, 33.76) (Prove-GP, 33.76)
               (Bug 1, 12.42) (Bug 2, 33.75) (Bug 3, 40.95)
               (Bug 4, 31.27) (Bug 5, 33.04) (Bug 6, 12.44)
               (Bug 7, 31.71) (Bug 7-2R, 74.51)
              };

\legend{SC, TSO, PSO}
\end{axis}
\end{tikzpicture}

\caption{Maximum Memory Consumption in Gigabytes}
\label{fig:barchart_memory}
\end{figure}

Figures~\ref{fig:barchart_sat_constr}--\ref{fig:barchart_sat_var} 
compare the formula size between SC, TSO and TSO. Comparison of 
runtime and memory can be found in Figures~\ref{fig:barchart_runtime} 
and \ref{fig:barchart_memory}. As we can see,
%Table~\ref{tab:results_cbmc} also shows that
the runtime and memory overhead for the TSO and PSO variants of a given experiment 
are quite similar.
The overheads of TSO
are slightly higher than those of PSO in all bug-injection scenarios except
for Bug 7 on which PSO had longer runtime.
However, the overhead of TSO and PSO is significantly larger than that of SC,
with up to 340\% (Bug 6 runtime) and 50\% (Bug 1 memory) increases.
The runtime was 5--19 hours and memory consumption exceeded
31\,GB in all scenarios except Bug 1 and 6.  The numbers of variables and
clauses are 37m--49m and 188m--245m, respectively, around 130\%
greater than SC.

The two-reader variant of Bug~7 has by far the longest
runtime, consuming more than 19~hours and
78~hours over SC and TSO, respectively, comparing to 9~hours and 11~hours
with one reader.  It also consumed about 75\,GB memory, more than double
the one-reader variant.  For PSO, with two reader threads CBMC's solver ran
out of memory after 85~hours whereas with one reader it completed
in less than 12~hours.  The increased overhead
is due to the
additional RCU reader's call to
\co{rcu_process_callbacks()}.
This in turn results in
more than a 125\% increase in the number of constraints, variables, and
clauses.  For example, the two-reader TSO formula has
triple the constraints and double the variables and clauses
of the one-reader case.
           % Experimental Studies
\section{Related Work}
%Formal verification has been applied to various aspects of RCU design 
%and implementation. 
%
McKenney applied the SPIN model checker to verify RCU's \co{NO_HZ_FULL_SYSIDLE}
functionality \cite{VerificationChallenges},
and interactions between dyntick-idle and 
non-maskable interrupts~\cite{ValDyntickNMI}. 
%
Desnoyers et al.~\cite{DesnoyersOSR13} propose a virtual architecture 
to model out-of-order memory accesses and instruction scheduling.
User-level RCU \cite{DesnoyersTPDS12UserRCU} is modeled and verified 
in the proposed architecture using the SPIN model checker. 

These efforts require an error-prone translation from C to SPIN's 
modeling language, and therefore are not appropriate for regression testing. 
By contrast, our work constructs an RCU model directly from its 
source code from the Linux kernel, and verifies it using automated 
verification tool. 

Alglave et al.~\cite{AlglaveCAV13} introduce a symbolic encoding 
for verifying concurrent software over a range of memory models 
including SC, TSO and PSO.
They implement the encoding in the CBMC bounded model checker and 
use the tool to verify \co{rcu_assign_pointer()} and \co{rcu_dereference()}.

McKenney used CBMC to verify Tiny
RCU~\cite{VerificationChallenges}, a trivial Linux-kernel
RCU implementation for uni-core systems.

Groce et al.~\cite{GroceASE15RCU} introduce a falsification-driven
verification methodology that is based on a variation of mutation 
testing. By using CBMC, they were able to find two holes in 
\co{rcutorture}--RCU's stress testing suite, one of which was hiding 
a real bug in Tiny RCU.
Further work on real hardware identified two more
\co{rcutorture} holes, one of which was hiding a real bug in Tasks
RCU~\cite{JonathanCorbet2014RCU-tasks} and the other of which was hiding
a minor performance bug in Tree RCU.
% \comment{Lihao: shall we add a reference to RCU-tasks. I found this one:
% https://lwn.net/Articles/607117/}
% Paul: Done!

In this work, we use CBMC to verify the implementation of Linux-kernel
Tree RCU for multi-core systems, which is more complex
and sophisticated, over SC, TSO, and PSO.

Gotsman et al.~\cite{YangESOP13RCU} use a extended concurrent separation logic 
%extended with temporal operators 
to formalise the concept 
of grace period and prove an abstract implementation of RCU over SC.
%
Tassarotti et al.~\cite{DreyerPLDI15RCU} use GPS, a recently 
developed program logic for the C/C++11 memory model, to carry out 
a formal proof of a simple implementation of user-level 
RCU for a singly-linked list assuming ``release-acquire'' semantics, 
which is weaker than SC but stronger than memory models used by
real-world RCU implementations.
%
These formal proofs were performed manually on simple implementations 
of RCU. By contrast, our work applies an automated verification tool with a 
test harness to verify the grace-period property of a real-world
implementation of RCU over SC, TSO, and PSO.

Formal verification has started to make its way into real-world practice 
of verifying large non-trivial code bases. Cal\-ca\-gno et al.~\cite{CalcagnoNASA15} 
describe integrating a static-analysis tool
into Facebook's software development cycle.
We believe that our work is an 
important step towards integration of verification into Linux-kernel RCU's 
regression test suite.
          % Related Work
\section{Conclusion}
This paper overviews the implementation of Tree RCU in the 
Linux Kernel, and describes how to construct a model directly from
its source code. It then shows how to use the CBMC model checker to 
verify a significant part of the Tree RCU implementation automatically, 
which to the best of our knowledge is unprecedented.
%
This work demonstrates that RCU is a rich example to drive research:
it is small enough to provide models that can just
barely be verified by existing tools, but it also has sufficient concurrency
and complexity to drive significant advances in techniques and tooling.

For future work, we plan to 
add quiescent-state forcing and grace-period expediting into our model 
and verify their safety and liveness properties, using more sophisticated 
test harnesses that pass through multiple grace periods and operate on 
a larger tree structure.
%
We also plan to model and verify the preemptible version of Tree RCU, 
which we expect to be quite challenging. Moreover, there is much fertile ground 
verifying uses of RCU in the Linux kernel, for example, the Virtual File
System (VFS).
%
%In addition, we plan to remove some of the limitations of our model
%listed at the end of Section~\ref{sec:model_rcu}.

There are also potential improvements for CBMC to better support future
RCU verification efforts. For instance, better support of lists is required
to verify RCU's callback handling mechanism. A field-sensitive SSA encoding
for structures and a thread-aware slicer will help reduce encoding size,
and therefore improve scalability.
% \comment{Paul: Would it make sense to list potential areas for
% improvement of CBMC?  Were there any CBMC bugs that had to be fixed
% for this work to proceed? Lihao: Daniel, could you review this paragraph?}

This work demonstrates the nascent ability of SAT-based
formal-verification tools to handle real-world pro\-duc\-tion-quality
synchronization primitives, as exemplified by Linux-kernel
Tree RCU on weakly ordered TSO and PSO systems.
Although modeling weak ordering incurs a significant performance
penalty, this penalty is not excessive.
We therefore hypothesize that use of these tools for highly concurrent
multithreaded software
will reach mainstream within 3-5 years, especially given recent
rates of improvement.
%We expect early-adopter use to commence much sooner.
% \comment{Lihao: as far as I know, they are already used 
% in automoblie industry to verify relatively small and simple embedded 
% software C code. Shall we emphasize this to be happening in the Linux kernel?
% A small concern though: saying it in authority/certainty is likey to 
% reveal the authorship in a double-blind review process...But we certainly
% can do it in the final version of the paper.}
% I avoided the auto-industry use by calling out highly concurrent etc.
% I made the early adopter prediction less precise.

%This work also confirms, in a very practical setting, the tractability and
%practicality of the use of bug injection to validate both the model and the
%tools.
%We hypothesize that use of bug injection will do much to increase
%practitioner confidence in these tools and techniques, particularly
%in those situations where they fail to locate bugs.
%We nevertheless anticipate that practitioner confidence will
%increase most dramatically in those cases where these tools
%and techniques locate real and relevant bugs in production code.
% \comment{Lihao: bug-injection is very useful in model validation so
% it is quite common in verification papers, especially when 
% no real bugs were found. So this paragraph might not add too much value
% to reviewers in the verfication field.}

%\comment{Lihao: improve reference entries with more details since they don't have page limits}

% Future work
% Weak memory models
% Start a new grace period
% Force quiescent state 
% Expedited grace period
% Dyntick-idle
% CPU hotplug

% Parametrised verification
% Inductive verification
                 % Summary and Furture Work
%\input{ack.tex}                   % Acknowledgements

%\newpage
%\bibliographystyle{abbrv}    % sig-alternate
% \bibliographystyle{abbrvnat} % sigplanconf
% \raggedright
% \bibliography{paper}

%\clearpage
%\appendix
%\vskip5mm
%\renewcommand{\thefigure}{\Asbuk{section}.\arabic{figure}}
\renewcommand{\thetable}{\Asbuk{section}.\arabic{table}}
\renewcommand{\thelstlisting}{\Asbuk{section}.\arabic{lstlisting}}

\chapter*{Приложение}
\addcontentsline{toc}{chapter}{Приложение}

\setcounter{section}{1}
\setcounter{figure}{0}
\setcounter{table}{0}
\setcounter{lstlisting}{0}

Приложения оформляются как продолжение диссертации на последующих ее страницах, располагая их в порядке появления ссылок в тексте. В приложения следует включать вспомогательный материал, необходимый для полноты восприятия диссертации: таблицы вспомогательных цифровых данных; протоколы и акты испытаний и внедрения; описание  алгоритмов и программ задач, решаемых на ЭВМ, разработанных в процессе выполнения работы; иллюстрации вспомогательного характера.

\end{document}
