\section{Implementation of Tree RCU}\label{sec:tree_rcu}

The primary advantage of RCU is that it is able to wait for an arbitrarily
large number of readers to finish without keeping track every single one of
them.  The number of readers can be large (up to the number of CPUs in
non-pre\-emptible implementations and up to the number of tasks in
preemptible implementations).
%
Although RCU's read-side primitives enjoy excellent performance and scalability, 
update-side primitives must defer the reclamation phase till all pre-existing 
readers have completed, either by blocking or by registering a callback that is 
invoked after a grace period. The performance and scalability of RCU relies on 
efficient mechanisms to detect when a grace period has completed. 
%
For example, a simplistic RCU implementation might require each CPU to
acquire a global lock during each grace period, but this would severely
limit performance and scalability.
Such an implementation would be quite unlikely to scale beyond
a few hundred CPUs.
This is woefully insufficient because Linux runs on systems with thousands
of CPUs.
This has motivated the creation of Tree RCU.

\subsection{Overview}
% Lihao: include this in PhD thesis; also look for 'preemptible RCU contents' 
%There are several flavors of RCU, including RCU-sched, RCU-preempt, 
%RCU Bot\-tom-Half, and Sleepable RCU (SRCU)~\cite{MckenneyRCUflavors}.
% Different flavors have different quiescent states, which are 
% discussed in Section \ref{sec:quiescent_state}.
%
% Classic RCU has two implementations: Tiny RCU and Tree RCU. Tiny RCU
% is only for uni-processor systems; we therefore focus on Tree RCU in this paper.
% Tree RCU further has non-pre\-emptible and preemptible variants, 
% configured by the kernel Kconfig option \co{CONFIG_PREEMPT}.
% There is no preemptible implementation of Tiny RCU: Instead,
% preemptible Tree RCU is used in single-CPU preemptible kernel builds.
%
% Tree RCU implements RCU-sched, RCU-bh, and RCU-preempt when \co{CONFIG_PREEMPT=y}.
% If \co{CONFIG_PREEMPT=n}, then RCU-preempt is mapped into RCU-sched.
% Tiny RCU requires \co{CONFIG_PREEMPT=n}, so it also maps
% RCU-preempt into RCU-sched.
%
We focus on the ``vanilla'' RCU API in a non-preemptible build of
the Linux kernel, specifically on the \co{rcu_read_lock()},
\co{rcu_read_unlock()}, and \co{synchronize_rcu()} primitives.
%
%\comment{Lihao: it seems the Tree implementation of Classic RCU also 
%implements other three flavors: RCU-sched, RCU-bh, and RCU-preempt;
%and Tiny RCU implements RCU-sched and RCU-bh. Am I right?
%Conceptually, what is the relationship between flavors Classic RCU and 
%RCU-sched/RCU-bh? I shall discuss their relationship here otherwise 
%readers may get confused when we discuss different flavors in the 
%Tree RCU implementation in later sections.}
%\comment{Paul: 
%	Yes, Tree RCU implements RCU-sched, RCU-bh, and RCU-preempt,
%	but only when \co{CONFIG_PREEMPT=y}.
%	If \co{CONFIG_PREEMPT=n}, then RCU-preempt is mapped into
%	RCU-sched.
%	Because Tiny RCU is requires \co{CONFIG_PREEMPT=n}, it behaves
%	the same as does Tree RCU when \co{CONFIG_PREEMPT=n},
%	implementing RCU-sched and RCU-bh, and mapping RCU-preempt into
%	RCU-sched.
%	For RCU-preempt, any location outside of an RCU read-side
%	critical section is a quiescent state.
%	For RCU-sched, context switch, idle, userspace,
%	\co{cond_resched_rcu_qs()}, and offline are all quiescent
%	states.
%	For RCU-bh, any location where bottom-half execution is enabled
%	is a quiescent state.
%	Use RCU-sched when you need updaters to wait on hardware interrupt
%	handlers (device drivers) or preempt-disable regions (tracing).
%	Use RCU-bh when networking denial-of-service attacks are a potential
%	issue.}
%
% In a non-pre\-empt\-ible kernel, Tiny and Tree RCU use the same
% \co{rcu_read_lock()} and \co{rcu_read_unlock()} implementation.
% Tiny RCU's \co{synchronize_rcu()} implementation is trivial, 
% while preemptible and non-pre\-emptible Tree RCU largely share a rather
% elaborate implementation.
% 
The key idea is that RCU read-side primitives are confined to kernel code and,
in non-pre\-emptible implementations, do not block.
Thus, when a CPU is blocking, in the idle loop, or running in user mode,
all RCU read-side critical sections that were previously running on that CPU
must have finished. Each of these states is therefore called a \emph{quiescent state}. 
After each CPU has passed through a quiescent state, the corresponding RCU grace period ends.
%\comment{Paul: Shouldn't the definition of quiescent state be before the
%first use?}
%\comment{Lihao: move here from the Read-Side Primitives subsection} 
% \comment{Lihao: add an overview and high-level idea of how the implementation of Tree RCU works.}
The key challenge is to determine when all necessary quiescent
states have completed for a given grace period---and to do so with
excellent performance and scalability.

For example, if RCU used a single data structure to record each CPU's
quiescent states, the result would be extreme lock contention on large systems,
in turn resulting in poor performance and abysmal scalability.
Tree RCU therefore instead uses a tree hierarchy of data structures, 
each leaf of which records quiescent states of a single CPU and propagates the information 
up to the root. When the root is reached, a grace period has ended. Then the grace-period
information is propagated down from the root to the leaves of the tree.
Shortly after the leaf data 
structure of a CPU receives this information, \co{synchronize_rcu()} will return.

In the remainder of this section, we discuss the implementation of the non-pre\-empt\-ible 
Tree RCU in the Linux kernel version 4.3.6. We first briefly discuss the implementation of 
read/write-side primitives.
%how non-pre\-empt\-ible Tree RCU detects quiescent states without individually tracking readers.
We then explain Tree RCU's hierarchical data structure which records quiescent states while 
maintaining bounded lock contention. Finally, we discuss how RCU uses this data structure 
to detect quiescent states and %communicate them among all the CPUs, as well as to detect a 
grace periods without individually tracking readers.

%\subsection{Read-Side Primitives} \label{sec:read_api_impl}
\subsection{Read/Write-Side Primitives} \label{sec:api_impl}
% Change in recent kernels.
В non-preemptible версии ядра любая область его исходного кода,
которая не использует добровольных блокировок, является неявной
критической секцией чтения RCU. В связи с этим, реализации
\co{rcu_read_lock()} и \co{rcu_read_unlock()} не должны выполнять
никакой работы. Действительно, в production сборках ядра
с выключенным режимом отладки, эта пара примитивов является
пустышками.

В общем случае, когда используется несколько вычислительных ядер процессора,
примитив записи \co{synchronize_rcu()} вызывает \co{wait_rcu_gp()},
которая является внутренней функцией, использующей механизм callback'ов
для отложенного вызова \co{wakeme_after_rcu()} по окончании некоторого grace-периода.
Как подсказывает название, данная функция повторно предназначена для повторного вызова
\co{wait_rcu_gp()}, которая на этот раз ничего не делает,
тем самым позволяя \co{synchronize_rcu()} вернуть управление в вызывающий поток.

%\comment{Lihao: comment out the following preemptible RCU contents if we need space.}
%In a preemptible kernel, \co{synchronize_rcu()} is implemented in
%\co{kernel/rcu/tree_plugin.h}. It first checks whether the variable
%\co{rcu_scheduler_active} is zero. If so, the system is so early in boot
%that there is only one non-preemptible task, again meaning that grace
%periods complete instantaneously, allowing an immediate return.
%Otherwise, if the grace period should be expedited,
%\co{synchronize_rcu_expedited()} is invoked.
%Otherwise, it passes \co{call_rcu()} to \co{wait_rcu_gp()}, which
%registers callback \co{wakeme_after_rcu()}, similar to
%the non-preemptible kernel discussed above.
%%\comment{Lihao: the source code comments state that \co{rcu_scheduler_active = 0}
%%allows RCU to optimize \co{synchronize_sched()} to a simple \co{barrier()}.
%%Where is the code that does this?}
%%\comment{Paul: The comment is incorrect.
%%The \co{synchronize_sched()} function instead checks the number of
%%online CPUs.
%%I have queued a patch with your
%%Reported-by changing the comment's \co{synchronize_sched()} to
%%\co{synchronize_rcu()}.}
%
%RCU's callback handling and grace period detection are explained in Sections
%\ref{sec:rcu_data} and \ref{sec:grace_period}, respectively.

% Lihao: understand how call_rcu_sched works and understand the differences from
% the Tiny RCU version which only calls the kernel function cond_resched()
%
% Lihao: In a preemptible kernel, the implementation of \co{synchronize_rcu()}
% http://lxr.free-electrons.com/source/kernel/rcu/tree_plugin.h#L539 and
% understand the differences between preemptible and non-preemptible versions

\subsection{Data Structures of Tree RCU} \label{sec:data_structure}
% In Section~\ref{sec:rcu_softirq}, we discuss how RCU's softirq handlers walk up the 
% tree hierarchy of the \co{rcu_node} data structure. In ths section, we explain 
% in detail how this data structure is implemented in \co{kernel/rcu/tree.h} and used 
% in Tree RCU.

\begin{figure}[tbp]
\centering
\includegraphics[scale=0.2]{tree_rcu_hierarchy.pdf}
\caption{Tree RCU Hierarchy}
\label{fig:tree_rcu_hierarchy}
\end{figure}

RCU's global state is recorded in the \co{rcu_state} structure, which consists of 
a tree of \co{rcu_node} structures with a child count of up to 64
(32 in a 32-bit system). Every leaf node can have at most 64 
\co{rcu_data} structures (again 32 on a 32-bit system), each representing
a single CPU, as illustrated in
Figure~\ref{fig:tree_rcu_hierarchy}.
%
Each \co{rcu_data} structure records its CPU's quiescent states, and
the \co{rcu_node} tree propagates these states up to the root, and then
propagates grace-period information back down to the leaves.
%
Quiescent-state information does not propagate upwards from a given node
until a quiescent state has been reported by each CPU covered by the subtree
headed by that node.
This propagation scheme dramatically reduces the lock contention experienced
by the upper levels of the tree.
%
% Lihao: include this in PhD thesis and the technical report
For example, consider a default \co{rcu_node} tree for a 4,096-CPU system,
which will have have 256 leaf nodes, four internal nodes, and one root node.
During a given grace period, each CPU will report its quiescent states
to its leaf node, but there will only be 16 CPUs contending for each of
those 256 leaf nodes.
Only 256 of the CPUs will report quiescent states to the internal nodes,
with only 64 CPUs contending for each of the four internal nodes.
Only four CPUs will report quiescent states to the root node, resulting
in extremely low contention on the root node's lock, so that contention
on any given \co{rcu_node} structure is sharply bounded even in very
large configurations.
%
The current RCU implementation in the Linux kernel supports up to a
four-level tree, and thus in total $64^4 = 16,777,216$ CPUs in a 64
bit machine.\footnote{
	Four-level trees are only used in stress testing,
	but three-level trees are used in production by 4096-CPU systems.}

\subsubsection{\co{rcu_state} Structure}

\begin{figure}[tbp]
\centering
\includegraphics[scale=0.9]{rcu_node_array.pdf}
\caption{Array Representation for a Tree of \co{rcu_node} Structures}
\label{fig:rcu_node_array}
\end{figure}

Each flavor of RCU has its own global \co{rcu_state} structure. 
%For example, \co{rcu_state} pointers \co{rcu_sched_state}, 
%\co{rcu_bh_state} and \co{rcu_state_p} are used by RCU-sched, RCU-bh 
%and RCU-preempt, respectively. 
The \co{rcu_state} structure includes
a array of \co{rcu_node} structures organized as a tree
\co{struct rcu_node node[NUM_RCU_NODES]}, with
\co{rcu_data} structures connected to the leaves.
Given this organization, a breadth-first traversal is 
simply a linear scan of the array.
%\comment{Lihao: we may also remove the following sentence and the figure 
%as it's a bit too technical and the level array is never refered again in the paper}
% Lihao: include this in the technical report
Another array \co{struct rcu_node} \co{*level[NUM_RCU_LVLS]} 
is used to point to the left-most node at each level of the tree,
as shown in Figure~\ref{fig:rcu_node_array}.

The \co{rcu_state} structure uses \co{unsigned long} fields \co{->gpnum}
and \co{->completed} to track RCU's grace periods.
The \co{->gpnum} field records the most recently started grace period,
whereas \co{->completed} records the most recently ended grace period.
If the two numbers are equal, then corresponding flavor of RCU is idle.
If \co{gpnum} is one greater than \co{completed}, then RCU is in the
middle of a grace period.
All other combinations are invalid.
% Which of course means that we could instead use a single counter with an
% odd/even scheme to track grace periods.

%(Lihao: ignore variables used to force quiescent states by force_quiescent_state())

\subsubsection{\co{rcu_node} Structure}
\label{sec:rcu_node}
The tree of \co{rcu_node} structures records and 
propagates quiescent-state information from the leaves to the root,
and also propagates grace-period information from the root to the leaves. 
%
The \co{rcu_node} structure has a spinlock \co{->lock} to protect its fields.
The \co{->parent} field references the parent \co{rcu_node} structure,
and is \co{NULL} for the root.
The \co{->level} field indicates the level in the tree, counting from zero
at the root.
The \co{->grpmask} field identifies this node's bit in the
\co{->qsmask} field of its parent.
The \co{->grplo} and \co{->grphi} fields indicates the lowest and highest 
numbered CPU that are covered by this \co{rcu_node} structure, respectively.

The \co{->qsmask} field indicates which of this node's children
still need to report quiescent states for the current grace period.
%
As with \co{rcu_state}, the \co{rcu_node} structure has \co{->gpnum} 
and \co{->completed} fields that have values identical to those of the
enclosing \co{rcu_state} structure, except at the beginnings and ends
of grace periods when the new values are propagated down the tree.
Each of these fields can be smaller than 
its \co{rcu_state} counterpart by at most one.

%\comment{Lihao: comment out the following preemptible RCU contents if we need space.}
%In a preemptible kernel, tasks can be preempted during RCU read-side
%critical sections.
%When an RCU read-side critical section is preempted,
%the preempted task's \co{task_struct} is enqueued onto the \co{->blkd_tasks}
%list in the leaf \co{rcu_node} structure covering the task's CPU.
%That task will remove itself once it reaches the RCU read-side critical
%section's outermost \co{rcu_read_unlock()},
%%
%When the \co{->gp_tasks} pointer is non-\co{NULL}, it references the first
%task blocking the current grace period.
%When a task referenced by \co{gp_tasks} points is removed 
%from \co{blkd_tasks}, the pointer will be advanced to the next task on the list,
%or is set to \co{NULL} if there are no more tasks.
%Note that
%tasks blocking the current grace period are queued in the reverse time order.
%Thus, if a task is blocking a grace period, 
%all subsequent tasks on the list are blocking the same grace period.
% Lihao: how the tasks are dequeued is described in quiescent state detection
% Lihao: we ignore expedited grace period for now
% Lihao: we don't model priority boosting

\subsubsection{\co{rcu_data} structure} \label{sec:rcu_data}
The \co{rcu_data} structure detects quiescent states and handles RCU
callbacks for the corresponding CPU.
The structure is accessed primarily from the corresponding CPU,
thus avoiding synchronization overhead.
As with the \co{rcu_state} structure, different flavors of RCU maintain 
their own per-CPU \co{rcu_data} structures. %For instance, RCU-sched's 
%\co{rcu_sched_state}, RCU-bh's \co{rcu_bh_state} and RCU-preempt's 
%\co{rcu_state_p} structures have \co{rcu_data} structures \co{rcu_sched_data}, 
%\co{rcu_bh_data}, and \co{rcu_data_p}, respectively.
%
The \co{->cpu} field identifies the corresponding CPU, the \co{->rsp}
field references the corresponding \co{rcu_state} structure, and the
\co{->mynode} field references the corresponding leaf \co{rcu_node}
structure.
The \co{->grpmask} field identifies this \co{rcu_data} structure's bit
in the \co{->qsmask} field of its leaf \co{rcu_node} structure.

The \co{rcu_data} structure's \co{->qs_pending} field indicates that RCU
needs a quiescent state from the corresponding CPU, and the
\co{->passed_quiesce} indicates that the CPU has already passed through
a quiescent state.
%
The \co{rcu_data} also has \co{->gpnum} and \co{->completed} fields,
which can lag arbitrarily behind their counterparts in
the \co{rcu_state} and \co{rcu_node} structures on idle CPUs.
However, on the non-idle CPUs that are the focus of this paper,
they can lag at most one grace period behind their leaf \co{rcu_node} 
counterparts.

The \co{rcu_state} structure's \co{->gpnum} and \co{->completed} fields
represent the most current values, and are tracked closely by those of
the \co{rcu_node} structure, which allows the \co{->gpnum} and
\co{->completed} fields in the \co{rcu_data} structures to be
are compared against their counterparts in the corresponding leaf \co{rcu_node}
to detect a new grace period. 
This scheme allows CPUs to detect beginnings and ends of grace periods without
incurring lock- or memory-contention penalties.
%
The \co{rcu_data} structure manages RCU callbacks using a 
four-segment list~\cite{LaiJiangshan2008NewClassicAlgorithm}.

% Lihao: but we need to carefully manage the numbers of each node as the consequences of
% using a quiescent state in a wrong grace period can be quite serious.
% Paul: Indeed!  And the grace-period initialization (rcu_gp_init()) and
% cleanup (rcu_gp_cleanup()) code first updates the rcu_state structure and
% then the rcu_node structures in breadth-first order to avoid such
% consequences.  In addition, cleanup propagates ->completed completely
% and only then is ->gpnum propagated for the new grace period.  Attempting
% to "optimize" this to propagate ->completed and ->gpnum changes in one
% pass results in nasty race conditions caused by different CPUs believing
% that different active grace periods are in effect.  Very low probability,
% but -very- nasty.

% Lihao: we don't model dyntick-idle handling
% Lihao: include this in the technical report
\begin{figure}[tbp]
\centering
\includegraphics[scale=0.25]{rcu_data_callbacks.pdf}
\caption{Очередь callback'ов в \co{rcu_data}}
\label{fig:rcu_data_callbacks}
\end{figure}

\subsubsection{Callback'и RCU}
Структура \co{rcu_data} управляет RCU callback'ами, используя указатель
\co{->nxtlist}, указывающим на начало списка, и массив \co{->nxttail[]}
указателей-на-конец, формирующий четырехсегментный список
callback'ов~\cite{LaiJiangshan2008NewClassicAlgorithm},
где каждый элемент массива \co{->nxttail[]} указывает на конец
соответсвующего сегмента, как показано на рисунке~\ref{fig:rcu_data_callbacks}.
Сегмент, оканчивающийся на \co{->nxttail[RCU_DONE_TAIL]}
(<<\co{RCU_DONE_TAIL} сегмент>>), содержит готовые к вызову callback'и,
связанные с предыдущим grace-периодом.
Сегменты \co{RCU_WAIT_TAIL} и \co{RCU_NEXT_READY_TAIL}
содержат callback'и, ожидающие окончания текущего и следующего
grace-периодов, соответственно.
Наконец, сегмент \co{RCU_NEXT_TAIL} содержит callback'и,
которые еще не были связаны с каким-либо grace-периодом.
Поле \co{->qlen} выполняет учет общего числа callback'ов,
а \co{->blimit} определяет максимальное число callback'ов,
которые могут быть вызваны в данный момент времени, тем самым
ограничивая размер окна времени, используемого для их вызова,
на длинных списках callback'ов.\footnote{
  Вычислительные окружения реального времени, требующие выполнения
  более строгих ограничений времени вызова, должны использовать
  callback offloading, который находится вне контекста рассмотрения данной статьи.}

Возвращаясь к рисунку~\ref{fig:rcu_data_callbacks} отметим,
что элемент массива \co{->nxttail[RCU_DONE_TAIL]} указывает на \co{->nxtlist},
что означает, что в данный момент ни один из callback'ов не готов к вызову.
Элемент \co{->nxttail[RCU_WAIT_TAIL]} указывает на \co{->next}-указатель
второго callback'а, что означает, что callback'и CB~1 и CB~2 ожидают окончания данного
grace-периода.
Элемент \co{->nxttail[RCU_NEXT_READY_TAIL]} указывает на этот же
\co{->next}-указатель, что означает, что список не содержит callback'ов,
связанных со следующим grace-периодом.
Наконец, callback'и, расположенные между \co{->nxttail[RCU_NEXT_READY_TAIL]} и
\co{->nxttail[RCU_NEXT_TAIL]} элементами (CB~3 и CB~4),
еще не связаны ни с каким grace-периодом.
Элемент \co{->nxttail[RCU_NEXT_TAIL]} всегда указывает либо на последний callback,
либо, если весь список пустой, на \co{->nxtlist}.

Cache locality достигается за счет вызова callback'ов на тех вычислительных ядрах,
которые их зарегистрировали.
Например, примитив записи RCU \co{synchronize_rcu()} добавляет
callback \co{wakeme_after_rcu()} в конец списка \co{->nxttail[RCU_NEXT_TAIL]}
на данном вычислительном ядре (раздел \ref{sec:update_api_impl}).
По окончании данного grace-периода, которому соответствует изменение значения
поля \co{->completed} структуры \co{rcu_data}, меньшего, чем соответствующее значение
структуры \co{rcu_node}, они смещаются на один сегмент списка
(с помощью \co{rcu_advance_cbs()}).
Кроме этого, ядро процессора периодически объединяет сегменты \co{RCU_NEXT_TAIL}
и \co{RCU_NEXT_READY_TAIL} путем вызова \co{rcu_accelerate_cbs()}.
В некоторых специальных случаях, ядро выполняет объединение сегментов
\co{RCU_NEXT_TAIL} и \co{RCU_WAIT_TAIL}, пропуская сегмент \co{RCU_NEXT_TAIL}.
Эта оптимизация применяется в тех случаях, когда ядро начинает новый grace-период.
Она \emph{не} используется, когда ядро обнаруживает новый grace-период,
поскольку этот период мог начаться до момента добавления callback'ов
в сегмент \co{RCU_NEXT_TAIL}.

Это особенность архитектуры неслучайна: требование обеспечения
независимости работы вычислительных ядер
(для избегания исопльзования блокировок) является более важным,
чем сокращение продолжительности grace-периодов.
В тех редких случаях, когда требуются grace-периоды должны быть
максимально короткими, требуется использовать \co{synchronize_rcu_expedited()}.
Эта функция имеет такую же семантику, как и \co{synchronize_rcu()},
но предпочитает уменьшение задержки остальным оптимизациям.

Каждый RCU callback представляет собой структуру \co{rcu_head},
имеющую поле \co{->next}, указывающее на следующий callback в списке,
и поле \co{->func}, указывающее на функцию, подлежащую вызову
по окончаниии предстоящего grace-периода.


\subsection{Quiescent State Detection} \label{sec:quiescent_state}
RCU has to wait until all pre-existing read-side critical sections have
finished before it can safely allow a grace period to end.
The performance and scalability of RCU rely on its ability to efficiently
detect quiescent states and determine whether the set of quiescent states
detected thus far allows the grace period to end.
If each CPU (or, in the case of preemptible RCU, each task)
has passed through a quiescent state, a grace period has elapsed. 

% Lihao: include this in PhD thesis; also look for 'preemptible RCU contents'
%Different flavors of RCU use different sets of quiescent states.
The non-preemptible RCU-sched flavor's quiescent states
apply to CPUs, and are user-space execution, context switch, idle, and 
offline state.
%
%RCU-bh's quiescent states are those of RCU-sched plus any execution 
%in which bottom-half (AKA softirq) is enabled, along with transitions
%from one softirq handler to another.
%%
%RCU-preempt's quiescent states are any execution outside of an
%RCU read-side critical sections.
%
Therefore, RCU-sched %and RCU-bh need only
only needs to track tasks and interrupt handlers that are actually running because
blocked and preempted tasks are always in quiescent states. Thus, RCU-sched %and RCU-bh 
needs only track CPU states.
%By contrast, RCU-preempt must track tasks states. In this section, 
%we focus on the quiescent-state detection of RCU-sched in a non-preemptible kernel.
%\comment{Lihao: comment out the following preemptible RCU contents if we need space.}
%However, there can be a great many tasks, and scanning all of them could
%result in excessive per-grace-period overheads.
%However, if a task has been preempted or has blocked outside of an RCU
%read-side critical section, its state need not be considered.
%Therefore, a given grace period need only consider tasks that have been
%preempted (or, in real-time variants of the Linux kernel, blocked)
%within an RCU read-side critical section that began before the current
%grace period did.
%These are exactly the tasks that are tracked by the \co{rcu_node} structure's
%\co{->blkd_tasks} list and \co{->gp_tasks} pointer, as discussed in
%Section~\ref{sec:rcu_node}.

\subsubsection{Scheduling-Clock Interrupt} \label{sec:timer_interrupt}
The \co{rcu_check_callbacks()} is invoked from the sched\-ul\-ing-clock interrupt
handler, which allows RCU to periodically check whether a given busy CPU
is in the user-mode or idle-loop quiescent states.
If the CPU is in one of these quiescent states, \co{rcu_check_callbacks()}
invokes \co{rcu_sched_qs()}, %and \co{rcu_bh_qs()}, 
which sets the per-CPU \co{rcu_sched_data.passed_quiesce} %and \co{rcu_bh_data.passed_quiesce}
fields to 1. %, respectively.
%In addition, when the scheduling-clock interrupt happens outside of softirq context,
%that is, outside of the softirq handler and also outside of any
%bottom-half-disable mode, then the CPU is in an RCU-bh quiescent state,
%in which case \co{rcu_check_callbacks()} will invoke \co{rcu_bh_qs()}
%to inform RCU of this quiescent state.

%\comment{Lihao: comment out the following preemptible RCU contents if we need space.}
%When RCU-preempt is present,
%\co{rcu_check_callbacks()} invokes
%\co{rcu_preempt_check_callbacks}, which
%checks for RCU-preempt quiescent states, which are in effect whenever
%the per-task \co{->rcu_read_lock_nesting} field is equal to 0.
%This condition causes \co{rcu_preempt_check_callbacks} to invoke
%\co{rcu_preempt_qs()}, which in turn records
%quiescent state for by setting the per-CPU \co{rcu_data_p->passed_quiesce} 
%field to true.
%
The \co{rcu_check_callbacks()} function invokes \co{rcu_pending()}
to determine whether a recent event or current condition means that
RCU requires attention from this CPU.
% so the next outermost \co{rcu_read_unlock()} will announce a quiescent state.
% http://lxr.free-electrons.com/source/kernel/rcu/tree_plugin.h#L485
If so, \co{rcu_check_callbacks()} invokes \co{raise_softirq()}, %with a \co{RCU_SOFTIRQ} argument, 
which will cause \co{rcu_process_callbacks()} to be invoked once the CPU 
reaches a state where it is safe to do so (roughly speaking, once the CPU 
has interrupts, preemption, and bottom halves enabled). This function is 
discussed in detail in Section \ref{sec:grace_period}.

% Paul: Yes, the order is different than the code, but it seems to flow better
% this way.  Perhaps I should re-order the code.  Except that I don't touch
% that particular piece of code without an extremely good reason.  ;-)

% Finally, if the scheduling-clock interrupt is raised in the user mode, we perform
% a context switch.
% Lihao: voluntary context switch?
% Paul: This is for Tasks RCU, which is probably out of scope.  That said,
% This function is not -performing- a voluntary context switch, but rather
% checking to see if a voluntary context switch has been performed.  In
% the context of Tasks RCU, executing in user mode is equivalent to having
% performed a voluntary context switch -- either way, whatever was happening
% in the kernel beforehand is now well and truly done.
% Lihao: understand the differences from the tiny RCU version and how the tiny RCU one 
% is related to cond_resched in synchronize_sched
% Lihao: understand how it is related to wait_rcu_gp(call_rcu_sched) in synchronize_sched()

\subsubsection{Context-Switch Handling} \label{sec:context_switch}
The context-switch quiescent state is recorded by invoking
\co{rcu_note_context_switch()} from \co{__schedule()} (and, for the
benefit of virtualization, also from \co{rcu_virt_note_context_switch()}).
% http://lxr.free-electrons.com/source/kernel/sched/core.c#L3057
%
The \co{rcu_note_context_switch()} function invokes \co{rcu_sched_qs()}
to inform RCU of the context switch, which is a quiescent state of the CPU.
%Note that although quiescent states of RCU-bh include those of RCU-sched,
%\co{rcu_note_context_switch()} does not invoke \co{rcu_bh_qs()}.
%This could in theory starve RCU-bh grace periods if a given CPU spent all
%its time in the kernel in bottom-half-disabled regions, without any
%calls to \co{schedule()}.
%No part of the kernel currently does this, but should this pattern arise,
%RCU-bh's quiescent-state recording strategy will need to be revisited.

%\comment{Lihao: comment out the following preemptible RCU contents if we need space.}
%It also invokes \co{rcu_preempt_note_context_switch()} to add the current
%task to the \co{->blkd_tasks} list of the CPU's leaf \co{rcu_node}
%structure for context switches that occur within an RCU-preempt read-side
%critical section.
%To prevent this task from being re-added while within its current
%RCU-preempt read-side critical section,
%the first \co{rcu_preempt_note_context_switch()} sets the
%\co{->rcu_read_unlock_special.b.blocked} field in the task structure.
%
%However, if current task has already reported an RCU-preempt
%quiescent state for the current grace period, and if at least one
%other task is blocking that grace period on this \co{rcu_node}
%structure,
%the task should be added to the head of the \co{->blkd_tasks} list
%in order to avoid blocking that grace period.
%In this case, the \co{->gp_tasks} field
%will be non-\co{NULL} and the current CPU's bit will already be cleared
%from the \co{->qsmask} field.
%In all other cases, the task should be added to the tail of the
%\co{->blkd_tasks} list.
%If the task is blocking the current RCU-preempt grace period and
%\co{->gp_tasks} is \co{NULL}, then this is the first task on this
%leaf \co{rcu_node} structure to block the this grace period, and
%therefore \co{->gp_tasks} is set to reference the current task.
%This approach allows RCU to easily identify which tasks are blocking
%the current grace period.
%
%The \co{rcu_preempt_note_context_switch()} function also invokes
%\co{rcu_preempt_qs()} to note a quiescent state for the current CPU.
%Nevertheless, any tasks queued on the \co{->gp_tasks} segment of
%\co{->blkd_tasks} will continue to block the grace period.
%
%All of the tasks on the \co{->blkd_tasks} list dequeue themselves
%during the outermost \co{rcu_read_unlock()}.
%This of course introduces a race condition where a task is preempted
%while executing its outermost
%\co{rcu_read_unlock()}~\cite{PaulEMcKenney2011RCU3.0trainwreck}.
%\comment{Paul: This citation isn't all that important, so feel free to remove.}
%This race is detected by having \co{rcu_read_unlock()} set the \co{task_struct}
%structure's \co{->rcu_read_lock_nesting} field to a negative value.
%When \co{rcu_preempt_note_context_switch()} sees this negative value,
%it invokes \co{rcu_read_unlock_special()} to complete the dequeuing
%of the current task from the \co{->blkd_tasks} list.
%Interrupt disabling prevents further destructive races.
%% Lihao: if show code, add the following text
%% Recall that \co{rcu_read_unlock()} sets \co{rcu_read_lock_nesting} of its  
%%\co{task_struct} structure to \co{INT_MIN} before invoking 
%%\co{rcu_read_unlock_special()} 
%%(and to 0 after \co{rcu_read_unlock_special()} returns)
%% If the {task_struct}'s \co{rcu_read_unlock_special.s} is not equal to 0,
%% we have work left to do for \co{rcu_read_unlock_special}, which is then invoked.
%% http://lxr.free-electrons.com/source/kernel/rcu/tree_plugin.h#L267 
%% Lihao: may need to explain the code of rcu_read_unlock_special() 
%% Paul: I did a minimal explanation, which can be expanded if necessary.

\subsection{Обнаружение grace-периодов} \label{sec:grace_period}
Как только каждое вычислительное ядро прошло через устойчивое состояние,
grace-период RCU закончился.
Как было рассмотрено в разделе \ref{sec:data_structure},
Tree-RCU использует иерархию структур \co{rcu_node} для
управления информацией об устойчивых состояниях и grace-периодах.
Информация об устойчивых состояниях распространяется в направлении
от терминальных узлов к корню, а информация о grace-периодах ---
от корня к терминальным узлам.
%
%The dyntick-idle mechanisms used for idle CPUs and \co{nohz_full} userspace
%execution are out of scope for this research, as are RCU CPU stall warnings. The focus is instead
Будем рассматривать процесс обнаружения grace-периодов на загруженных
вычислительных ядрах, как показано на рисунке~\ref{fig:grace_period_state_diagram}.

\begin{figure}[tb]
\centering
\includegraphics[scale=0.25]{grace_period_state_diagram.pdf}
\caption{Диаграма состояний процесса обнаружения grace-периодов}
\label{fig:grace_period_state_diagram}
\end{figure}

\subsubsection{Softirq Handler for RCU} \label{sec:rcu_softirq}
RCU's busy-CPU grace period detection relies on the
\co{RCU_SOFTIRQ} handler function \co{rcu_process_callbacks()},
which is scheduled from the scheduling-clock interrupt.
This function first calls
\co{rcu_check_quiescent_state()} to report recent quiescent states
on the current CPU.
Then \co{rcu_process_callbacks()} starts a new grace period if needed,
and finally calls \co{invoke_rcu_callbacks()} to invoke any callbacks
whose grace period has already elapsed.

Function \co{rcu_check_quiescent_state()} first invokes \co{note_gp_changes()}
to update the CPU-local \co{rcu_data} structure to record the end of
previous grace periods and the beginning of new grace periods.
%, which are detected via differences in the \co{->completed} and
%\co{->gpnum} fields, respectively.
Any new values for these fields are copied from the leaf \co{rcu_node}
structure to the \co{rcu_data} structure.
If an old grace period has ended, \co{rcu_advance_cbs()} is invoked to
advance all callbacks, otherwise, \co{rcu_accelerate_cbs()} is invoked
to assign a grace period to any recently arrived callbacks.
If a new grace period has started, \co{->passed_quiesce} is set to zero,
and if in addition RCU is waiting for a quiescent state from this CPU,
\co{->qs_pending} is set to one, so that a new quiescent state will
be detected for the new grace period.
%
% Lihao: this is one of the two places where qs_pending gets updated in Tree RCU
% \comment{(Lihao: does it mean even this softirq is invoked because of a quiescent state of this CPU
%   (\co{rdp->passed_quiesce} is set to 1 in \co{rcu_check_callbacks} so \co{rcu_pending}
%   return 1), if for some reason gpnum in \co{rcu_data} of this CPU is one lag behind its parent
%   counterpart, this CPU needs to wait for its next quiescent to commit?
%   \url{http://lxr.free-electrons.com/source/kernel/rcu/tree.c#L1747}
% )}
% Paul: Yes.  I did look into immediately detecting quiescent states for
% RCU-preempt, but it didn't seem worth the coding contortions required.

Next,
\co{rcu_check_quiescent_state()} checks whether \co{->qs_pending} indicates
that RCU needs a quiescent state from this CPU.
If so, it checks whether \co{->passed_quiesce} indicates that this
CPU has in fact passed through a quiescent state.
If so, it invokes \co{rcu_report_qs_rdp()} to report that quiescent
state up the %\co{rcu_data} and \co{rcu_node}
combining tree.

The \co{rcu_report_qs_rdp()} function first verifies that the CPU has
in fact detected a legitimate quiescent state for the current grace period,
and under the protection of the leaf \co{rcu_node} structure's \co{->lock}.
If not, it resets quiescent-state detection and returns, thus ignoring
any redundant quiescent states belonging to some earlier grace period.
Otherwise, if the \co{->qsmask} field indicates that RCU needs to report a
quiescent state from this CPU, \co{rcu_accelerate_cbs()} is invoked to assign
a grace-period number to any new callbacks, and then \co{rcu_report_qs_rnp()}
is invoked to report the quiescent state to the \co{rcu_node} combining tree.

% \comment{(Lihao: did we just check this in \co{rcu_check_quiescent_state()}?
%   \url{http://lxr.free-electrons.com/source/kernel/rcu/tree.c#L2394}
% )}
% \comment{(Lihao: but did we just update \co{rdp->gpnum = rnp->gpnum} in \co{note_gp_changes()}...?
%   are they just double-checks or something may happen in between which I miss?
% )}
% Paul:  The code could probably be simplified.  The first step is to
% add assertions to verify the suspicions, and if the assertions don't
% trigger over a period of a year or so, simplify the code.  Sometimes
% the assertions have triggered, hence the caution.  ;-)
%
% \comment{(Lihao: what are \co{rcu_qs_ctr} and \co{rcu_qs_ctr_snap} used for?
%  \url{http://lxr.free-electrons.com/source/kernel/rcu/tree.c#L2341}
% )}
% Paul: These are used by cond_resched_rcu_qs(), which records a quiescent
% state for all flavors of RCU.
%
%
% \comment{(Lihao: can we use \co{rdp->qs_pending} in the following line of code since
%  it's also get updated in \co{note_gp_changes()}, right?
%  \url{http://lxr.free-electrons.com/source/kernel/rcu/tree.c#L2357}
%)}
%
% \comment{(Lihao: comments in \url{http://lxr.free-electrons.com/source/kernel/rcu/tree.c#L2272}
%   say if this CPU is the last one to pass through a quiescent state in the current grace period,
%   \co{rcu_report_qs_rsp()} is invoked to do the clean up and let \co{rcu_start_gp()}
%   start a new grace period if one is needed.~But where is \co{rcu_start_gp()} called in
%   \co{rcu_report_qs_rsp()}?
% )}
% Paul: This is done indirectly by waking up the RCU grace-period kthread.

The \co{rcu_report_qs_rnp()} function traverses up the \co{rcu_node} tree,
at each level holding the \co{rcu_node} structure's \co{->lock}.
At any level, if the child structure's \co{->qsmask} bit is already clear,
or if the \co{->gpnum} changes, traversal stops.
Otherwise, the child structure's bit is cleared from \co{->qsmask},
after which, if \co{->qsmask} is non-zero, %or if any tasks are queued on the
%\co{->blkd_tasks} list (which applies only to RCU-preempt),
traversal stops. Otherwise, traversal proceeds on to the parent \co{rcu_node} structure.
%If there is no parent (that is, the previous \co{rcu_node} structure was the root),
%the current grace period has completed. In that case, traversal stops and
Once the root is reached, traversal stops and \co{rcu_report_qs_rsp()} is
invoked to awaken the grace-period kthread (kernel thread).
The grace-period kthread will then clean up after the now-ended grace
period, and, if needed, start a new one.

\subsubsection{Grace-Period Kernel Thread} \label{sec:rcu_gp_kthread}
The RCU grace-period kthread invokes \co{rcu_gp_kthread()}, which
contains an infinite loop that initializes, waits for, and cleans up after
each grace period.

% rcu_gp_init()
When no grace period is required, the grace-period kthread
sets its \co{rcu_state} structure's \co{->flags} field to
\co{RCU_GP_WAIT_GPS}, and then
waits within an inner infinite loop for that structure's
\co{->gp_state} field to be set.
Once set, \co{rcu_gp_kthread()} invokes \co{rcu_gp_init()} to initialize
a new grace period, which
rechecks the \co{->gp_state} field under
the root \co{rcu_node} structure's \co{->lock}.
If the field is no longer set, \co{rcu_gp_init()} returns zero.
Otherwise, it
increments \co{rsp->gpnum} by 1 to record a new grace period number.
%
Finally, it performs a breadth-first traversal of the \co{rcu_node}
structures in the combining tree.
For each \co{rcu_node} structure \co{rnp},
% drop preemptible RCU contents
%we invoke \co{rcu_preempt_check_blocked_tasks()}, which responds to
%a non-empty list of blocked tasks by setting \co{rnp->gp_tasks} to
%\co{rnp->blkd_tasks.next}, so that those tasks block the new grace period.
%
we set the \co{rnp->qsmask} to indicate which children
must report quiescent states for the new grace period (Section
\ref{sec:rcu_node}), and set \co{rnp->gpnum} and \co{rnp->completed}
to their \co{rcu_state} counterparts.
%
If the \co{rcu_node} structure \co{rnp} is the parent of the current CPU's \co{rcu_data},
we invoke \co{__note_gp_changes()} to set up the CPU-local \co{rcu_data} state.
Other CPUs will invoke \co{__note_gp_changes()} after their next
scheduling-clock interrupt. %(Section~\ref{sec:timer_interrupt}).

%Note that other CPUs will access only the leaves of the hierarchy, thus seeing that
%no grace period is in progress, at least until the corresponding leaf node has been
%initialized. In addition, we have included CPU-hotplug operations since v4.1.

% Lihao: include this in PhD thesis; also look for 'preemptible RCU contents'
% rcu_gp_fqs()
%During a grace period, the grace-period kthread periodically
%calls \co{force_qs_rnp()} to detect idle and offline CPUs.
%For each such CPU, \co{force_qs_rnp()} invokes \co{rcu_report_qs_rnp()}
%to report a quiescent state on its behalf, thus avoiding the degraded
%energy efficiency that would be incurred should RCU awaken idle CPUs.
%CPUs that fail to report quiescent states will be sent an
%inter-processor interrupt (IPI), and if that fails, warning messages
%will be emitted.

% rcu_gp_cleanup()
To clean up after a grace period, \co{rcu_gp_kthread()}
calls \co{rcu_gp_cleanup()} after setting the \co{rcu_state} field \co{rsp->gp_state}
to \co{RCU_GP_CLEANUP}. After the function returns, \co{rsp->gp_state} is set to
\co{RCU_GP_CLEANED} to record the end of the old grace period.
%
Function \co{rcu_gp_cleanup()} performs a breadth-first traversal of
\co{rcu_node} combining-tree.
It first sets each \co{rcu_node} structure's \co{->completed} field
to the \co{rcu_state} structure's \co{->gpnum} field.
It then updates the current CPU's CPU-local \co{rcu_data} structure by
calling \co{__note_gp_changes()}.
For other CPUs, the update will take place when they handle the scheduling-clock
interrupts, in a fashion similar to \co{rcu_gp_init()}.
After the traversal, it marks the completion of the grace period by setting the
\co{rcu_state} structure's \co{->completed}
field to that structure's \co{->gpnum} field, and invokes
\co{rcu_advance_cbs()} to advance callbacks.
%
Finally, if another grace period is needed,
we set \co{rsp->gp_flags} to \co{RCU_GP_FLAG_INIT}.
Then in the next iteration of the outer loop, the grace-period kthread
will initialize a new grace period as discussed above.

% Lihao: understand how nodes in the tree sync with information for each grace period

% Lihao: Tree RCU starts a new grace by calling rcu_gp_kthread_wake() that wakes up
% the rcu_gp_kthread() kernel thread which does the clean up and invokes rcu_gp_init()
% to start a new grace period

% Lihao: other places that may start a new grace period
% 1. rcu_check_quiescent_state() calls note_gp_changes() that checks
% rcu_accelerate_cbs() or rcu_advance_cbs()
% 2. rcu_report_qs_rdp() by checking rcu_accelerate_cbs()
% 3. __rcu_process_callbacks() by checking cpu_needs_another_gp and rcu_start_gp()
% which in turn calls rcu_advance_cbs() and rcu_start_gp_advanced
% 4. __call_rcu_core by checking rcu_start_gp()
% 5. force_quiescent_state()

\section{Verification Scenario}
% \comment{Lihao: alternative titles: A Running Example? Putting Things Together?}

We use the example in Figure~\ref{fig:verify_rcu_gp} to demonstrate how the
different components of Tree RCU work together to guarantee that all
pre-existing read-side critical sections finish before RCU allows a grace
period to end.  This example will drive the verification, which will check
for violations of the assertion at this end of the code.

We focus on the implementation of the non-preemptible RCU-sched flavor.  We
further assume there are only two CPUs, and that CPU~0 executes function
\co{rcu_reader()} and CPU~1 executes \co{rcu_updater()}.  When the system
boots, the Linux kernel calls \co{rcu_init()} to initialize RCU, which
includes constructing the combining tree of \co{rcu_node} and \co{rcu_data}
structures via \co{rcu_init_geometry()} and initializing the fields of the
nodes in the tree for each RCU flavor via \co{rcu_init_one()}.  In our
example it will be a one-level tree that has one \co{rcu_node} structure as
root and two children that are \co{rcu_data} structures for each CPU. 
Function \co{rcu_spawn_gp_kthread()} is also called to initialize and spawn
the RCU grace-period kthread for each RCU flavor.

Referring again to Figure~\ref{fig:verify_rcu_gp},
suppose that \co{rcu_reader()} begins
execution on CPU~0 while \co{rcu_updater()} concurrently sets \co{x} to 1
and then invokes \co{synchronize_rcu()} on CPU~1.
As discussed in Section \ref{sec:api_impl}, \co{synchronize_rcu()}
invokes \co{wait_rcu_gp()}, which in turn registers an RCU callback
that will invoke \co{wakeme_after_rcu()} some time after \co{rcu_reader()}
exits its critical section.

However, this critical-section exit has no immediate effect.
Instead, a later context switch will invoke
\co{rcu_note_context_switch()}, which in turn invokes
\co{rcu_sched_qs()}, recording the quiescent state in the
CPU's \co{rcu_sched_data} structure's \co{->passed_quiesce} field.
Later, a scheduling-clock interrupt will invoke
\co{rcu_check_callbacks()}, which calls \co{rcu_pending()} and 
notes that the \co{->passed_quiesce} field is set.
This will cause \co{rcu_pending()} to return \co{true}, which
in turn causes \co{rcu_check_callbacks()} to invoke
\co{rcu_process_callbacks()}.
In its turn, \co{rcu_process_callbacks()} will invoke
\co{raise_softirq(RCU_SOFTIRQ)}, which,
once the CPU has interrupts, preemption, and
bottom halves enabled, %(Section \ref{sec:timer_interrupt}),
calls \co{rcu_process_callbacks()}.

As discussed in Section \ref{sec:rcu_softirq}, RCU's softirq handler function \co{rcu_process_callbacks()} 
first calls \co{rcu_check_quiescent_state()} to report any recent quiescent states on the 
current CPU (CPU~0). Then it checks whether the CPU~0 has passed a quiescent state. Since 
a quiescent state has been recorded for CPU~0, \co{rcu_report_qs_rnp()} is invoked to traversal
up the combining tree. It clears the first bit of the root \co{rcu_node} structure's \co{qsmask} 
field (recall that the RCU combining tree has only one level). Since the second bit for CPU~1 has 
not been cleared, the function returns.

Since \co{synchronize_rcu()} blocks in CPU~1, it will result in a context switch. 
This triggers a sequence of events similar to that described above for
CPU~1, which results in the clearing of the
second bit of the root \co{rcu_node} structure's \co{->qs_mask} field, the value of which is now 0, indicating the end of the current grace period.
CPU~1 therefore invokes \co{rcu_report_qs_rsp()} to 
awaken the grace-period kthread, 
which will clean up the ended grace period, and, if needed, 
start a new one (Section \ref{sec:rcu_gp_kthread}).

Lastly, \co{rcu_process_callbacks()} calls \co{invoke_rcu_callbacks()} to invoke any callbacks whose
grace period has already elapsed, for example, \co{wakeme_after_rcu()},
which will allow \co{synchronize_rcu()} to return.

%\comment{Lihao: When CPU~1 is waiting for \co{synchronize_rcu()} to return, how does it reach a 
%quiescent state? Is it via a scheduling-clock interrupt? What kind of quiescent states would it be?}
%\comment{Paul: Any number of possibilities.
%	First, \co{synchronize_rcu()} blocks, which results in a context
%	switch.
%	This context switch acts as a quiescent state, and a later
%	scheduling-clock interrupt would notice this and cause
%	\co{RCU_SOFTIRQ} to run, thus reporting the queiscent state
%	to the RCU core code.
%	Second, it is possible that there was nothing else for the
%	CPU to run, so that it went idle.
%	In this case, the grace-period kthread might notice that the CPU
%	was idle before the CPU got around to reporting the context switch
%	to the RCU core code.
%	Third, the context switch might result in a task running
%	in usermode.
%	In this case, a subsequent scheduling-clock interrupt causing
%	\co{RCU_SOFTIRQ} to run might
%	report the userspace-execution quiescent state to the RCU
%	core code.
%	Fourth, this might be a \co{CONFIG_NO_HZ_FULL} kernel.
%	In that case, the RCU grace-period kthread could note
%	the userspace execution in the same way that it might note
%	the idle loop.
%	Hey, you asked!
%}
% Lihao: understand when/where in the code \co{wakeme_after_rcu()} gets moved to the head of the 
% \co{->nxttail} to be invoked

