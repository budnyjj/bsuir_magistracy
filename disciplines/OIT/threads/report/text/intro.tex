\section*{ВВЕДЕНИЕ}
\addcontentsline{toc}{section}{Введение}

Современные операционные системы предоставляют средства для
многопоточного программирования.
Каждая исполняемая программа в рамках своего процесса
может иметь один или несколько потоков выполнения.
Поток выполнения --- это наименьшая единица обработки задачи,
исполнение которой может быть назначено ядром операционной
системы~\cite{wiki_thread}.
Операционная система выделяет каждому потоку некоторый короткий
промежуток времени, в течение которого происходит его выполнение.
По истечении этого промежутка активный
поток блокируется, и операционная система переходит к следующему.
Таким образом достигается видимость одновременного выполнения
нескольких потоков в рамках одного процесса --- многопоточность.

Основным отличием потоков выполнения от процессов является
использование общего адресного пространства.
Потоки в рамках одного процесса выполнения имеют общий доступ к данным
и другим ресурсам, предоставляемых данному процессу операционной системой.

Существуют различные программные реализации многопоточности.
В различных ОС потоки могут быть реализованы на уровне ядра,
в пользовательском пространстве, или с использованием различных гибридных схем.
Каждый из этих подходов имеет ряд достоинств и недостатков.

Программный интерфейс управления потоками также зависит от конкретной
операционной системы. Тем не менее, существуют стандартизированные
программные интерфейсы управления потоками.
Наиболее известным из них является стандарт управления потоками POSIX.
В данном реферате производится обзор данного стандарта,
на его примере рассматриваются основные операции
управления потоками, а также базовые примитивы синхронизации.
Кроме этого, рассматриваются средства поддержки многопоточности,
предоставляемые стандартной библиотекой языка C++.
