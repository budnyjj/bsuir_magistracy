\section*{ВВЕДЕНИЕ}
\addcontentsline{toc}{section}{Введение}

Значительное число задач физики и техники приводят к дифференциальным
уравнениям в частных производных.
Точные решения краевых задач для данных уравнений удается получить лишь
в частных случаях, поэтому эти задачи в основном решают приближенно.
Одним из наиболее универсальных и эффективных методов,
получивших широкое распространение для приближенного решения
уравнений математической физики, является метод конечных разностей
или метод сеток~\cite{samarskij1978}.

Данная работа посвящена краткому описанию метода сеток,
а также различных методов построения конечно-разностных схем
в соответствии с методическим пособием~\cite{sinitsyn2007}.