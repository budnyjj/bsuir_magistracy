\section[Конечно-разностные схемы]{КОНЕЧНО-РАЗНОСТНЫЕ СХЕМЫ}

Большое разнообразие методов обусловлено возможностью
по-разному выбирать узыл и квадратурные формулы для апрроксимации интеграла
в~\eqref{eq:approximation} при получении схемы~\eqref{eq:diff_scheme}.

\subsection{Явная схема первого порядка (Эйлера)}

Заменив интеграл в~\eqref{eq:approximation} по формуле левых прямоугольников:
\begin{equation*}
  \dfrac{1}{h} \int^{x_{k+1}}_{x_k} \overline{f}(x, \overline{u}(x) dx \approx
  \dfrac{\overline{f}(x_k, \overline{y}^k) h}{h} =
  \overline{f}(x_k, \overline{y}^k),
\end{equation*}
получим
\begin{equation}
  \dfrac{\overline{y}^{k+1} - \overline{y}^k}{h} =
  \overline{f}(x_k, \overline{y}^k), \quad
  k = 0, 1, 2, \ldots, n.
  \label{eq:scheme_1}
\end{equation}

Задавая \( \overline{y}^0 = \overline{u}^0 \), с помощью~\eqref{eq:scheme_1}
можно получить все последующие значения
\( \overline{y}^k, k = 1, 2, \ldots, n+1 \), так как формула явно разрешается
относительно \( \overline{y}^{k+1} \).
Погрешность аппроксимации \( \psi(h) \) и соответственно точность
\( \varepsilon(h) \) имеют первый порядок в силу того,
что формула левых прямоульников на интервале
\( [x_k, x_{k+1}] \) имеет погрешность второго порядка, а схема устойчива.

\subsection{Неявная схема первого порядка}

Используя в~\eqref{eq:approximation} формулу правых прямоугольников, получим
\begin{equation}
  \dfrac{\overline{y}^{k+1} - \overline{y}^k}{h} =
  \overline{f}(x_{k+1}, \overline{y}^{k+1}), \quad
  k = 0, 1, 2, \ldots, n.
  \label{eq:scheme_2}
\end{equation}

Эта схема неразрешима явно относительно \( \overline{y}^{k+1} \),
поэтому для получения \( \overline{y}^{k+1} \) требуется использовать итерационную
процедуру решения уравнения~\eqref{eq:scheme_2}:
\begin{equation}
  \overline{y}^{k+1, s} = \overline{y}^k +
  h \overline{f}(x_{k+1}, \overline{y}^{k+1, s-1}), \quad
  s = 1, 2, \ldots - \text{номер итерации}.
\end{equation}

В качестве начального приближения берется значение
\( \overline{y}^{k+1, 0} = \overline{y}^k \) с предыдущего шага.
Преимущество неявной схемы заключается в том, что у неё константа устойчивости
\( c_0 \) значительно меньше, чем у явной схемы.

\subsection{Неявная схема второго порядка}

Используя в~\eqref{eq:approximation} формулу трапеций, получим
\begin{equation}
  \dfrac{\overline{y}^{k+1} - \overline{y}^k}{h} =
  \dfrac{
    \overline{f}(x_{k+1}, \overline{y}^{k+1}) +
    \overline{f}(x_{k+1}, \overline{y}^{k+1})}{
    2
  }, \quad
  k = 0, 1, 2, \ldots, n.
  \label{eq:scheme_3}
\end{equation}

Так как формула трапеций имеет третий порядок точности на интервале
\( [x_k, x_{k+1}] \), то погрешность аппроксимации \( \psi(h) \) --- второй.

Cхема \eqref{eq:scheme_3} неразрешима явно относительно \( \overline{y}^{k+1} \),
поэтому требуется использовать итерационную процедуру:
\begin{equation}
  \overline{y}^{k+1, s} = \overline{y}^k +
  \dfrac{h}{2} \bigg(
  \overline{f}(x_k, \overline{y}^k) +
  \overline{f}(x_{k+1}, \overline{y}^{k+1, s-1})
  \bigg),
\end{equation}
где \( s = 1, 2, \ldots, \) --- номер итерации, а
\( \overline{y}^{k+1, 0} = \overline{y}^k \).

\subsection{Схема предиктор-корректор (Рунге-Кутта) второго порядка}

Используя в~\eqref{eq:approximation} формулу средних, получим
\begin{equation}
  \dfrac{\overline{y}^{k+1} - \overline{y}^k}{h} =
  \overline{f}(x_{k+\frac{1}{2}}, \overline{y}^{k+\frac{1}{2}}), \quad
  k = 0, 1, 2, \ldots, n.
  \label{eq:scheme_4}
\end{equation}

Уравнение~\eqref{eq:scheme_4} разрешимо относительно \( \overline{y}^{k+1} \),
однако в правой части присутствует неизвестное значение
\( \overline{y}^{k+\frac{1}{2}} \) в середине отрезка \( [x_k, x_{k+1}] \).
Для решения этого уравнения существует следующий способ.
Вначале по явной схеме~\eqref{eq:scheme_1} рассчитывают
\( \overline{y}^{k+\frac{1}{2}} \) (предиктор):
\begin{equation*}
  \overline{y}^{k+\frac{1}{2}} = \overline{y}^{k} +
  \dfrac{h}{2} \overline{f}(x_k, \overline{y}^k),
\end{equation*}
а затем с помощью~\eqref{eq:scheme_4} ---
\( \overline{y}^{k+1} \) (корректор).
В результате схема оказывается явной и имеет второй порядок.
Заметим, что данная схема получается из схемы~\eqref{eq:scheme_3},
если в ней выполнять всего две итерации \( (s = 1, 2) \).

\subsection{Схема Рунге-Кутта четвертого порядка}

Используя в~\eqref{eq:approximation} формулу Симпсона, получим
\begin{equation}
  \dfrac{\overline{y}^{k+1} - \overline{y}^k}{h} =
  \dfrac{1}{6} \bigg(
  \overline{f}(x_{k}, \overline{y}^{k}) +
  4 \overline{f}(x_{k+\frac{1}{2}}, \overline{y}^{k+\frac{1}{2}}) +
  \overline{f}(x_{k+1}, \overline{y}^{k+1}) +
  \bigg).
  \label{eq:scheme_5}
\end{equation}
Так как формула Симпсона имеет пятый порядок точности, то погрешность
аппроксимации данной схемы имеет четвертый порядок.

Можно по-разному реализовать расчет неявного по \( \overline{y}^{k+1} \)
уравнения~\eqref{eq:scheme_5}, однако наибольшее распространение получил
следующий способ. Делают предиктор вида
\begin{align*}
  \overline{y}^{k+\frac{1}{2}, 1} &= \overline{y}^{k} + \dfrac{h}{2} \overline{f}(x_k, \overline{y}^k), \\
  \overline{y}^{k+\frac{1}{2}, 2} &= \overline{y}^{k} + \dfrac{h}{2} \overline{f}(x_{k+\frac{1}{2}}, \overline{y}^{k+\frac{1}{2}, 1}), \\
  \overline{y}^{k+1, 1} &= \overline{y}^{k} + \dfrac{h}{2} \overline{f}(x_{k+\frac{1}{2}}, \overline{y}^{k+\frac{1}{2}, 2}),
\end{align*}
а затем корректор по формуле
\begin{equation*}
  \overline{y}^{k+1} =
  \overline{y}^k + \dfrac{h}{6}
  \bigg(
  f(x_k, \overline{y}^k) +
  2 f(x_{k + \frac{1}{2}}, \overline{y}^{{k + \frac{1}{2}}, 1}) +
  2 f(x_{k + \frac{1}{2}}, \overline{y}^{{k + \frac{1}{2}}, 2}) +
  2 f(x_{k + 1}}, \overline{y}^{{k + 1}, 1})
  \bigg).
\end{equation*}
