\section[Метод сеток]{МЕТОД СЕТОК}

Пусть задана система дифференциальных уравнений первого порядка
\begin{equation}
  \left\{\begin{aligned}
    \dfrac{du_1(x)}{dx} &= f_1(x, u_1, \cdots, u_m), \\
    & \cdots \\
    \dfrac{du_m(x)}{dx} &= f_m(x, u_1, \cdots, u_m),
  \end{aligned}
  \right.
  \label{eq:formulation_diffs}
\end{equation}
а также имеется множество начальных ограничений вида
\( u_i(x_0) = u_{i,0} \).
Требуется найти значения \( \overline{u}(x) = (u_i(x)) \) для
\( x_0 \le x \le x_{n+1} \) (задача Коши).

Для получения приближенного решения данной задачи можно использовать метод сеток.
Суть данного метода состоит в следующем:
\begin{enumerate}
\item В области интегрирования выбирается упорядоченная система точек
  \( x_0 < x_1 < \cdots < x_{n+1} \), называемая сеткой.
  Точки \( x_i \) называются \emph{узлами},
  а величина \( h_k = x_k - x_{k-1} \) --- \emph{шагом сетки}.
  Для упрощения в дальнейшем будем считать сетку \emph{равномерной},
  т.~е. \( h_k = h = \dfrac{(x_{n+1} - x_{0})}{n+1} \).
\item Поиск решения \( \overline{u}(x) \) производится в виде
  таблицы значений в узлах выбранной сетки,
  для чего дифференциальное уравнение заменяется системой
  алгебраических уравнений, связывающих между собой значения
  искомой функции в соседних узлах.
  Такая система называется \emph{конечно-разностной схемой}.
\end{enumerate}

Существует несколько распространенных способов получения конечно-разностных схем.
Приведем один из наиболее универсальных --- \emph{интегро-интерполяционный метод}.
Согласно ему, для получения конечно-разностной схемы требуется выполнить
интегрирование всех уравнений системы~\eqref{eq:formulation_diffs}
на каждом интервале \( [x_k, x_{k+1} ]\),
а затем разделить их на длину этого интервала:
\begin{equation}
  \dfrac{1}{h} \int^{x_k+1}_{x_k} \dfrac{d \overline{u}}{dx} dx =
  \dfrac{\overline{u}^{k+1} - \overline{u}^{k}}{h} =
  \dfrac{1}{h} \int^{x_{k+1}}_{x_k} f(x, \overline{u}(x)) dx.
  \label{eq:approximation}
\end{equation}

\pagebreak

Интеграл в левой части~\eqref{eq:approximation} аппроксимируется с помощью
квадратурных формул, после чего получается система уравнений относительно
\emph{приближенных} неизвестных значений искомой функции,
обозначаемых \( \overline{y}^k \approx \overline{u}^k \):
\begin{equation}
  \dfrac{\overline{y}^{k+1} - \overline{y}^{k}}{h} =
  \dfrac{1}{h} \sum_j \alpha_j \overline{f}(x_j, \overline{y}^j), \quad
  x_k \le x_j \le x_{k+1}.
  \label{eq:diff_scheme}
\end{equation}

При замене интеграла приближенной квадратурной формулой вносится
\emph{погрешность аппроксимации} дифференциального уравнения разностным,
которая зависит от шага сетки:
\begin{equation}
  \psi_k(h) = \dfrac{1}{h}
  \Bigg|
  \int^{x_k+1}_{x_k} \overline{f}(x, \overline{u}(x))dx
  - \sum_j \alpha_j \overline{f}(x_j, \overline{y}^j)
  \Bigg|.
  \label{eq:approx_error}
\end{equation}

Говорят, что разностная схема~\eqref{eq:diff_scheme} аппроксимирует
исходную дифференциальную задачу с порядком \( p \), если при
\( р \rightarrow 0, \psi_k(h) \le ch^p, c=const \).
Из~\eqref{eq:approx_error} следует, что порядок аппроксимации
на единицу меньше, чем порядок погрешности используемой квадратурной формулы
на данном интервале. Таким образом, можно утверждать, что
чем больше порядок аппроксимации p, тем выше точность решения.

\emph{Основная теорема теории метода сеток} утверждает, что если схема
устойчива, то при \( h \rightarrow 0 \) погрешность решения \( \varepsilon(h) \)
стремится к нулю с тем же порядком, что и погрешность аппроксимации:
\begin{equation*}
  \varepsilon(h) \le
  c_0 \cdot \max_k \big| \psi_k(h) \big| \le
  c_0 \cdot c \cdot h^p,
\end{equation*}
где \( c_0 \) --- константа устойчивсти.

Неустойчивость обычно проявляется в том, что с уменьшением \( h \) решение
\( \overline{y}^k \rightarrow \infty \) при возрастании \( k \),
что легко проверяется экспериментально.
Таким образом, если имеется аппроксимация и схема устойчива, то, выбрав
достаточно малый шаг \( h \), можно получить решение с заданной точностью.
При этом затраты на вычисления резко уменьшаются с увеличением порядка
аппроксимации \( p \), т.~е. при большем \( p \) можно достичь той же точности,
используя более крупный шаг \( h \).

% \subsection{Алгоритм метода сеток}

% \subsection{Основная теорема теории метода сеток}
