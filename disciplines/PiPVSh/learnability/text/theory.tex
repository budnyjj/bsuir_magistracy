\section[Теоретические сведения]{ТЕОРЕТИЧЕСКИЕ СВЕДЕНИЯ}

\textit{Обучаемость} в широком смысле слова представляет собой
способность к усвоению знаний и способов действий,
готовность к переходу на новые уровни обученности.
В этом случае она выступает как проявление общих способностей учащегося,
отражающих познавательную активность субъекта и его возможности к
усвоению новых знаний, действий, сложных форм деятельности.
Выражая общие способности, обучаемость выступает как общая возможность
психического развития, достижения более обобщенных систем знаний,
общих способов действий.
Обучаемость характеризуется индивидуальными показателями
скорости и качества усвоения человеком знаний,
умений и навыков в процессе обучения.

Обучаемость в биологическом смысле есть аспект биологической защиты:
изменение среды до определенных пределов,
не превышающих возможности функционирования системы,
вызывает реакцию защитного типа,
посредством которой проявляется и устанавливается свойство обучаемости.
Обучаемость у человека является уже не единичным актом биологической защиты,
а включает в себя социальный опыт предшествующих поколений~\cite{mandel2007}.

При рассмотрении биологии развития обучаемости нужно ясно представлять себе,
что формирование способностей к обучаемости является функцией ряда органов и
функцией организма как целого. Можно выделить по крайней мере три фактора,
которые оказывают существенное влияние на процесс обучаемости:
\begin{enumerate}
\item Созревание самой нервной системы.
  В этом случае особенно большое влияние оказывает процесс
  миелинизации отростков нейронов, который продолжается и в постнатальном периоде.
\item Развитие инстинктивной основы, на базе которой формируется обучение.
\item Формирование типологических особенностей нервной системы
  (соотношение процессов возбуждения и торможения), которые, в свою очередь,
  находятся в большой зависимости от ряда гормональных и
  гуморальных факторов~\cite{krushinskij}.
\end{enumerate}

В психологии рассматриваются различные виды обучаемости:
\textit{обучаемость общая} как способность усвоения любого материала, и
\textit{обучаемость специальная} как способность усвоения его отдельных видов:
различных областей науки, искусства, направлений практической
деятельности.
Общая обучаемость зависит от уровня развития познавательных процессов субъекта:
восприятия, воображения, памяти, мышления, внимания, речи;
уровня развития его мотивационно-волевой и эмоциональной сфер;
уяснения содержания учебного материала из прямых и косвенных объяснений,
овладение материалом до степени активного применения.
Наряду с этим, исследования показали, что обучаемость различным навыкам
определяется в основном их спецификой~\cite{druzhinin2007}.

Обучаемость как индивидуальное, относительно устойчивое свойство личности,
тесно связана с умственным развитием, однако эти понятия не тождественны.
Высокая обучаемость способствует более интенсивному умственному развитию,
при этом с высоком умственным развитием может сочетаться и относительно низкая обучаемость,
которая компенсируется большой усидчивостью ученика.

Обучаемость на протяжении длительного периода может оставаться относительно постоянной,
а умственное развитие с возрастом повышается и характеризуется рядом показателей:
запасом знаний и степенью их систематизации,
владением рациональными приёмами устной деятельности.
Эти показатели применительно к обучаемости приобретают иной характер,
поскольку здесь имеют значение сам процесс формирования знаний,
степень легкости и быстроты их приобретения, овладения приёмами умственной деятельности.
Таким образом, интегральные показатели, изменение которых позволяет контролировать
уровень когнитивного развития учащихся, управлять им и совершенствовать качество учебного процесса, --- это показатели когнитивного развития. К ним относятся:
\begin{itemize}
\item целостность восприятия;
\item словарный запас, осведомленность;
\item устойчивость и концентрация внимания;
\item уровень развития зрительной и слуховой памяти, способы запоминания
  (механический и смысловой компоненты);
\item владение операциями мышления: классификация, обобщение, аналогии, закономерности.
\end{itemize}

При обучении необходимо определять и учитывать динамические характеристики деятельности:
работоспособность, утомляемость и свойства мыслительной деятельности:
самостоятельность, обобщенность и др.

На основе диагностики свойств познавательных процессов выделяется четыре уровня обучаемости:
нулевой, первый, второй, и третий.
Нулевой и первый уровни соответствуют низкой обучаемости,
второй --- средней, третий --- высокой.
Рассмотрим их подробнее.

\textit{3 уровень}. Ученик работает быстро и имеет высокую умственную работоспособность,
свободно владеет операциональными способами освоения знаний
(сравнение, анализ, ассоциирование аналогических связей,
умение делать обобщающие выводы, видение причинно-следственных связей,
выделяет существенные признаки); имеет большой словарный запас; легко принимает помощь.
Мыслительная деятельность характеризуется обобщенностью, осознанностью, самостоятельностью.

\textit{2 уровень}. Ученик обладает умениями всех умственных действий на среднем уровне;
имеет средний уровень словарного запаса; к помощи восприимчив;
развитие свойств внимания и памяти в пределах нормы; динамика работоспособности положительная.
Мыслительная деятельность характеризуется средним уровнем обобщенности, осознанности,
гибкости и самостоятельности.

\textit{1 уровень}. Учащийся характеризуется низким уровнем владения операциональными способами
освоения знаний, как следствие, у него недостаточная самостоятельность, гибкость мышления;
бедность словарного запаса; низкая общая работоспособность и интенсивность деятельности;
недостаточное развитие умений запоминания и воспроизведения материала;
невнимательность при восприятии материала; низкая мотивация учебной деятельности.

\textit{0 уровень}. Ученик характеризуется низким уровнем владения операциональными способами
освоения знаний, как следствие, недостаточная гибкость мышления;
не умеет различать существенные и несущественные признаки понятий;
плохо организует самостоятельную деятельность;
уклоняется от активной умственной деятельности;
медленный темп работы, быстрая утомляемость, рассеянность.

Эта информация лежит в основе прогнозируемых результатов
успешности развития и обучения~\cite{zimnyaya1997}.