\section*{ВВЕДЕНИЕ}
\addcontentsline{toc}{section}{Введение}

Научно-техническая революция и небывалый рост населения Земли обусловили
потребность в организации системы образования.
В настоящее время данная система носит массовый и стандартизированный характер,
сопровождая человека в течение всей его жизни.
Детский сад, школа, университет --- деятельность этих и подобных им
институтов направлена на то,
чтобы научить индивида ориентироваться в сложной и, зачастую,
противоречивой окружающей действительности.

Обучение можно рассматривать как процесс, объектом которого является
обучаемый (ученик), а одной из его важнейших характеристик ---
длительность обучения, под которой понимается время, затрачиваемое на то,
чтобы обучаемый усвоил материал.
Замечено, что данная величина не является постоянной и зависит от множества факторов.
Ввиду массовости процесса обучения существует необходимость в её минимизации.

Ученые мира с древних времен занимаются изучением факторов,
определяющих эффективность обучения.
Одним из таких факторов считается обучаемость, под которой понимается
<<восприимчивость ученика к усвоению новых знаний и новых способов их добывания,
а также готовность к переходу на новые уровни умственного развития>>~\cite{markova1990}.

В данной работе рассматривается обучаемость как общая познавательная способность человека:
перечисляются основные теоретические положения, описывающие данный феномен,
а также приводится ряд связанных с этим личных практических соображений.
