\section[Адаптивная сеть на основе системы нечеткого вывода]{%
  АДАПТИВНАЯ СЕТЬ НА ОСНОВЕ СИСТЕМЫ НЕЧЕТКОГО ВЫВОДА
}

- гибридная сеть
- примеры: И, ИЛИ
- описание сети

% Целью данного раздела является описание отношения автора работы
% к рассматриваемой теме. К сожалению, в настоящее время я не веду
% преподавательской деятельности, а также не занимаюсь исследованиями
% в данной области, а посему мои рассуждения не претендуют на то,
% чтобы носить сколь-нибудь научный характер.
% Скорее, их следует рассматривать как
% неформальную рефлексию по поводу изучения рассматриваемой темы.

% Понятие обучаемости выходит за рамки психологии и педагогики.
% Насколько я понимаю, оно характеризует скорость приобретения навыков в широком смысле.
% Если исходить из этого, получается, что собаку, наученную давать лапу по команде хозяина,
% следует признать обученной, и, следовательно, обучаемой.

% Возникает вопрос: где находится грань, разделяющая организмы на обучаемые и необучаемые?
% Судя по всему, теплокровность или, скажем, наличие позвоночника,
% не могут однозначно свидетельствовать о способности к обучению:
% с одной стороны, условные рефлексы проявляются даже у весьма непохожих на нас
% представителей животного мира (например,~\cite{tarakany});
% c другой стороны, внутри каждого вида существуют его представители,
% практически не способные к обучению.
% По-видимому, для такого разделения можно использовать как минимум два подхода:
% \begin{enumerate}
% \item Оценивать сложность нервной системы.
% \item Рассматривать обучаемость как производную от процесса обучения в неформальном смысле.
% \end{enumerate}

% Каждый из них имеет свои недостатки.
% Например, на данный момент мне неизвестны точные методы оценки и
% сравнения сложности нервных систем.
% Кроме этого, результаты подобной оценки будут говорить лишь о максимально возможном
% уровне обучаемости исследуемого вида на данном этапе его развития.
% Второй подход неявно преполагает факт успешного обучения,
% чего может не быть на практике по причинам, не зависящим от <<ученика>>.
% Что, если мы имеем дело с <<плохим учителем>>, принципиально неспособным его научить?

% Другой вопрос связан с соотношением между общей и специальной обучаемостью.
% Результаты исследований говорят о том, что обучение индивида некоторому виду деятельности
% не сказывается (по крайней мере, статистически) на том, как он преуспевает в прочих.
% Это говорит о том, что общая обучаемость является понятием синтетическим,
% не имеющем отражения в реальном мире.
% Лично я рассматриваю её как совокупность приобретенных метанавыков,
% влияющих на ход конкретного процесса обучения.
% Данные навыки формируются косвенным образом в результате обучения
% (обратная связь), характеризующегося специальной обучаемостью.
% Таким образом, получается, что специальная обучаемость определяет общую, а не наоборот.

% Следует отдельно подчеркнуть роль мотивации в процессе обучения.
% <<Важно четко осознавать необходимость в совершении любых своих действий:
% совершать их в случае осознанной необходимости и не совершать их в противном случае>>.
% В контексте обучения этот принцип выражается в следующих рекомендациях:
% \begin{itemize}
% \item учитель должен объяснять ученикам,
%   зачем он собирается преподавать им какой-либо материал;
% \item каждый ученик на основании объяснений учителя должен решить,
%   будет ли он этот материал изучать.
% \end{itemize}

% Мне представляется, что сознательное обучение является наиболее эффективным.
% Тем не менее, следует признать, что часто на практике стоит задача обучения вопреки
% желанию ученика, что делает мои рекомендации неприменимыми.
