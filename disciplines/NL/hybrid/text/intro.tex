\section*{ВВЕДЕНИЕ}
\addcontentsline{toc}{section}{Введение}

Математический аппарат нечеткой логики находит широкое применение
в различных отраслях экономики: металлургии, автомобилестроении,
логистике, медицине и др.~\cite{terano92}.
Основными прикладными задачами, решаемыми с помощью нечеткого логического вывода,
являются задачи оценивания и управления в условиях неполноты или
недостаточной точности входных данных.
К достоинствам нечеткого подхода обычно причисляют простоту реализации
и высокую эффективность его работы.
Основным его недостатком является необходимость в услугах эксперта,
способного составить исходный набор правил системы вывода.

Искусственные нейронные сети явялются конкурирующим направлением развития
искусственного интеллекта.
Как и системы нечеткого вывода, нейросети позволяют получать ответы
приемлемой точности в задачах, точное решение которых является невозможным
или нецелесообразным.
Кроме этого, при использовании нейросетей отпадает необходимость в
услугах эксперта, поскольку их настройка производится на этапе обучения
автоматически.
К недостаткам нейронных сетей обычно причисляют отсутствие
детерминированной зависимости между их топологией,
разрешающей способностью (числом искусственных нейронов),
размером обучающей выборки и качеством решения задачи.
Складывается впечатление, что необходимость эмпирического подбора параметров сети
для того, чтобы она наилучшим образом подходила для каждой конкретной задачи,
сдерживает темпы роста использования данного подхода в промышленных условиях.

В данной работе приводятся необходимые сведения из
теории нечеткого логического вывода и теории искусственных нейронных сетей,
затем на их основании рассматривается гибридный подход,
сочетающий преимущества результатов данных теорий.
Целью работы является изложение принципа работы \emph{ANFIS} ---
адаптивной сети на основе системы нечеткого
вывода~\cite{Jang93anfis:adaptive-network-based}.
Изложение материала производится на основании~\cite{kruglov2001}.
