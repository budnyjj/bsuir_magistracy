\section*{ВВЕДЕНИЕ}
\addcontentsline{toc}{section}{Введение}

Математический аппарат нечеткой логики находит широкое применение
в различных отраслях экономики: металлургия, автомобилестроение,
логистика, медицина и др.~\cite{terano92}.
Основными прикладынми задачами, решаемыми с помощью нечеткого логического вывода,
являются задачи оценивания и управления в условиях неполноты или
недостаточной точности входных данных.
К достоинствам данного подхода обычно причисляют простоту реализации,
низкую вычислительную сложность, высокую эффективность работы.
Основным его недостатком является необходимость в
привлечении эксперта, способного составить исходный набор правил системы вывода.

Еще одним направлением развития искусственного интеллекта являются нейронные сети.
В их работе можно выделить два этапа:
\begin{enumerate}
\item Обучение, в ходе которого производится настройка сети на основании
  учебных примеров исходных данных, носящих репрезентативный характер.
\item Нормальный режим работы, в ходе которого решается поставленная задача
  классификации, прогнозирования и~т.~п. на основании реальных исходных данных.
\end{enumerate}

Нейронные сети позволяют получать ответы приемлемой точности в задачах,
точное решение которых является невозможным или нецелесообразным.
Кроме этого, при их использовании отпадает необходимость в услугах эксперта,
поскольку её настройка производится на этапе обучения автоматически.

К недостаткам нейронных сетей обычно причисляют отсутствие
детерминированной зависимости между её топологией,
разрешающей способностью (числом искусственных нейронов),
размером обучающей выборки и качеством решения задачи.
Складывается впечатление, что необходимость эмпирического подбора параметров сети
для того, чтобы она наилучшим образом подходила для каждой конкретной задачи,
сдерживает темпы роста использования данного подхода для
решения широкого прикладных задач в различных отраслях народного хозяйства.

В данной работе рассматривается гибридный подход,
основанный на идеях нечеткого логического вывода и
искусственных нейронных сетей.
Целью работы является изложение принципа работы \emph{ANFIS} ---
адаптивной сети на основе системы нечеткого
вывода~\cite{Jang93anfis:adaptive-network-based}.
Изложение основных понятий, необходимых для понимания работы ANFIS,
производится на основании~\cite{kruglov2001}.
