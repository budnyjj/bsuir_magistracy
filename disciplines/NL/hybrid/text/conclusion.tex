\addcontentsline{toc}{section}{Заключение}
\section*{ЗАКЛЮЧЕНИЕ}

В данной работе были рассмотрены основные положения теории нечеткого
логического вывода, теории искусственных нейронных сетей,
а также гибридный подход, сочетающий преимущества данных теорий.

Действительно, для работы систем нечеткого логического вывода
требуются наличие набора правил с априорно известными мерами истинности
каждого из них. Для его составления на практике требуется помощь
стороннего человека --- эксперта в данной области.
Кроме этого, это ограничивает разрешающую способность данных систем.
Объединение систем нечеткого вывода с искусственными нейронными
сетями позволяет решить данную проблему путем автоматизации
процесса составления базы знаний путем за счет ее обучения.
С другой стороны, подобное объединение позволяет зафиксировать
топологию нейросети, делает смысл её коэффициентов наглядным.

Следует отметить, что существуют и альтернативные подходы к оптимизации
топологии и параметров сети. Для этих целей можно использовать, например,
генетический алгоритм, как было предложено в~\cite{Tahmasebi201218}.
