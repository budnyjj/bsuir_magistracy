\section*{ВВЕДЕНИЕ}
\addcontentsline{toc}{section}{Введение}

\emph{Система поддержки принятия решений} (СППР) (англ. Decision Support System, DSS) ---
компьютерная автоматизированная система, целью которой является помощь людям, принимающим
решение в сложных условиях для полного и объективного анализа предметной деятельности.

Современные СППР представляют собой системы, максимально приспособленные к решению
задач повседневной управленческой деятельности, являются инструментом, призванным
оказать помощь \emph{лицам, принимающим решения} (ЛПР).
С помощью СППР может производится выбор решений некоторых неструктурированных и
слабоструктурированных задач, в том числе и многокритериальных.
СППР, как правило, являются результатом мультидисциплинарного исследования,
включающего теории баз данных, искусственного интеллекта, интерактивных
компьютерных систем, методов имитационного моделирования~\cite{popov2008}.

В данной работе рассматриваются системы поддержки и принятия решений,
используемые в банковской сфере.
Целью работы является обзор существующих решений,
предназначенных для кредитного скоринга.
Данный выбор обусловлен наличием достаточного количеством материала для анализа,
находящегося в открытом доступе, а также сравнительно низким порогом вхождения
в предметную область.
