\section[Классификация систем поддержки и принятия решений]{%
  КЛАССИФИКАЦИЯ СИСТЕМ ПОДДЕРЖКИ И \\ ПРИНЯТИЯ РЕШЕНИЙ
}

С целью краткого введения в предметную область систем поддержки
и принятия решений выполним их классификацию.
В отечественной литературе~\cite{larichev1987} выделяют следующие признаки
для классификации СППР:
\begin{itemize}
\item концептуальные модели;
\item пользователи системы;
\item решаемые задачи;
\item обеспечивающие средства;
\item области применения.
\end{itemize}

С точки зрения используемой концептуальной модели различают
\emph{специализированные СППР}, с которыми работают конечные пользователи,
служащие для поддержки решения отдельных прикладных задач в конкретных ситуациях;
\emph{СППР-генераторы}, представляющие собой пакеты связанных программных средств
поиска и выдачи данных, моделирования и т.~п., которые используются для разработки
специализированных систем и
\emph{СППР-инструментарий}, соответствующий высшему уровню технологического развития,
и предствляющий в распоряжение разработчиков наиболее мощные средства, в том числе
специализированные языки, операционные системы, средства ввода-вывода и отображения
информации и прочее.

С точки зрения способов взаимодействия с пользователем различают следующие
режимы работы СППР:
\begin{itemize}
\item \emph{терминальный режим} --- ЛПР работает непосредственно с
  системой в интерактивном режиме и сам конструирует запросы к системе,
  получает, интерпретирует и использует её ответы в процессе принятия
  решения или для поиска дополнительной информации;
\item \emph{режим <<клерка>>} --- ЛПР работает с системой в режиме непрямого доступа,
  конструируя запросы, которые затем обрабатываются системой с использованием
  кодируемых форм; ожидая получения ответов, ЛПР может заниматься другой работой;
\item \emph{режим посредника} --- ЛПР использует систему через посредников,
  которые, получив запросы руководителя, формализуют их, выполняют с помощью
  системы анализ проблемы, фильтруют и интерпретируют выдаваыемые системой результаты;
\item \emph{автоматизированный режим} --- ЛПР получает стандартные, повторяющиеся сообщения,
  которые автоматически генерируются системай; ЛПР использует выдаваемую системой
  информацию совместно с информацией, получаемой из других источников.
\end{itemize}

Из всего множества проблем, решаемых СППР, выделяют \emph{уникальные} и \emph{повторяющиеся},
\emph{целостные} и \emph{многокритериальные}, а также решаемые с помощью \emph{объективных моделей},
и \emph{моделей, основанных на предпочтениях ЛПР}. Декартово произведение приведенных дихотомий
задает восемь классов задач принятия решений, каждому из которых могут быть поставлены в соответствие
различные методы принятия решений и использующие их СППР.

По специфике используемого программного обеспечения СППР подразделяются на семь типов:
\begin{itemize}
\item \emph{системы извлечения данных}, обеспеяивающие возможность непосредственного
  доступа к отдельным элементам данных;
\item \emph{системы анализа данных}, позволяющие манипулировать данными,
  используя специально разработанные средства или средства общего назначения;
\item системы анализа информации, обеспечивающие доступ к нескольким базам данных
  и небольшим моделям;
\item \emph{расчетные модели}, позволяющие оценивать последствия на основе вычисляемых процедур;
\item \emph{изобразительные модели}, позволяющие оценивать последствия действий на основе
  частично определенных имитационных моделей;
\item \emph{оптимизационные модели}, обеспечивающие выбор направления действий,
  генерируя оптимальные решения, удовлетворяющие набору ограничений;
\item \emph{рекомендующие модели}, позволяющие сгенерировать наборы альтернатив
  в случае слабоструктурированных задач.
\end{itemize}

Классификация СППР по профессиональным сферам деятельности является
достаточно естественной и аналогична классификации других автоматизированных систем.
В соответствии с ней, выделяют СППР, предназначенные для
\emph{стратегического планирования и руководства},
\emph{планирования и прогнозирования деятельности предприятий},
\emph{автоматизации конторского труда},
\emph{медицинские}, и~т.~д.