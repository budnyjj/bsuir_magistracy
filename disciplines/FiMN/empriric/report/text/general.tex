\section[Общие методы эмпирического исследования]{%
  ОБЩИЕ МЕТОДЫ ЭМПИРИЧЕСКОГО \\
  ИССЛЕДОВАНИЯ
}

\subsection{Наблюдение}

\emph{Наблюдение} --- целенаправленное пассивное изучение предметов,
опирающееся в основном на данные органов чувств~\cite{nekrasov2010}.
Оно предполагает минимальное влияние на активность объекта и максимальное
использование возможностей естественных органов чувств субъекта.
При этом различные посредники, используемые в процессе наблюдения
(различного рода приборы), предназначены лишь для количественного усиления
различительной способности органов чувств субъекта~\cite{moiseev2004}.

Различают различные методы наблюдения:
\emph{вооруженное} (с использованием приборов) и \emph{невооруженное} (без их использования);
\emph{полевое} (в естественной среде существования объекта) и
\emph{лабораторное} (в исскуственной среде, специально приспособленной для наблюдения).

Результат наблюдения представляет собой множество уникальных сведений,
присущих объекту в данном месте и времени наблюдения.
Данные результаты являются основой фактов, на которых строятся дальнейшие рассуждения.

Следует отметить, что научное наблюдение концептуально не отличается
от наблюдения обыденного, совершаемого в быту.
Тем не менее, для обеспечения достоверности результатов научного наблюдения
необходимо выполнить ряд требований:
\begin{enumerate}
\item Обеспечить высококачественное восприятие объекта достаточной длительности.
  Ошибки измерения должны быть значительно ниже действительных
  значений измеряемых параметров объекта.
  Как правило, это достигается либо за счет использования приборов высокой точности,
  либо за счет использования статистических методов обработки многократных измерений.
\item Убедиться в том, что факт наблюдения за объектом не влияет существенным
  образом на его поведение (\emph{нейтральность наблюдения}).
  Следует отметить, что существуют такие объекты исследования,
  по отношению к которым выполнить данное требование принципиально невозможно
  (поведение элементарных частиц).
\item Выполнить требование \emph{интерсубъективности наблюдения}, или, иными словами,
  позаботиться о том, чтобы результаты наблюдения не зависили от субъекта наблюдения.
\end{enumerate}

Кроме этого, важно, чтобы сам процесс наблюдения был беспристрастным:
все наблюдаемые факты должны регистрироваться безотносительно различных
гипотез их объяснения (\emph{теоретическая ненагруженность наблюдения}).
Сами же попытки объяснения наблюдаемых фактов должны выполняться после
окончания процесса наблюдения.

Существует два крайних подхода к исследованию объектов посредством наблюдения.
Сторонники первого, называемого \emph{феноменолизмом}, утверждают,
что выполнять наблюдение можно лишь посредством внешних органов чувств.
Сторонники второго, называемого \emph{ноуменализмом} утверждают,
что наряду со зрением, слухом, вкусом, обонянием и осязанием
могут использоваться и внутренние механизмы восприятия,
такие, как интуиция, интеллектуальное созерцание, интроспекция.
По-видимому, выбор подхода зависит от объекта наблюдения.
Всякий раз, когда субъект наблюдения совпадает с объектом,
применим второй подход.

{\color{red} Найти еще материал!}

\pagebreak

\subsection{Эксперимент}

\emph{Эксперимент} --- активное и целенаправленное вмешательство в протекание изучаемого процесса,
его воспроизведение в специально созданных и контролируемых условиях~\cite{nekrasov2010}.
Основным отличием эксперимента от наблюдения является тот факт, что в ходе эксперимента
субъект выполняет направленное воздействие на объект, при этом направление и характер
воздействия определяется принятой научной теорией.
Таким образом, эксперимент является более теоретически нагруженным методом исследования,
чем наблюдение.

Перечислим основные классы экспериментов.
Эксперимент называется \emph{прямым}, если в нем участвует непосредственно объект исследования,
и \emph{модельным}, если вместо объекта исследования используется его модель.
Как и в случае с наблюдением, различают \emph{полевой} и \emph{лабораторный} эксперименты.
По основанию цели различают \emph{поисковый эксперимент}, при котором исследуется влияние
какого-либо фактора на объект исследования,
\emph{измерительный эксперимент}, в ходе которого выполняется сложное измерение
параметров объекта и \textit{проверочный эксперимент}, выполняемый с целью проверки
достоверности некоторой гипотезы.
По основанию используемых методов различают эксперименты на основе метода проб и ошибок,
эксперименты, построенные с использованием определенного алгоритма;
эксперименты, проводимые по методу <<черного ящика>>
(на основании знания о функции объекта исследования предполагается его структура) и
проводимые по методу <<белого ящика>>
(на основании знания о структуре объекта предполагаются его функции)~\cite{moiseev2004}.

В качестве примера рассмотрим порядок проведения так называемого
\emph{каузального эксперимента}, являющегося наиболее типичным
и важным видом экспериментального исследования.
В качестве исходных данных выступает собственно объект эксперимента,
а также некоторая каузальная гипотеза, состоящая в утверждении,
что фактор А является причиной (одной из причин) проявления фактора В.
Целью каузального эксперимента является подтверждение или опровержение данной гипотезы.
\begin{enumerate}
\item С точки зрения эксперимента факторы A и B являются существенными,
  а остальные факторы, способные повлиять на объект исследования,
  рассматриваются как несущественные.
  Для проведения эксперимента требуется посторить такую систему условий
  для объекта исследования, при которой несущественные факторы будет ослаблены,
  а существенные --- усилены.
  Для этого используется два метода --- \emph{удаление} и \emph{рандомизация}.
  В первом случае производится удаление несущественных факторов из
  объекта исследования, а во втором --- проведение эксперимента на достаточном
  множестве объектов со случайно изменяемыми значениями несущественных параметров
  с последующей статистической обработкой полученных результатов.
\item Каждый из существенных факторов A и B представляется как измеримая величина.
  Для этой цели могут привлекаться различные методики измерения и нормировки.
\item Формируются две группы исследуемых объектов, одна из которых называется
  \emph{контрольной группой}, а другая --- \emph{экспериментальной группой}.
  Соответствующие объекты этих двух групп должны быть как можно более
  похожи друг на друга.
\item Для каждого элемента экспериментальной группы создают ситуацию
  возникновения или усиления фактора A.
\item В случае, если в экспериментальной группе вслед за возникновением
  или изменением фактора A возникает или изменяется фактор B,
  при этом в контрольной группе подобное изменение не проявляется,
  делают вывод о существовании определенной вероятности каузальной гипотезы,
  т.~е.~о том, что фактор A является по крайней мере одной их причин для фактора В.
\end{enumerate}

{\color{red} Найти еще материал!}

\pagebreak

\subsection{Измерение}

\emph{Измерение} --- совокупность действий, выполняемых при помощи средств измерений
с целью нахождения числового значения измеряемой величины в принятых единицах
измерения~\cite{nekrasov2010}.
Измерение является неотъемлемой частью как наблюдения, так и эксперимента.

В общем случае процесс измерения предполагает наличие объекта измерения и некоторой шкалы,
на основе которой производится измерение.
\textit{Шкала} --- это специальная математическая структура, состоящая из множества элементов,
операций и отношений на этих элементах.
Процесс измерение представляет из себя процедуру отнесения объекта к тому или иному элементу
данной шкалы~\cite{moiseev2004}.
Такой процесс можно еще называть \emph{квантификацией} --- установлением количественных
определений объекта.
Выделяют 4 основных вида шкал:
\begin{itemize}
\item номинальные;
\item порядковые;
\item интервальные;
\item шкалы отношений.
\end{itemize}

Каждый последующий тип шкалы в этом списке является более сложным,
сохраняя все свойства предыдущего вида шкалы и добавляя к ним некоторые новые средства измерения.
Кратко рассмотрим основные свойства названных типов шкал.

Номинальная шкала представляет собой множество элементов с заданным на нем отношением равенства.
Данного отношения достаточно для того, чтобы отождествлять или различать элементы шкалы.
Процесс измерения в этом случае представляет из себя процедуру присваивания объектам тех
или иных элементов шкалы идентификаторов, позволяющих отличать объекты между собой.
Нетрудно заметить, что в рамках номинальной шкалы процесс квантификации является весьма
поверхностным, ограничиваясь лишь первичной квантификацией объектов.

Шкала порядка представляет собой множество элементов с заданным на нем отношением
строгого порядка \( R(a, b) \). Данное отношение удовлетворяет следующим трем свойствам:
\begin{itemize}
\item нерефлексивности:
  \( \forall a \in \Omega: \overline{R(a, a)} \);
\item несимметричности:
  \( \forall a, b \in \Omega: R(a, b) \rightarrow \overline{R(b, a)} \);
\item транзитивности:
  \( \forall a, b, c \in \Omega: R(a, b) \& R(b, c) \rightarrow R(a, c) \).
\end{itemize}

Простейшим примером подобного отношения является отношение <<меньше>> (\( < \)).
Задание такого отношения позволяет упорядочивать элементы шкалы.
Однако в такой шкале еще нельзя определить, насколько один элемент больше или меньше другого,
поскольку разницы элементов шкалы не являются элементами данной шкалы.

В шкале интервалов к отношению порядка добавляются операции сложения и вычитания
её элементов. Данные операции позволяют определить в качестве элементов шкалы
разницы между её элементами.
Количественное определение объекта в собственном смысле этого слова начинается только
со шкал интервалов. Однако и у этих шкал есть свои границы квантификации,
выражающиеся в наличии порогов измерения, в частности, нижнего порога,
своего рода кванта шкалы --- некоторого минимального интервала,
части которого уже не могут быть измерены средствами данной шкалы (\emph{цены деления}).

Шкала отношений представляет собой шкалу интервалов с дополнительно определенными
операциями умножения и деления, позволяющими преодолеть пороги шкалы интервалов.
В качестве нового элемента шкалы здесь можно выразить любую сколь угодно малую часть
или сколь угодно большое целое любого элемента шкалы.
Следует отметить, что элементы такой уже не могут быть реализованы физически,
поскольку шкалы отношений не имеют порогов количественных изменений.

Обычно процесс измерения развивается от номинальных шкал и шкал порядка по направлению
к созданию и использованию шкал интервалов и отношений.

В гуманитарных науках более распространены порядковые и интервальные шкалы,
а шкалы отношений больше используются в естественнонаучных дисциплинах.
С одной стороны, это можно объяснить меньшим теоретическим оснащением гуманитарного знания.
С другой стороны, возможно, что в случае субъектных онтологий гуманитарных наук мы имеем дело
с особым состоянием количества, которое более адекватно выражается порядковыми и интервальными шкалами.

{\color{red} Найти еще материал!}
%
% Прямое и косвенное измерение. Эталон?

\pagebreak