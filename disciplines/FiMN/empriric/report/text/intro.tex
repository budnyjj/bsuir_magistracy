\section*{ВВЕДЕНИЕ}
\addcontentsline{toc}{section}{Введение}

Научные знания представляют собой сложную развивающуюся систему, в которой по мере
эволюции возникают все новые уровни организации.
В их составе обычно выделяют два основных уровня --- уровень эмпирического и теоретического знания.
В \emph{эмпирическом знании} преобладает чувственное познание, т.~е. это вид познания,
преимущественно опирающийся на данные органов чувств --- зрения, слуха, вкуса, обоняния, осязания.
В \emph{теоретическом познании} преобладают рациональные методы познания,
преимущественно опирающиеся на логику, интеллект и мышление.

Данная работа посвящена изучению методов эмпирического познания
и содержит основные определения, классификации, а также краткое описание
актуальных проблем рассматриваемой предметной области.

Следует отметить, что в философской литературе нет единства в по отношению к тому,
что какие методы исследования считать самостоятельными, а какие --- нет.
В различных источниках можно найти следующие примеры общих методов эмпирического исследования:
\begin{itemize}
\item наблюдение и эксперимент~\cite{stepin1999};
\item наблюдение, эксперимент, измерение~\cite{moiseev2004};
\item наблюдение, эксперимент, измерение, описание и сравнение~\cite{nekrasov2010, ushakov2008}.
\end{itemize}

Автор данной работы склонен считать измерение частным случаем наблюдения или эксперимента,
а сравнение и измерение --- более теоретическими методами, играющими вспомогательную роль.