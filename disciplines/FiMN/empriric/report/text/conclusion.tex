\section*{ЗАКЛЮЧЕНИЕ}
\addcontentsline{toc}{section}{Заключение}

В данной работе было выполнено описание структуры эмпирических методов исследования,
дана краткая характеристика как общих, так и частных методов,
приведены ссылки на актуальные проблемы, а также на более подробную литературу в данной области.

В завершение хочется отметить,
что на практике невозможно говорить о некотором <<чистом>> эмпирическом знании,
совершенно независимом от знания теоретического.
Всякий фрагмент человеческой жизни тесно взаимодействует со всеми другими ее частями,
все бытие представляет из себя сеть взаимных влияний.
Развитие идей сетевой модели рациональности, циклической причинности проявляет
себя в критике разного рода <<абсолютных оснований>>,
которые якобы только определяют все иное, но сами ничем не определяются.
Развитие темы теоретической нагруженности эмпирического познания --- одно из проявлений
методологии всеобщей взаимосвязи, которая особенно выходит на первый план в современной науке,
начиная со второй половины XX века.
