\section[Частные методы эмпирического исследования]{%
  ЧАСТНЫЕ МЕТОДЫ ЭМПИРИЧЕСКОГО \\ ИССЛЕДОВАНИЯ
}

В данном разделе кратко опиываются частные методы эмпирического исследования:
идиографический метод, диалог, понимание, интроспекция, эмпатия, тестирование,
проективные методы, биографический метод, социальный эксперимент,
ролевые и имитационные игры.
Следует отметить, что данные методы находят свое применение главным образом
в социально-гуманитарных науках~\cite{nekrasov2010}.

\emph{Идиографический метод} --- метод исторических наук о культуре, сущность
которого состоит в описании индивидуальных особенностей исторических фактов,
формируемых наукой на основе так называемого <<отнесения к личности>>.
Под этим понимается способ выделения среди индивидуальных событий и явлений
действительности <<существенных>>.
Представления о <<ценности>> помогают отличать культурные феномены от природных.
Если естествознание устанавливает законы и обобщает, то история --- индивидуализирует,
выделяя из бесконечного многообразия явлений «исторический индивидуум».
Тем самым исторический процесс превращается в совокупность отдельных уникальных событий,
теряя связность и закономерность.
Основанное на метафизическом противопоставлении индивидуального и общего,
учение об идиографическом методе в конечном счёте сводит исторический процесс
к взаимодействию отдельных личностей~\cite{ilychev1983}.

\emph{Опросно-ответный метод} --- метод обратно-информационного взаимодействия
педагога и учащихся, которые в разнообразных формах учебной, трудовой,
творческой деятельности специально воспроизводят свои знания,
умения и навыки, качество которых анализируется и оценивается педагогом.
На основе полученной информации педагог осуществляет коррекцию процесса обучения,
организует систематический контроль, обучает приемам и логике устного
и письменного изложения знаний~\cite{vishnyakova1999}.

\emph{Понимание} определяется как универсальная операция мышления,
связанная с усвоением нового содержания, включением его в систему устоявшихся идей
и представлений. Понимание наделяет смыслом объекты социально-культурной и природной
реальности и вводит их тем самым в привычный и связный мир человека.
Оно всегда обусловлено социально-историческими и культурными предпосылками.
Уяснение смысла объекта как целого предполагает понимание его частей;
в свою очередь, уяснение смысла частей требует понимания смысла целого.~\cite{ivin1997}.
Исследованием данного понятиязанимается раздел современной философии
под названием \emph{герменевтика}.

\emph{Эмпатия} ---  способность человека отождествлять (идентифицировать)
один из своих Я-образов с воображаемым образом <<иного>>: с образами других людей,
живых существ, неодушевленных предметов и даже с линейными и пространственными формами.
Это сугубо индивидуальная способность людей, по-видимому, является одним из важнейших
условий творческого процесса --- в науке, технике, искусстве и т.~д.
Психофизиологические механизмы эмпатии, протекающие главным образом на бессознательном
и подсознательном уровнях, предполагают самовнушение (или внушение со стороны, гипноз),
которое позволяет преодолеть сопротивление сознательного Я инсталляции воображаемого Я-образа.
Благодаря самовнушению (гипнозу) новый Я-образ становится частью личности творца,
элементом его самосознания, он создает окрашенный позитивными эмоциями фон,
позволяющий наслаждаться творческим процессом~\cite{ivin2004}.

\emph{Тест} --- способ изучения глубинных процессов деятельности системы,
посредством помещения системы в разные ситуации и отслеживание доступных
наблюдению изменений в ней. Различают тесты для изучения интеллектуальных способностей,
уровня умственного развития личности и тесты успеваемости.
С их помощью можно выяснить уровень развития отдельных психических процессов,
уровни усвоения знаний, общего умственного развития личности.
Тесты как стандартизированные методы дают возможность сравнивать уровни развития
и успешности подопытных требованиям школьных программ и профессиограммы различных
специальностей~\cite{psyznayka_test}.

\emph{Проективные методики} --- группа методик, предназначенных для диагностики личности,
для которых характерен в большей мере целостный, глобальный подход к оценке личности,
а не выявление отдельных её черт.
Проективный метод характеризуется созданием экспериментальной ситуации,
допускающей множественность возможных интерпретаций при восприятии её испытуемыми.
Наиболее существенным признаком проективных методик является использование в них
неопределенных стимулов, которые испытуемый должен сам дополнять, интерпретировать,
развивать и т.~д. Так, испытуемым предлагается интерпретировать содержание сюжетных картинок,
завершать незаконченные предложения, давать толкование неопределенных очертаний и т.~п.
В этой группе методик ответы на задания также не могут быть правильными или неправильными;
возможен широкий диапазон разнообразных решений.
При этом предполагается, что характер ответов обследуемого определяется особенностями его личности,
которые <<проецируются>> в его ответах.
Цель проективных методик относительно замаскирована, что уменьшает возможность испытуемого
давать такие ответы, которые позволяют произвести желательное о себе впечатление.
За каждой такой интерпретацией вырисовывается уникальная система личностных смыслов
и особенностей когнитивного стиля субъекта~\cite{project_psychology}.

\emph{Биографический метод} --- метод исторического подхода к творцам произведений культуры,
науки и философии, при котором объектом исследования становятся жизненная и идейная траектория
авторов, воплощающаяся в личных документах (автобиографиях, биографиях, переписке и др.).
В начале XX века данный метож стал использоваться социальными науками,
прежде всего социологией и социальной психологией.
Его задача --- понимание субъективной стороны общественной жизни,
выявление типов личности и характерных для них интерпретаций социальных процессов и явлений.
Биографический метод основан на исследовании личных документов
(писем, автобиографий, дневников, мемуаров),
в которых находит свое выражение личное отношение к пережитой или
переживаемой общественной ситуации~\cite{kasavin2009}.

\emph{Социальный эксперимент} --- способ получения информации о наличии причинно-следственных
связей между показателями функционирования, деятельности, поведения социального объекта и
воздействующими на него некоторыми управляемыми и контролируемыми факторами.
Использование экспериментального метода требует наличия четко сформулированной гипотезы
о причинно-следственных связях, в соответствии с которой данное теоретическое суждение
может быть представлено в виде эмпирически проверяемых утверждений;
наличия объекта, допускающего
возможность описания системы переменных, детерминирующих его поведение,
возможность количественных и качественных измерений воздействующих на него
управляемых факторов и изменения его деятельности и поведения,
а также контроль воздействующих факторов, состояния объекта и условий во время проведения эксперимента.
Общая логика социального эксперимента заключается в том, чтобы, выбрав некоторую
экспериментальную группу (или группы), воздействовать на нее определенными факторами
и проследить направление, величину и устойчивость изменения характеристик,
интересующих исследователя~\cite{volovich1990}.

\emph{Игровые методы} --- активные методы, применяемые при обучении и выработке
управленческих решений. К ним относятся \emph{имитационные игры}, а также игры открытого типа.
Среди игровых методов выделяют \emph{психодраму} и \emph{социодраму}, где участники проигрывают
соответственно индивидуальные и групповые ситуации~\cite{komarova2014}.