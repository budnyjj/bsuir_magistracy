\section[Структура эмпирического исследования]{%
  СТРУКТУРА ЭМПИРИЧЕСКОГО \\ ИССЛЕДОВАНИЯ
}

Обычно в чувственном познании, составляющем основу эмпирических методов научного познания,
выделяют три основные формы --- ощущения, восприятия и представления.

\emph{Ощущения} --- это наиболее элементарные чувственные данные,
своего рода <<сенсорные атомы>> чувственного познания.
Как правило, они просты по сенсорной модальности, т.~е. представляют из себя чистый звук,
цвет, вкус и т.~д., и, кроме того, мгновенны во времени.

\emph{Восприятие} --- это более интегральная форма чувственного познания,
представляющая из себя комплексы ощущений, организованные в пространстве и времени.

\emph{Представление} является самым высоким уровнем организации чувственного восприятия,
объединяя в себе множество восприятий в пространстве и времени.
В отличие от ощущений и восприятий, представления в гораздо большей степени может вызываться
и убираться человеком по собственной воле.

Общими характеристиками чувственного познания являются его конкретность и конечность.
\emph{Конкретность} --- это сильная сторона чувственного познания, выражающаяся в том,
что оно сообщает нам уникальную информацию о нашем материальном мире в отдельном месте и времени.
\emph{Конечность} --- слабая сторона чувственного познания, связанная с тем,
что в чувственном познании мы можем получить информацию только о конечном ---
конечном числе объектов, событий, конечной части пространства и времени.
В то же время в научном познании очень важна информация о бесконечном,
и эту информацию чувственное познание дать не в состоянии.
Ее можно получить только на основе рационального познания~\cite{moiseev2004}.
Таким образом получается, что эмпирический и теоретический виды познания
не существуют в изоляции, а связаны между собой.

Эмпирический уровень исследований состоит из минимум двух подуровней:
непосредственных наблюдений и экспериментов, результатом которых являются
данные наблюдения, и познавательных процедур, посредством которых осуществляется
переход от данных наблюдения к эмпирическим зависимостям и фактам~\cite{stepin1999}.

Кроме этого, различают \emph{общие} методы эмпирического исследования,
используемые во всех отраслях науки, и \emph{частные},
используемые лишь в некотором её ограниченном подмножестве.