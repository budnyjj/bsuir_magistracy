\section*{ЗАКЛЮЧЕНИЕ}
\addcontentsline{toc}{section}{Заключение}

В данной работе были описаны основные положения философии экзистенциализма:
ключевая роль человека как творца самого себя;
принципиальная невозможность существования априорных моральных ориентиров;
противопоставление <<человеческой реальности>>
социальному детерминизму, выражаемому в виде таких понятий,
как <<человеческая натура>>, <<рок>>, <<судьба>>.

Хочется отметить, что, несмотря на свободную форму изложения,
экзистенциализм является весьма строгим учением.
Лишая нас единой точки моральной опоры,
он перекладывает на нас всю полноту ответственности за каждый осуществляемый
выбор и его последствия.
