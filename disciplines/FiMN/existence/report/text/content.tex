\section[Основные положения философии экзистенциализма]{%
    ОСНОВНЫЕ ПОЛОЖЕНИЯ ФИЛОСОФИИ \\
    ЭКЗИСТЕНЦИАЛИЗМА
}

\subsection{Человек --- это будущее человека}

В философии экзистенциализма человек занимает исключительную роль,
поскольку, в отличие ото всех остальных объектов,
его существование предшествует его сущности.
Это означает, что человек сначала существует, встречается,
появляется в мире, и только затем он определяется~\cite{sartr_exist_human}.
Таким образом, отрицаются такие понятия, как <<человеческая сущность>>,
<<человеческая натура>>, данные нам от природы,
ограничивающие наши возможности и детерминирующие наше поведение.
Напротив, человек рассматривается как творец самого себя
в соответствии с набором идей, воспринятых им в процессе
развития --- \emph{проекта бытия}.
Человек есть лишь то, что он сам из себя делает.
Человек --- это будущее человека.

\subsection{Заброшенность}

Постулируется, что человек \emph{всегда}
обладает свободой выбора. С другой стороны, можно сказать, что человеку всегда
\emph{приходится} выбирать, то есть он не имеет возможности не выбирать.
Возникает закономерный вопрос: как осуществлять этот выбор?
Существуют ли какие-либо моральные ориентиры,
на которые мы можем положиться?
В соответствии с учением экзистенциализма, таких ориентиров не существует.
Это означает, что в умопостигаемом мире не существует никаких априорных ценностей,
более того, нет никакой возможности их обнаружить.

Основным отличием атеистического направления экзистенциализма от религиозного является
не отрицание понятия бога как такового, а признание того факта,
что его существование ничего не меняет.
<<Не может быть блага a priori, так как нет бесконечного и совершенного разума,
который бы его мыслил. И нигде не записано, что благо существует, что нужно быть честным,
что нельзя лгать; и это именно потому, что мы находимся на равнине, и на этой равнине
живут одни только люди>>~\cite{sartr_exist_human}.

Следует отметить, что философы-экзистенциалисты обеспокоены
таким положением дел, называемым \emph{заброшенностью}.

\subsection{Тревога}

Заброшенность предполагает, что не существует ни какого-либо свода естественных
моральных законов, ни органа, который бы следил за его соблюдением.
Возникает вопрос: как при этом избежать полной анархии?

Замечено, что человек в своем выборе всегда старается выбрать
самый лучший для него вариант.
Выбирая, он примеряет его на себя, переживая эмоции, неотличимые от тех,
которые испытывались бы, если бы данный вариант был уже выбран.
Кроме этого, он примеряет его не только на себя, но и на окружающих,
спрашивая себя: <<Что было бы, если бы все так же поступали>>?
От этой мысли можно уйти, лишь проявив некоторую нечестность.
Тот, кто лжет, оправдываясь тем, что все так поступают, ---
не в ладах с совестью, так как сам факт лжи означает, что ей придается
значение универсальной ценности.

Философами-экзистенциалисты утверждают, что, совершая выбор,
человек испытывает \emph{тревогу} не только за себя, но и за человечество в целом.
<<Человек, который на что-то решается и сознает, что выбирает не только
свое собственное бытие, но что он еще и законодатель, выбирающий одновременно с собой
все человечество, не может избежать чувства полной и глубокой
ответственности>>.
Кроме этого, эта тревога объясняется прямой ответственностью за других людей.
Это не барьер, отделяющий нас от действия, но часть самого
действия~\cite{sartr_exist_human}.

Как следствие, таким понятиям, как <<человеческая натура>>,
<<судьба>> и <<рок>>, зачастую служащих удобными оправданиями нашему поведению,
противопоставляется понятие <<человеческой реальности>> как процесса конструктивного
выбора и реализаций выбранных индивидуумом решений.
При этом в глазах окружающих человек представляет собой серию реализаций
последовательных выборов, нашедших свое отражение в окружающей действительности,
и, кроме этой серии, ничего нет.

\subsection{Отчаяние}

Понятие \emph{отчаяния} напрямую связано с вопросом выбора.
Поскольку недвижимой точки опоры, на которую можно было бы опереться при выборе,
не существует, человек, осуществляющий свой выбор, не может ни на что надеятся.
Он не может надеятся ни на то, что его выбор непременно приведет к желаемому результату,
ни на помощь других людей, поскольку они находятся в аналогичной ситуации.
Все, что нам остается --- осуществлять свой выбор вслепую в соответствии лишь
со своей совестью и приводить его в жизнь, не надеясь на помощь со стороны.