\documentclass[a4paper,10pt,twoside]{article}
\usepackage{conf_template}
\setcounter{section}{0}
\setcounter{figure}{0}
\setcounter{table}{0}
\setcounter{equation}{0}
\setcounter{secnumdepth}{1}
\setcounter{secnumdepth}{1}
\begin{document}

\authors {
  Р.~И.~Будный
}

\topic{
  Численный анализ методов линейной аппроксимации статистических данных
}

\annotation{
  Выполнено сравнение точности оценивания параметров линейной модели
  по измерениям её входа и выхода с ошибками
  методами классической линейной регрессии и симметричной аппроксимации.
  Получены условия предпочтительного использования рассмотренных методов.
}

\begin{multicols}{2}

  \section*{
    Введение
  }
  В работе рассматривается задача линейной
  аппроксимации детерминированного объекта по измерениям с ошибками
  входных и выходных переменных, не получившая к настоящему времени
  достаточно полного разрешения.

  \section{
    Математическая модель
  }
  \content{
    Будем исходить из следующей математической модели рассматриваемой задачи:
    $$
      H = \alpha + \beta \Xi,
    $$
    $$
    X = \Xi + E, \eqno(1)
    $$
    $$
    Y = H + \Delta,
    $$
    где
    $ \Xi = (\xi_1, \xi_2, \ldots, \xi_n), H = (\eta_1, \eta_2, \ldots, \eta_n) $ --- \\
    векторы фактических значений входной и выходной переменных соответственно, \par
    \hspace{-3.95mm}
    $ X = (x_1, x_2, \ldots, x_n), Y = (y_1, y_2, \ldots, y_n) $ --- \\
    векторы измеренных значений входной и выходной переменных соответственно, \par
    \hspace{-3.8mm}
    $ E = (\varepsilon_1, \varepsilon_2, \ldots, \varepsilon_n),
    \Delta = (\delta_1, \delta_2, \ldots, \delta_n) $ --- \\
    векторы независимых ошибок измерений значений входной и
    выходной переменных соответственно, распределенных по нормальному закону:
    $ \varepsilon_i = N(0, \sigma_{\varepsilon}),
    \delta_i = N(0, \sigma_{\delta}), i = \overline{1, n} $, \par
    \hspace{-3.8mm} $ \alpha, \beta $ ---
    фактические значения параметров модели: постоянной составляющей и
    коэффициента усиления соответственно.

    Требуется по результатам измерений значений переменных $ X, Y $
    найти оценки $ \hat{\alpha}, \hat{\beta} $ параметров модели $ \alpha, \beta $.
  }

  \section{
    Оценивание точности параметров модели
  }
  \content{
    В работе было выполнено сравнение точности оценок
    классической линейной регресси [1] и симметричной аппроксимации [2, 3]
    для объекта со скалярными входом и выходом,
    в зависимости от значений параметров модели $ \alpha, \beta $,
    а также от значений с.~к.~о. ошибок измерений
    $ \sigma_{\varepsilon}, \sigma_{\delta} $.

    Значения $ \xi_i $ выбирались из равномерного в $ [0, 10] $ распределения.
    Для получения каждой оценки $ ( \alpha, \beta ) $ использовались результаты
    ста наблюдений $ ( x_i, y_i ), i = \overline{1, n}, n = 100 $.

    В качестве величины, характеризующей точность оценок,
    использовалось среднее Евклидово расстояние в пространстве параметров:
    $$
    d = \frac{1}{k} \sum_{j=1}^k \sqrt{(\hat{\alpha}_j - \alpha)^2 + (\hat{\beta}_j - \beta)^2}. \eqno(2)
    $$

    Расчеты расстояний (2) производились в узлах сетки значений
    $ \sigma_{\varepsilon}, \sigma_{\delta} $ в прямоугольнике
    \( [0, 2] \times [0, 2] \) с шагом 0{,}1.
    В каждом узле сетки вычислялось $ k = 100 $ оценок.
  }

  \section{
    Выводы
  }
  \content{
    Моделирование показало, что
    точность оценивания параметров зависит не только от
    с.к.о. ошибок измерений $ \sigma_{\varepsilon}, \sigma_{\delta} $,
    но и от величины коэффициента усиления $ \beta $.

    Для принятия решения о том, какой метод даёт более точные оценки параметров,
    предлагается использовать следующее эмпирическое правило:
    $$
    \sigma_{\delta} \le (0{,}7 + |\beta|) \sigma_{\epsilon} \eqno(3).
    $$

    Если условие (3) выполняется, то при данных значениях
    $ \beta, \sigma_{\varepsilon}, \sigma_{\delta} $
    симметричная аппроксимация дает более точные оценки параметров,
    чем классическая линейная регрессия.
    В противном случае использование классической регрессии
    является более предпочтительным.
  }

  \references{
  }
  \ListReferences{
  \item Муха, В.~С. Статистические методы обработки данных: учеб. пособие. / Муха~В.~С. //
    Минск: издат. центр БГУ.~-- 2009.~-- 183 с.
  \item Pearson, K. On lines and planes of closest fit to systems of points in space /
    K. Pearson / Philosophical Magazine. --- 1901. --- V. VI. --- №2. --- P. 559 --- 572.
  \item Муха В. С.
    Симметричная аппроксимация векторных статистических данных линейными многообразиями //
    Весцi Нац. акад. навук Беларусi. Сер. фiз.-мат. навук. --- 2016. --- №4. --- С. 23--31.
  }

\end{multicols}

\authorFIO{Будный Роман Игоревич}
\authorAbout{
  магистрант кафедры информационных технологий автоматизированных систем БГУИР,
  budnyjj@gmail.com.
}

\authorFIO{Научный руководитель: Муха Владимир Степанович}
\authorAbout{
  доктор технических наук, профессор, mukha@bsuir.by.
}

\end{document}