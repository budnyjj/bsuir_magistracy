\documentclass[hyperref={pdftex,unicode}]{beamer}

\usepackage[T2A]{fontenc}
\usepackage[utf8]{inputenc}
\usepackage[russian]{babel}
\usepackage{cmap}

\usepackage{xcolor}

\usepackage{helvet}
\usepackage{pscyr}

\usepackage{multicol}

\usepackage{amssymb,amsfonts,amsmath,mathtext}
\usepackage{cite,enumerate,float}

\usepackage{listings}
\usepackage{tikz}

\graphicspath{{images/}}
../../../templates/presentation/sys/styles.tex

\title{Численный анализ методов линейной аппроксимации статистических данных}
\author{%
    \small{Будный Р. И.} \\
    \smallskip
    \scriptsize{Научный руководитель: Муха В. С.}
}

\institute{
    Белорусский государственный университет информатики и радиоэлектроники
}
\date{2017}


\begin{document}

\begin{frame}
  \maketitle
\end{frame}

\setbeamertemplate{headline}{%
  \vspace{2mm}
  \hfill \insertframenumber{}\hspace*{2mm}}
\addtocounter{framenumber}{-1}

\begin{frame}{Постановка задачи идентификации}
  \begin{align*}
    \overline{\eta} &= \psi (\overline{\theta}, \overline{\xi}), \\
    \overline{x} &= \overline{\xi} + \overline{\varepsilon_x}, \\
    \overline{y} &= \overline{\eta} + \overline{\varepsilon_y},
  \end{align*}
  где \( \overline{\xi}, \overline{\eta} \)
  "--- векторы фактических значений входа и выхода, \\
  \hspace{10.5mm}\( \overline{\theta} \)
  "--- вектор фактических значений параметров, \\
  \hspace{10mm}\( \psi \)
  "--- векторная функция регрессии, \\
  \hspace{7mm}\( \overline{x}, \overline{y} \)
  "--- векторы набл. значений вх. и вых. переменной, \\
  \hspace{4mm}\( \overline{\varepsilon_x}, \overline{\varepsilon_y} \)
  "--- векторы независимых ошибок набл. вх. и вых.

  \bigskip
  Требуется, считая вид функции \( \psi \) известным и \\
  располагая измеренными значениями входа и выхода \( \overline{x}, \overline{y} \), \\
  найти оценку \( \hat{\overline{\theta}} \) параметров системы \( \overline{\theta} \).
\end{frame}

\begin{frame}{Классификация стохастических систем}
  \emph{Полустохастический объект} "--- \\
  объект с детерминированным входом и случайным выходом.

  \bigskip
  \emph{Стохастический объект первого типа} "--- \\
  объект со случайным входом и выходом. \\

  \bigskip
  \emph{Стохастический объект второго типа} "--- \\
  детерминированный объект с ошибками в измерениях входных и выходных переменных.
\end{frame}

\begin{frame}{Критерии идентификации стохастических систем}
  \begin{minipage}{0.45\linewidth}
    \centering
    % Graphic for TeX using PGF
% Title: /home/budnyjj/univer/magistracy/conferences/53/presentation/images/classic.dia
% Creator: Dia v0.97.3
% CreationDate: Tue May  2 14:01:26 2017
% For: budnyjj
% \usepackage{tikz}
% The following commands are not supported in PSTricks at present
% We define them conditionally, so when they are implemented,
% this pgf file will use them.
\ifx\du\undefined
  \newlength{\du}
\fi
\setlength{\du}{15\unitlength}
\begin{tikzpicture}
\pgftransformxscale{0.700000}
\pgftransformyscale{-0.700000}
\definecolor{dialinecolor}{rgb}{0.000000, 0.000000, 0.000000}
\pgfsetstrokecolor{dialinecolor}
\definecolor{dialinecolor}{rgb}{1.000000, 1.000000, 1.000000}
\pgfsetfillcolor{dialinecolor}
\pgfsetlinewidth{0.100000\du}
\pgfsetdash{}{0pt}
\pgfsetdash{}{0pt}
\pgfsetbuttcap
{
\definecolor{dialinecolor}{rgb}{0.000000, 0.000000, 0.000000}
\pgfsetfillcolor{dialinecolor}
% was here!!!
\pgfsetarrowsend{stealth}
\definecolor{dialinecolor}{rgb}{0.000000, 0.000000, 0.000000}
\pgfsetstrokecolor{dialinecolor}
\draw (15.950000\du,22.050000\du)--(16.000000\du,13.000000\du);
}
\pgfsetlinewidth{0.100000\du}
\pgfsetdash{}{0pt}
\pgfsetdash{}{0pt}
\pgfsetbuttcap
{
\definecolor{dialinecolor}{rgb}{0.000000, 0.000000, 0.000000}
\pgfsetfillcolor{dialinecolor}
% was here!!!
\pgfsetarrowsend{stealth}
\definecolor{dialinecolor}{rgb}{0.000000, 0.000000, 0.000000}
\pgfsetstrokecolor{dialinecolor}
\draw (15.000000\du,21.000000\du)--(28.000000\du,21.000000\du);
}
\pgfsetlinewidth{0.100000\du}
\pgfsetdash{}{0pt}
\pgfsetdash{}{0pt}
\pgfsetbuttcap
{
\definecolor{dialinecolor}{rgb}{0.000000, 0.000000, 0.000000}
\pgfsetfillcolor{dialinecolor}
% was here!!!
\definecolor{dialinecolor}{rgb}{0.000000, 0.000000, 0.000000}
\pgfsetstrokecolor{dialinecolor}
\draw (18.000000\du,14.000000\du)--(26.000000\du,20.000000\du);
}
\pgfsetlinewidth{0.100000\du}
\pgfsetdash{{1.000000\du}{1.000000\du}}{0\du}
\pgfsetdash{{0.600000\du}{0.600000\du}}{0\du}
\pgfsetbuttcap
{
\definecolor{dialinecolor}{rgb}{0.000000, 0.000000, 0.000000}
\pgfsetfillcolor{dialinecolor}
% was here!!!
}
\definecolor{dialinecolor}{rgb}{0.000000, 0.000000, 0.000000}
\pgfsetstrokecolor{dialinecolor}
\draw (20.000000\du,16.200000\du)--(20.000000\du,19.000000\du);
\pgfsetbuttcap
\pgfsetmiterjoin
\pgfsetdash{}{0pt}
\pgfsetlinewidth{0.100000\du}
\definecolor{dialinecolor}{rgb}{0.000000, 0.000000, 0.000000}
\pgfsetfillcolor{dialinecolor}
\pgfpathellipse{\pgfpoint{20.000000\du}{15.450000\du}}{\pgfpoint{0.250000\du}{0\du}}{\pgfpoint{0\du}{0.250000\du}}
\pgfusepath{fill}
\definecolor{dialinecolor}{rgb}{0.000000, 0.000000, 0.000000}
\pgfsetfillcolor{dialinecolor}
\fill (20.250000\du,16.200000\du)--(20.000000\du,15.700000\du)--(19.750000\du,16.200000\du)--cycle;
\pgfsetbuttcap
\pgfsetmiterjoin
\pgfsetdash{}{0pt}
\pgfsetlinewidth{0.100000\du}
\definecolor{dialinecolor}{rgb}{0.000000, 0.000000, 0.000000}
\pgfsetfillcolor{dialinecolor}
\pgfpathellipse{\pgfpoint{20.000000\du}{19.750000\du}}{\pgfpoint{0.250000\du}{0\du}}{\pgfpoint{0\du}{0.250000\du}}
\pgfusepath{fill}
\definecolor{dialinecolor}{rgb}{0.000000, 0.000000, 0.000000}
\pgfsetfillcolor{dialinecolor}
\fill (19.750000\du,19.000000\du)--(20.000000\du,19.500000\du)--(20.250000\du,19.000000\du)--cycle;
% setfont left to latex
\definecolor{dialinecolor}{rgb}{0.000000, 0.000000, 0.000000}
\pgfsetstrokecolor{dialinecolor}
\node[anchor=west] at (18.000000\du,17.800000\du){$ \rho_{\text{к}_i} $};
% setfont left to latex
\definecolor{dialinecolor}{rgb}{0.000000, 0.000000, 0.000000}
\pgfsetstrokecolor{dialinecolor}
\node[anchor=west] at (20.300000\du,20.000000\du){$ (x_i, y_i) $};
% setfont left to latex
\definecolor{dialinecolor}{rgb}{0.000000, 0.000000, 0.000000}
\pgfsetstrokecolor{dialinecolor}
\node[anchor=west] at (18.500000\du,20.000000\du){};
% setfont left to latex
\definecolor{dialinecolor}{rgb}{0.000000, 0.000000, 0.000000}
\pgfsetstrokecolor{dialinecolor}
\node[anchor=west] at (20.300000\du,15.000000\du){$ (x_i, \hat{\alpha}_{\text{к}} + \hat{\beta}_{\text{к}} x_i ) $};
\end{tikzpicture}

    Классический: \\
    \( \sum_i \rho_{\text{к}_i} \rightarrow \min \)
  \end{minipage}
  \hfill
  \begin{minipage}{0.45\linewidth}
    \centering
    % Graphic for TeX using PGF
% Title: /home/budnyjj/univer/magistracy/conferences/53/presentation/images/symmetric.dia
% Creator: Dia v0.97.3
% CreationDate: Tue May  2 14:15:14 2017
% For: budnyjj
% \usepackage{tikz}
% The following commands are not supported in PSTricks at present
% We define them conditionally, so when they are implemented,
% this pgf file will use them.
\ifx\du\undefined
  \newlength{\du}
\fi
\setlength{\du}{15\unitlength}
\begin{tikzpicture}
\pgftransformxscale{0.700000}
\pgftransformyscale{-0.700000}
\definecolor{dialinecolor}{rgb}{0.000000, 0.000000, 0.000000}
\pgfsetstrokecolor{dialinecolor}
\definecolor{dialinecolor}{rgb}{1.000000, 1.000000, 1.000000}
\pgfsetfillcolor{dialinecolor}
\pgfsetlinewidth{0.100000\du}
\pgfsetdash{}{0pt}
\pgfsetdash{}{0pt}
\pgfsetbuttcap
{
\definecolor{dialinecolor}{rgb}{0.000000, 0.000000, 0.000000}
\pgfsetfillcolor{dialinecolor}
% was here!!!
\pgfsetarrowsend{stealth}
\definecolor{dialinecolor}{rgb}{0.000000, 0.000000, 0.000000}
\pgfsetstrokecolor{dialinecolor}
\draw (15.950000\du,22.050000\du)--(16.000000\du,13.000000\du);
}
\pgfsetlinewidth{0.100000\du}
\pgfsetdash{}{0pt}
\pgfsetdash{}{0pt}
\pgfsetbuttcap
{
\definecolor{dialinecolor}{rgb}{0.000000, 0.000000, 0.000000}
\pgfsetfillcolor{dialinecolor}
% was here!!!
\pgfsetarrowsend{stealth}
\definecolor{dialinecolor}{rgb}{0.000000, 0.000000, 0.000000}
\pgfsetstrokecolor{dialinecolor}
\draw (15.000000\du,21.000000\du)--(28.000000\du,21.000000\du);
}
\pgfsetlinewidth{0.100000\du}
\pgfsetdash{}{0pt}
\pgfsetdash{}{0pt}
\pgfsetbuttcap
{
\definecolor{dialinecolor}{rgb}{0.000000, 0.000000, 0.000000}
\pgfsetfillcolor{dialinecolor}
% was here!!!
\definecolor{dialinecolor}{rgb}{0.000000, 0.000000, 0.000000}
\pgfsetstrokecolor{dialinecolor}
\draw (17.006017\du,14.015552\du)--(26.000000\du,18.000000\du);
}
\pgfsetlinewidth{0.100000\du}
\pgfsetdash{{1.000000\du}{1.000000\du}}{0\du}
\pgfsetdash{{0.600000\du}{0.600000\du}}{0\du}
\pgfsetbuttcap
{
\definecolor{dialinecolor}{rgb}{0.000000, 0.000000, 0.000000}
\pgfsetfillcolor{dialinecolor}
% was here!!!
}
\definecolor{dialinecolor}{rgb}{0.000000, 0.000000, 0.000000}
\pgfsetstrokecolor{dialinecolor}
\draw (21.552786\du,16.894427\du)--(20.447214\du,19.105573\du);
\pgfsetbuttcap
\pgfsetmiterjoin
\pgfsetdash{}{0pt}
\pgfsetlinewidth{0.100000\du}
\definecolor{dialinecolor}{rgb}{0.000000, 0.000000, 0.000000}
\pgfsetfillcolor{dialinecolor}
\pgfpathellipse{\pgfpoint{21.888197\du}{16.223607\du}}{\pgfpoint{0.250000\du}{0\du}}{\pgfpoint{0\du}{0.250000\du}}
\pgfusepath{fill}
\pgfpathellipse{\pgfpoint{20.111803\du}{15.3923607\du}}{\pgfpoint{0.250000\du}{0\du}}{\pgfpoint{0\du}{0.250000\du}}
\pgfusepath{fill}
\definecolor{dialinecolor}{rgb}{0.000000, 0.000000, 0.000000}
\pgfsetfillcolor{dialinecolor}
\fill (21.776393\du,17.006231\du)--(21.776393\du,16.447214\du)--(21.329180\du,16.782624\du)--cycle;
\pgfsetbuttcap
\pgfsetmiterjoin
\pgfsetdash{}{0pt}
\pgfsetlinewidth{0.100000\du}
\definecolor{dialinecolor}{rgb}{0.000000, 0.000000, 0.000000}
\pgfsetfillcolor{dialinecolor}
\pgfpathellipse{\pgfpoint{20.111803\du}{19.776393\du}}{\pgfpoint{0.250000\du}{0\du}}{\pgfpoint{0\du}{0.250000\du}}
\pgfusepath{fill}
\definecolor{dialinecolor}{rgb}{0.000000, 0.000000, 0.000000}
\pgfsetfillcolor{dialinecolor}
\fill (20.223607\du,18.993769\du)--(20.223607\du,19.552786\du)--(20.670820\du,19.217376\du)--cycle;
% setfont left to latex
\definecolor{dialinecolor}{rgb}{0.000000, 0.000000, 0.000000}
\pgfsetstrokecolor{dialinecolor}
\node[anchor=west] at (18.500000\du,18.000000\du){$ \rho_{\text{с}_i} $};
% setfont left to latex
\definecolor{dialinecolor}{rgb}{0.000000, 0.000000, 0.000000}
\pgfsetstrokecolor{dialinecolor}
\node[anchor=west] at (20.300000\du,20.000000\du){$ (x_i, y_i) $};
% setfont left to latex
\definecolor{dialinecolor}{rgb}{0.000000, 0.000000, 0.000000}
\pgfsetstrokecolor{dialinecolor}
\node[anchor=west] at (18.500000\du,20.000000\du){};
% setfont left to latex
\definecolor{dialinecolor}{rgb}{0.000000, 0.000000, 0.000000}
\pgfsetstrokecolor{dialinecolor}
\node[anchor=west] at (20.300000\du,14.700000\du){$ (x_i, \psi(\hat{\theta_{\text{с}}}, x_i)) $};
\end{tikzpicture}

    % $ \varepsilon = N(0, \sigma_{\varepsilon}), \delta = N(0, \sigma_{\delta}) $
    % --- ошибки измерений, \\

    Симметричный: \\
    \( \sum_i \rho_{\text{с}_i} \rightarrow \min \)
  \end{minipage}
\end{frame}

\begin{frame}{Задачи исследования}
  \begin{itemize}
  \item Выполнить численный анализ точности
    методов идентификации стохастических систем второго типа.
  \item Описать условия предпочтительного использования
    сравниваемых методов для оценивания параметров систем.
  \end{itemize}
\end{frame}

\begin{frame}{%
    Идентификация линейных стохастических систем \\
    \small{Методы идентификации}
  }
  \begin{itemize}
  \item Классический метод наименьших квадратов (МНК)~[1]. \\
    Основан на классическом критерии идентификации.
  \item Метод симметричной аппроксимации (МСА)~[2]. \\
    Основан на симметричном критерии идентификации.
  \end{itemize}
\end{frame}

\begin{frame}{%
    Идентификация линейных стохастических систем \\
    \small{Математическая модель идентифицируемой системы}
  }

  \begin{align*}
  h &= \alpha + \beta \xi, \\
  x &= \xi + \varepsilon_x, \\
  y &= h + \varepsilon_y,
  \end{align*}
  где \( \xi, h \)
  "--- фактические значения вх. и вых. переменной, \\
  \hspace*{4.5mm} \( \alpha, \beta \)
  "--- фактические значения параметров объекта, \\
  \hspace*{5.5mm} \( x, y \)
  "--- наблюдаемые значения вх. и вых. переменной, \\
  \hspace*{2.5mm} \( \varepsilon_x, \varepsilon_y \)
  "--- независимые ошибки наблюдений вх. и вых. переменной:
\(
\varepsilon_x = N(0, \sigma_{\varepsilon_x}),
\varepsilon_y = N(0, \sigma_{\varepsilon_y})
\).

\bigskip
\scriptsize{%
  Значения \( \xi_i \) выбирались из \( U(0, 10) \).
  Для получения каждой оценки \( \hat{\alpha}, \hat{\beta} \) использовались результаты
  ста наблюдений \( ( x_i, y_i ), i = \overline{1, n}, n = 100 \).
}
\end{frame}

\begin{frame}{%
    Идентификация линейных стохастических систем \\
    \small{Численный анализ точности методов идентификации}
  }
  Сравнительная точность оценивания параметров:
  \small{
    \begin{equation*}
      d =
      \frac{1}{k} \sum_{j=1}^k
      \Bigg(
      \sqrt{(\hat{\alpha}_{\text{МНК}_j} - \alpha)^2 + (\hat{\beta}_{\text{МНК}_j} - \beta)^2} -
      \sqrt{(\hat{\alpha}_{\text{МСА}_j} - \alpha)^2 + (\hat{\beta}_{\text{МСА}_j} - \beta)^2}
      \Bigg).
    \end{equation*}
  }

  Сравнительная точность прогнозирования наблюдений выхода:
  \small{
    \begin{equation*}
      p =
      \frac{1}{k} \sum_{j=1}^k
      \Bigg(
      \sqrt{ \sum_{i=1}^n (\hat{\alpha}_{\text{МНК}_j} + \hat{\beta}_{\text{МНК}_j} x_{ij} - y_{ij})^2} -
      \sqrt{ \sum_{i=1}^n (\hat{\alpha}_{\text{МСА}_j} + \hat{\beta}_{\text{МСА}_j} x_{ij} - y_{ij})^2}
      \Bigg).
    \end{equation*}
  }

  Исследовались зависимости
  \( d(\alpha, \beta, \sigma_{\varepsilon_x}, \sigma_{\varepsilon_y}) \) и
  \( p(\alpha, \beta, \sigma_{\varepsilon_x}, \sigma_{\varepsilon_y}) \).

\bigskip
\scriptsize{%
  Расчеты величин \( d, p \) производились в узлах сетки значений
  \( \sigma_{\varepsilon_x}, \sigma_{\varepsilon_y} \) в прямоугольнике
  \( [0, 2] \times [0, 2] \) с шагом 0{,}1. \\
  В каждом узле сетки вычислялось сто оценок (\( k = 100 \)).
}
\end{frame}

\begin{frame}{%
    Идентификация линейных стохастических систем \\
    \small{Результаты анализа точности методов идентификации}
  }
  \begin{enumerate}
  \item Точность оценивания параметров зависит от \( |\beta| \)
    и соотношения между \( \sigma_{\varepsilon_x} \) и \( \sigma_{\varepsilon_y} \). \\
  \item Если эмпирическое условие
    \begin{equation}
      \sigma_{\varepsilon_y} > (0{,}7 + |\beta|) \sigma_{\varepsilon_x}
      \label{eq:linear_condition}
    \end{equation}
    выполняется, то МНК оценивает параметры системы более точно, чем МСА.
    В противном случае МСА даёт более точные оценки, чем МНК.
    \item При больших значениях \( |\beta| \) и \( \sigma_{\varepsilon_x} \)
      МСА оценивает параметры системы значительно точнее, чем МНК.
  \item МНК прогнозирует наблюдения выхода по наблюдениям входа более точно, чем МСА.
  \end{enumerate}
\end{frame}

\begin{frame}{%
    Идентификация нелинейных стохастических систем \\
    \small{Методы идентификации}
  }
  \begin{itemize}
  \item Нелинейный метод наименьших квадратов (НМНК)~[3]. \\
    Итерационный. Требует расчета значения опорной точки.

  \item Метод рядов Тейлора (МРТ)~[4]. \\
    Использует информацию о точности наблюдений.
  \end{itemize}
\end{frame}

\begin{frame}{%
    Идентификация нелинейных стохастических систем \\
    \small{Математическая модель идентифицируемой системы}
  }

  \begin{align*}
    h &= \psi(\overline{\theta}, \xi), \\
    x &= \xi + \varepsilon_x, \\
    y &= h + \varepsilon_y,
  \end{align*}
  где \( \xi, h \)
  "--- фактические значения вх. и вых. переменной, \\
  \hspace*{9.5mm}\( \psi \) "--- скалярно-векторная функция регрессии, \\
  \hspace*{10.3mm}\( \overline{\theta} \)
  "--- вектор факт. значений пар-ов объекта, \\
  \hspace*{6.5mm}\( x, y \) "--- наблюдаемые значения вх. и вых. переменной, \\
  \hspace*{3.7mm}\( \varepsilon_x, \varepsilon_y \)
  "--- независимые ошибки измерений значений вх. и вых. переменной:
\(
\varepsilon_x = N(0, \sigma_{\varepsilon_x}),
\varepsilon_y = N(0, \sigma_{\varepsilon_y})
\).

\bigskip
\scriptsize{%
  Значения \( \xi_i \) выбирались из \( U(0, 10) \).
  Для получения каждой оценки \( \hat{\overline{\theta}} \) использовались результаты
  ста наблюдений \( ( x_i, y_i ), i = \overline{1, n}, n = 100 \).
}
\end{frame}

\begin{frame}{%
    Идентификация линейных стохастических систем \\
    \small{Численный анализ точности методов идентификации}
  }
  Сравнительная точность оценивания параметров:
  \small{
    \begin{equation*}
      \begin{split}
        d(\sigma_{\varepsilon_x}, \sigma_{\varepsilon_y}) &=
        \dfrac{1}{k} \sum_{j=1}^k
        \sqrt{\sum_{\text{i=1}}^m (\hat{\theta}_{\text{МНК}_{ij}}(\sigma_{\varepsilon_x}, \sigma_{\varepsilon_y}) - \theta_{ij})^2} - \\
        &- \dfrac{1}{k} \sum_{j=1}^k
        \sqrt{\sum_{\text{i=1}}^m (\hat{\theta}_{\text{МРТ}_{ij}}(\sigma_{\varepsilon_x}, \sigma_{\varepsilon_y}) - \theta_{ij})^2},
      \end{split}
    \end{equation*}
  }
  где \( k \) "--- число оценок, \\
  \hspace*{6.5mm}\( m \) "--- число параметров модели.
\end{frame}

\begin{frame}{%
    Идентификация линейных стохастических систем \\
    \small{Численный анализ точности методов идентификации}
  }
  Исследовалась зависимость
  \( d(\overline{\theta}, \sigma_{\varepsilon_x}, \sigma_{\varepsilon_y}) \) для следующих функций регрессии:
  \begin{itemize}
  \item линейной:
    \( \psi = \theta_0 + \theta_1 \xi \)
  \item параболической:
    \( \psi = \theta_0 + \theta_1 \xi + \theta_2 \xi^2 \)
  \item синусоидальной:
    \( \psi_{\varphi} = \theta_0 + \theta_1 \sin{(\varphi \xi)}, \:
    \varphi \in \{ 0{,}2, 1, 5 \} \)
  \item экспоненциальной:
    \( \psi = e^{\theta_0 + \theta_1 \xi}, \; \theta_0 > 0, \theta_1 > 0 \)
  \item обратной:
    \( \psi = \theta_0 + \dfrac{1}{\theta_1 + \theta_2 x} \)
  \end{itemize}

\bigskip
\scriptsize{%
  Расчеты величин \( d, p \) производились в узлах сетки значений
  \( \sigma_{\varepsilon_x}, \sigma_{\varepsilon_y} \) в прямоугольнике
  \( [0, 2] \times [0, 2] \) с шагом 0{,}1. \\
  В каждом узле сетки вычислялось сто оценок (\( k = 100 \)).
}
\end{frame}

\begin{frame}{%
    Идентификация нелинейных стохастических систем \\
    \small{Результаты анализа точности методов идентификации}
  }
  \begin{enumerate}
  \item Точность оценивания параметров зависит от
    \( \overline{\theta}, \sigma_{\varepsilon_x}, \sigma_{\varepsilon_y} \).
  \item Точность НМНК определяется значением опорной точки. \\
    Выбор алгоритма расчета опорной точки зависит от \( \psi \).
  \item МРТ позволяет получать оценки параметров с точностью, \\
    сравнимой с точностью НМНК(1)-оценивания, \\
    без указания опорной точки.
  \item Для \( \psi \), нелинейных по параметрам,
    точность МРТ-оценок значительно превосходит точность НМНК-оценивания.
  \end{enumerate}

  \bigskip
  \scriptsize{%
    НМНК(\( i \))-оценка
    "--- оценка параметров, полученная на \( i \)-той итераций НМНК
    с использованием <<удачной>> опорной точки.
  }
\end{frame}

\begin{frame}{Заключение}
  \begin{enumerate}
  \item Если при идентификации линейной стохастической системы
    второго типа выполняется условие~\eqref{eq:linear_condition}, то
    МСА-оценки являются более точными, \\
    чем оценки, полученные с помощью МНК. \\
    Для идентификации и прогнозирования наблюдений выхода
    линейных стохастических систем в остальных случаях предпочтительнее
    использовать МНК.
  \item Для идентификации нелинейных стохастических систем второго типа
    целесообразно применять НМНК с использованием МРТ для расчета опорной точки.
  \end{enumerate}
\end{frame}

\begin{frame}[fragile]{Литература}
  \small{
    [1] Линник Ю. В. Метод наименьших квадратов и основы
    математико-статистической теории обработки наблюдений /
    Под ред. А. А. Лукьянов.
    "--- 2 изд.
    "--- Л : Физматгиз, 1962.
    "--- 352 с.

    \bigskip
    [2] Муха В. С. Симметричная аппроксимация векторных
    статистических данных линейными многообразиями // Весцi
    Нацыянальнай акадэмii навук Беларусi.
    "--- 2016.
    "--- \textnumero 4.
    "--- С. 23--31.

    \bigskip
    [3] Муха В. С. Статистические методы обработки данных:
    учеб. пособие.
    "--- Минск : издат. центр БГУ, 2009.
    "--- 108 с.

    \bigskip
    [4] Муха В. С. Оценивание векторных параметров по результатам
    косвенных измерений // Электромагнитные волны и
    электронные системы.
    "--- 2000.
    "--- Т. 5, \textnumero 4.
    "--- С. 44--50.
  }
\end{frame}

\begin{frame}
  \centering
  Спасибо за внимание!
\end{frame}

\end{document}