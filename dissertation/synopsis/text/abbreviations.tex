\chapter*{Перечень условных обозначений и терминов}
\addcontentsline{toc}{chapter}{Перечень условных обозначений и терминов}

ИСС --- идентификация стохастических систем.

% Перечень условных обозначений и терминов может быть представлен в виде
% отдельного списка, помещаемого перед введением в том случае,
% если в диссертации принята  специфическая терминология,
% а также употребляются мало распространенные сокращения,
% новые символы, обозначения и т.п.
% Перечень должен располагаться столбцом, в котором слева в алфавитном
% порядке приводят, например, сокращение, справа – его детальную расшифровку.

% Если в диссертации специальные термины, сокращения, символы, обозначения и т. п.
% повторяются менее трех раз, перечень не составляют,
% а их расшифровку приводят в тексте при первом упоминании.