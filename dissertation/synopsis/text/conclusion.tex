\chapter*{Заключение}
\addcontentsline{toc}{chapter}{Заключение}

В данной работе были рассмотрены некоторые методы идентификации стохастических систем,
на основании численного анализа точности были
определены области их предпочтительного использования.

Метод симметричной аппроксимации, использующий симметричный критерий наименьших квадратов,
оценивает параметры линейных детерминированных систем с большим коэффициентом усиления по
наблюдениям с ошибками более точно, чем классическая линейная регрессия,
которую рекомендуется использовать в остальных случаях.
Использование классического критерия наименьших квадратов позволяет наиболее точно
прогнозировать значения наблюдений выхода линейной стохастической системы по наблюдениям входа.


% Использование классической линейной регрессии остается предпочтительным
% для оценивания параметров линейных стохастических систем в остальных случаях.



% а также для предсказания наблюдений выхода


% Метод наименьших квадратов, использующий классический критерия аппроксимации,
% предпочтительнее использовать для оценивания параметров




% % показал, что .


% идентфификация линейных стохастических систем.
% классификация стохастических систем.
% описаны критерии идентификации.

% метод симметричной аппроксимации предпочтительнее использовать для оценивания параметров стохастических систем второго типа с большим коэффициентом усиления (ссылка на условие). В противном случае классическая линейная регрессия дает более точные оценки.

% классический критерий предпочтительно использовать для прогнозирования наблюдений выхода по наблюдениям входа всех типов линейных стохастических систем.