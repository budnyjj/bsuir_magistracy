\chapter*{ЗАКЛЮЧЕНИЕ}
\addcontentsline{toc}{chapter}{Заключение}

В данной работе был выполнен численный анализ точности методов идентификации
стохастических систем.
Были определены области предпочтительного использования
метода наименьших квадратов и метода симметричной аппроксимации применительно к
идентификации линейных стохастических систем.
Были описаны условия предпочтительного использования метода наименьших квадратов
и метода рядов Тейлора для оценивания параметров нелинейных стохастических систем
второго типа с различными функциями регрессии.

Метод симметричной аппроксимации оценивает параметры линейных детерминированных систем
с большим коэффициентом усиления по наблюдениям с ошибками во входных и выходных наблюдениях
более точно, чем метод наименьших квадратов.
Для оценивания параметров линейных стохастических систем во всех остальных случаях,
а также для прогнозирования наблюдений выхода линейной стохастической системы по
наблюдениям входа рекомендуется использовать метод наименьших квадратов.

Моделирование показало, что при идентификации нелинейных детерминированных систем по
наблюдениям с ошибками во входных и выходных наблюдениях точность метода наименьших
квадратов существенным образом зависит от того, насколько <<удачной>> является опорная точка.
Выбор подходящего алгоритма расчета опорной точки зависит от вида функции
регрессии идентифицируемой системы.
В данных условиях метод рядов Тейлора позволяет получать оценки параметров
приемлемой точности без необходимости выбора опорной точки.

В связи с этим представляется целесообразным применение метода рядов Тейлора
для расчета опорной точки, используемой для оценивания параметров методом наименьших квадратов.
Данный подход позволит получать точные оценки параметров идентифицируемой системы
независимо от вида её функции регрессии.
