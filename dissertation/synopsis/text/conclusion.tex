\chapter*{ЗАКЛЮЧЕНИЕ}
\addcontentsline{toc}{chapter}{Заключение}

В данной работе рассматривается задача идентификации стохастических систем.
Была предложена классификация стохастических объектов для целей
их идентификации и сравнения.
Выполнен численный анализ точности оценивания параметров и прогнозирования
наблюдений выхода по наблюдениям входа линейных стохастических систем второго типа
классическим методом наименьших квадратов и методом симметричной аппроксимации.
Предложено эмпирическое правило для выбора метода оценивания параметров данных систем.

Моделирование показало, что метод симметричной аппроксимации оценивает параметры систем
с большим коэффициентом усиления более точно, чем метод наименьших квадратов.
Для оценивания параметров линейных стохастических систем во всех остальных случаях,
а также для прогнозирования наблюдений выхода систем по наблюдениям входа
рекомендуется использовать метод наименьших квадратов.

Для идентификации нелинейных стохастических систем было предложено
использовать нелинейный метод наименьших квадратов и метод рядов Тейлора.
Описаны условия предпочтительного использования метода этих методов
для оценивания параметров нелинейных стохастических систем
второго типа с различными функциями регрессии.
Разработаны программные реализации всех перечисленных методов.

По результатам моделирования можно сделать вывод,
что точность метода наименьших квадратов существенным образом
зависит от того, насколько <<удачной>> является опорная точка.
Выбор подходящего алгоритма расчета опорной точки зависит от вида функции
регрессии идентифицируемой системы.
В аналогичных условиях метод рядов Тейлора позволяет получать оценки параметров
приемлемой точности без необходимости указания опорной точки.
В связи с этим представляется целесообразным применение метода рядов Тейлора
для определения опорной точки, используемой для оценивания параметров методом
наименьших квадратов.
Данный подход позволит получать точные оценки параметров идентифицируемой системы
независимо от вида её функции регрессии.
