\chapter{Теоретический раздел}

В последующих главах с исчерпывающей полнотой  необходимо изложить собственное исследование с выявлением того нового и оригинального, что вносится в разработку проблемы. Все идеи и положения автора должны быть обстоятельно обоснованы на базе принятой методики, вытекающей из сущности предмета исследования.

Весь порядок изложения в диссертации должен быть подчинен  руководящей идее, четко сформулированной в теоретическом разделе диссертации. Логичность построения и целеустремленность изложения  глав достигается в случае, если каждая из глав имеет определенное целевое назначение и является базой для последующей.

Желательно, чтобы в каждой главе приводились краткие выводы. Это позволит четко сформулировать итоги каждого этапа выполнения работы и даст возможность освободить общие выводы от второстепенных подробностей.

\chapter{Экспериментальный раздел}
