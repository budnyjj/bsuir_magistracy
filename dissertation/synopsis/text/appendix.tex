\renewcommand{\thefigure}{\Asbuk{section}.\arabic{figure}}
\renewcommand{\thetable}{\Asbuk{section}.\arabic{table}}
\renewcommand{\thelstlisting}{\Asbuk{section}.\arabic{lstlisting}}

\chapter*{Приложение}
\addcontentsline{toc}{chapter}{Приложение}

\setcounter{section}{1}
\setcounter{figure}{0}
\setcounter{table}{0}
\setcounter{lstlisting}{0}

Приложения оформляются как продолжение диссертации на последующих ее страницах, располагая их в порядке появления ссылок в тексте. В приложения следует включать вспомогательный материал, необходимый для полноты восприятия диссертации: таблицы вспомогательных цифровых данных; протоколы и акты испытаний и внедрения; описание  алгоритмов и программ задач, решаемых на ЭВМ, разработанных в процессе выполнения работы; иллюстрации вспомогательного характера.