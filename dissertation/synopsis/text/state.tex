\chapter[Анализ состояния проблемы. Постановка задачи исследования]{%
  Анализ состояния проблемы. \hspace{2cm}
  Постановка задачи исследования
}

\section[История развития проблемы идентификации стохастических систем]{%
  \color{red} История развития проблемы идентификации \\
  стохастических систем
}

Начало идентификации систем, как предмета построения математических моделей на основе наблюдений,
связано с работами~\cite{gauss_1809, gauss_1810, gauss_1821} Гаусса,
в которых он разработал метод наименьших квадратов и использовал его для предсказания
траектории движения планет.
Впоследствии этот метод нашёл применение во множестве других приложений,
в том числе для построения математических моделей управляемых объектов,
используемых в автоматизации.
Существенный вклад в развитие метода наименьших квадратов внесли
Лаплас~\cite{laplace_1812}, Чебышев~\cite{chebyshev_1859}, Марков~\cite{markov_1898},
Эйткен~\cite{aitken_1935}, Колмогоров~\cite{kolmogorov_1946} и Рао~\cite{rao_1946}.
Системное изложение этого метода представлено, например, в работе Линника~\cite{linnik62}.

Большая часть ранних работ по идентификации систем была сделана специалистами в области статистики,
эконометрики (особенно их интересовали приложения идентификации, связанные с временными рядами) и
образовала область под названием статистическое оценивание.
{\color{red} Проблема заключалась в том, что область приложений этих методов была ограничена чаще всего
скалярными системами.}
В 1960 году Калман представил описание управляемой системы в
виде пространства состояний, что позволяло работать и с многомерными системами,
и заложил основы для оптимальной фильтрации и оптимального управления,
основывавшихся на данном типе описания.

Методы идентификации систем для задач управления были разработаны в 1965 году в работах
Хо и Калмана~\cite{ho_1965}, Острёма и Болина~\cite{astrom_1965}.
Работа Хо и Калмана посвящена поиску модели изучаемого объекта в пространстве состояний,
имеющей наименьший порядок вектора состояний, на основе информации об импульсной переходной характеристике.
Данная задача, но уже при наличии реализаций случайного процесса, где формируется марковская модель,
была решена в 70-х годах в работах Форре~\cite{faurre_1973} и Акайка~\cite{akaike_1974}.
Эти работы заложили создание метода подпространства в начале 90-х.

Работа Острёма и Болина представила для сообщества специалистов по идентификации метод
максимального правдоподобия, который был разработан специалистами по временным рядам для
оценивания параметров моделей в виде разностных уравнений~\cite{koopmans_1950}.
Эти модели, которые известны в статистической литературе как ARMA (авторегрессионное скользящее среднее) и
ARMAX (авторегрессионное скользящее среднее со входом), позднее,
образовали основу для создания метода ошибки предсказания.
В 1970-х годах этот метод стал доминировать в теории и, что более важно, в приложениях идентификации.
Большая часть исследовательской активности сфокусировалась на проблемах идентификации многомерных и
замкнутых систем.
Ключевой задачей для этих двух классов систем являлось найти условия для эксперимента и
способы параметризации проблемы, при которых найденная модель приблизится к единственно точному
описанию реальной системы.

К 1976 году была сделана первая попытка рассмотреть идентификацию систем как теорию аппроксимации,
в которой стоит задача наилучшей возможной аппроксимации реальной системы внутри данного класса
моделей~\cite{ljung_1976, anderson_1978, ljung_1979}.
Таким образом, преобладающая точка зрения среди специалистов по идентификации сменилась с
поиска описания для истинной системы на поиск описания наилучшей возможной аппроксимации.

Идея, что качество модели может быть изменено с помощью выбора переменных синтеза,
привела к всплеску активности в 90-х годах XX века, который продолжается до сих пор.
Главное применение новой парадигмы --- это идентификация для MBC (управление на основе модели).
Идентификация для задач управления расцвела с небывалой силой со времени своего появления и
применение к управлению методов идентификации вдохнуло вторую жизнь в такие уже
известные области исследования, как планирование эксперимента, идентификация в замкнутом контуре,
частотная идентификация, робастное управление при наличии неопределенности.

{\color{red}
  Как обстоят дела в СССР и СНГ?
  Нужно написать что-то о нашей работе.
}

\section[Идентификация стохастических систем как математическая проблема]{%
  Идентификация стохастических систем как \\
  математическая проблема}

\subsection{Постановка задачи идентификации}

Задача идентификации в общем виде формулируется следующим образом:
на основании априорной информации, а также по результатам наблюдений над
входными и выходными переменными системы должна быть построена оптимальная в
некотором смысле модель, т.~е. формализованное представление этой системы~\cite{eikhoff_1975}.

В зависимости от характера располагаемой априорной информации различают
задачи идентификации систем в узком и широком смысле.
Задача идентификации в узком смысле состоит в оценивании параметров и
состояния системы по результатам наблюдений над входными и выходными переменными,
полученными в условиях функционирования объекта.
При этом известна структура системы и задан класс, к которому эта модель относится.

В случае, когда априорная информация об объекте при идентификации в широком смысле отсутствует
или очень бедная, принято говорить о задаче идентификации в широком смысле.
При её решении приходится решать большое число дополнительных задач:
выбор структуры системы и задание класса моделей,
оценивание степени стационарности, линейности объекта и действующих переменных,
оценивание степени и формы влияния входных переменных на выходные,
выбор информативных переменных и др.

Стохастическая система --- система, связь входа и выхода которой имеет недетерминированный
(стохастический) характер. Это может быть обусловлено как недетерминированностью самой системы,
так и ошибками измерения её переменных.
Основными источниками таких ошибок являются:
ошибки, вызванные помехами;
ошибки усечения;
ошибки из-за неправильного определения состояния;
ошибки, связанные с упрощением при реализации;
ошибки выборочной аппроксимации~\cite{eikhoff_1975}.

{\color{red} Предметом данной работы является идентификация стохастических систем в узком смысле.
Данный выбор обусловлен следующими причинами.
Поскольку вход и выход реальных систем всегда измеряется с ограниченной точностью,
при идентификации удобно считать сами эти системы стохастическими.
Кроме этого, решение задачи идентификации в узком смысле необходимо для
решения задачи идентификации в широком смысле.
Наконец, задача идентификации в узком смысле является строго формализуемой.}

Математическая модель задачи идентификации имеет следующий вид:
\begin{equation}
  \label{eq:model_general}
  \begin{aligned}
    H &= \Psi (\Theta, \Xi), \\
    X &= \Xi + E_x, \\
    Y &= H + E_y,
  \end{aligned}
\end{equation}
где \( \Xi, H \) --- векторы фактических значений входной и выходной переменной, \par
\( \Theta \) --- вектор фактических значений параметров объекта, \par
\( \Psi \) --- векторная функция регрессии, \par
\( X, Y \) ---  вектор измеренных значений входной и выходной переменной, \par
\( E_x, E_y \) --- векторы независимых ошибки измерений значений входной и выходной переменной,
распределенные по нормальному закону:
{\color{red} \( E_x = N(0, \sigma_{E_x}), E_y = N(0, \sigma_{E_y}) \)}.

В этой задаче требуется,
считая вид функции \( \Psi \) известными и располагая измеренными значениями входа и выхода \( X, Y \),
найти оценку $ \hat{\Theta} $ вектора параметров системы $ \Theta $.

Следует отметить, что в ряде частных случаев данная модель имеет более простой вид.
Так, например, в скалярном случае, когда система имеет всего один вход и выход,
модель~\ref{eq:model_general} сводится к
\begin{equation}
  \label{eq:model_scalar}
  \begin{aligned}
    h &= \psi (\theta_1, \ldots, \theta_n, \xi), \\
    x &= \xi + e_x, \\
    y &= h + e_y,
  \end{aligned}
\end{equation}
где \( \xi, H \) --- фактические значения входной и выходной переменной, \par
\( \theta \) --- фактические значения параметров объекта, \par
\( \psi \) --- скалярная функция регрессии, \par
\( x, y \) --- измеренные значения входной и выходной переменной, \par
\( e_x, e_y \) --- независимые ошибки измерений значений входной и выходной переменной,
распределенные по нормальному закону: \( e_x = N(0, \sigma_{e_x}), e_y = N(0, \sigma_{e_y}) \).

Если же скалярная система~\ref{eq:model_scalar} является к тому же линейной,
то её модель можно записать в еще более простом виде:
\begin{equation}
  \label{eq:model_scalar_linear}
  \begin{aligned}
    h &= \psi (\alpha, \beta, \xi), \\
    x &= \xi + e_x, \\
    y &= h + e_y,
  \end{aligned}
\end{equation}
где \( \alpha, \beta \) --- фактические значения параметров объекта.

\subsection{Классификация стохастических систем}

{\color{red} Статистические данные могут порождаться системами нескольких видов.}
{\color{red} TODO: проверить обозначения}.

Полустохастическим объектом будем называть объект с детерминированным входом и случайным выходом.
Это либо регрессионный объект (объект с внутренним шумом на выходе),
либо детерминированный объект с ошибками в измерениях выходных переменных (рисунок~\ref{fig:type_half}).

\begin{figure}[h!]
  \centering
  \fcolorbox{gray}{white}{
    \includegraphics[width=92mm]{fig/half_new.png}
  }
  \caption{Схема полустохастического объекта}
  \label{fig:type_half}
\end{figure}

Объект описывается условной плотностью вероятности \( f(\eta / \xi, \theta) \),
\( \eta \) --- выходная переменная,
\( \xi \) --- входная переменная,
\( \theta \) --- параметр объекта,
\( y \) --- наблюдение выходной переменной,
\( f(y / \eta) \) --- условная плотность вероятности, описывающая измерительную систему,
ВУ --- вычислительное устройство,
\( \Delta(\xi, y) \) --- результат аппроксимации.
Входная и выходная переменная, а также параметр могут быть многомерными.

Стохастическим объектом первого типа будем называть объект со случайным входом и выходом
(рисунок~\ref{fig:type_first}).
Объект описывается совместной плотностью вероятности \( f(\xi, \eta) \).
Возможно наличие ошибок в измерениях входных и выходных переменных.

\begin{figure}[h!]
  \centering
  \fcolorbox{gray}{white}{
    \includegraphics[width=92mm]{fig/first_new.png}
  }
  \caption{Схема стохастического объекта \\ первого типа}
  \label{fig:type_first}
\end{figure}

Стохастическим объектом второго типа назовем детерминированный объект с ошибками
в измерениях входных и выходных переменных (рисунок~\ref{fig:type_second}).
Объект описывается детерминированной зависимостью \( \eta = \phi(\xi, \theta), \)
\( \theta \) --- вектор параметров объекта.

\begin{figure}[h!]
  \centering
  \fcolorbox{gray}{white}{
    \includegraphics[width=92mm]{fig/second_new.png}
  }
  \caption{Схема стохастического объекта \\ второго типа}
  \label{fig:type_second}
\end{figure}

\newpage
\subsection{Критерии идентификации стохастических систем}

Критерием выбора оптимума должен быть функционал от выходных
сигналов или от математического ожидания ошибок оценок параметров.

\section{Постановка задачи исследования}

% Теория идентификации систем, рассматривающая данную задачу,
% получила свое наибольшее развитие во второй половине XX века.
% Её выводы находят свое применение в различных отраслях науки и техники:
% энергетике, машиностроении, авиации, химической промышленности,
% физике, экономике, биологии и медицине при решении таких прикладных задач, как
% диагностика, управление, автоматический контроль,
% автоматизация принятия решений, распознавание образов.
% Данная теория опирается на следующие дисциплины:
% метрологию, теории управления, систем, сигналов, информации,
% стохастическую аппроксимацию и математическую статистику~\cite{eikhoff_1975}.


% Построение модели сводится к следующим этапам:
% \begin{itemize}
% \item выбор структуры модели из физических соображений;
% \item подгонка параметров к имеющимся данным (оценивание);
% \item проверка и подтверждение модели (диагностическая проверка);
% \item испольование модели по назначению.
% \end{itemize}

% Структура модели выбирается на основе априорной информации о системе и
% преследуемых целях.

% В большинстве реальных ситуаций налюдения над системой искажены
% случайными воздествиями (возмущениями, ошибками).

% Под оцениванием параметров понимается экспериментальное определение
% значений параметров, характеризующих динамику поведения объекта,
% в предположении, что структура модели объекта известна.

% Как оценить точность идентификации -- по отклонениям параметров
% модели или её отклика? Если основная цель состоит в
% проектировании системы управления, то представляется логичным
% оценивать точность идентификации по результатам функционирования
% созданной на основе идентификации объекта системы управления.

% теория систем -- линейные и нелинейные модели
% теоретическая дисциплина -- стохастическая аппроксимация ?
% цели применения.

% Выбор между линейными и нелинейными моделями.
% Скорость изменения параметров?
% Выбор типа модели тесно связан с решаемой задачей.

% Два вида реализаций:
% - использующая явные математические выражения;
% - реализации по настраиваемой модели.

% Методы идентификации:
% - вне контура регулирования;
% - внутри замкнутого контура регулирования.

% Дополнительная априорная информация приводит к улучшению оценик,
% однако затраты на реализацию могут оказаться чрезмерными.

% Метод наименьших квадратов может быть получен из метода максимума правдоподобия.

% Источники ошибок:
% - ошибки, вызванные помехами;
% - ошибки усечения;
% - ошибки из-за неправильного определения состояния;
% - ошибки, связанные с упрощением при реализации;
% - ошибки выборочной аппроксимации.

% Есть ли связь между НМНК и марковскими оценками?

% В обзоре литературы, охватывающем не менее 30 источников за последние 10–15 лет (включая зарубежные публикации и электронные ресурсы), необходимо показать основные этапы в развитии знания по проблеме диссертации, критически осветив известные работы, необходимо назвать неразрешенные вопросы и таким образом определить свое место в решении проблемы. Желательно закончить этот раздел кратким резюме о той конкретной задаче, которую автор стремиться поставить и решить в диссертации.



% Это вызвано тем, что она является неотъемлемой частью задачи идентификации
% в широком смысле, а также поддаётся полной формализации.

% Для идентификации стохастических систем можно использовать математические методы
% аппроксимации статистических данных.



% https://ru.wikipedia.org/wiki/%D0%98%D0%B4%D0%B5%D0%BD%D1%82%D0%B8%D1%84%D0%B8%D0%BA%D0%B0%D1%86%D0%B8%D1%8F_%D1%81%D0%B8%D1%81%D1%82%D0%B5%D0%BC#CITEREFРастригин1977