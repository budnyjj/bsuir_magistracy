\chapter[Обзор состояния проблемы идентификации стохастических систем]{%
  Обзор состояния проблемы идентификации стохастических систем
}

% По сути, это обзор литературы о проблеме идентификации.

% % В обзоре литературы, охватывающем не менее 30 источников за последние 10–15 лет (включая зарубежные публикации и электронные ресурсы), необходимо показать основные этапы в развитии знания по проблеме диссертации, критически осветив известные работы, необходимо назвать неразрешенные вопросы и таким образом определить свое место в решении проблемы. Желательно закончить этот раздел кратким резюме о той конкретной задаче, которую автор стремиться поставить и решить в диссертации.


% \section[Анализ состояния проблемы идентификации стохастических систем]{%
%   Анализ состояния проблемы идентификации \\
%   стохастических систем
% }


% \section[Математическая модель задачи идентификации стохастических систем]{%
%   Математическая модель задачи идентификации \\
%   стохастических систем
% }

% % Математическаяя модель задачи идентификации стохастических систем

% \section[Критерии и методы идентификации стохастических систем]{%
%   Критерии и методы идентификации \\
%   стохастических систем
% }

% В обзоре литературы, охватывающем не менее 30 источников за последние 10–15 лет (включая зарубежные публикации и электронные ресурсы), необходимо показать основные этапы в развитии знания по проблеме диссертации, критически осветив известные работы, необходимо назвать неразрешенные вопросы и таким образом определить свое место в решении проблемы. Желательно закончить этот раздел кратким резюме о той конкретной задаче, которую автор стремиться поставить и решить в диссертации.

% \section{Постановка задачи исследования}

% В обзоре литературы, охватывающем не менее 30 источников за последние 10–15 лет (включая зарубежные публикации и электронные ресурсы), необходимо показать основные этапы в развитии знания по проблеме диссертации, критически осветив известные работы, необходимо назвать неразрешенные вопросы и таким образом определить свое место в решении проблемы. Желательно закончить этот раздел кратким резюме о той конкретной задаче, которую автор стремиться поставить и решить в диссертации.
