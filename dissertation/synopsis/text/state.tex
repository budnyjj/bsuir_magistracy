\chapter[Анализ состояния проблемы. Постановка задачи]{%
  Анализ состояния проблемы. \hspace{2cm}
  Постановка задачи
}

\section[Анализ состояния проблемы идентификации стохастических систем]{%
  Анализ состояния проблемы идентификации \\
  стохастических систем
}

Задача идентификации в общем виде формулируется следующим образом:
на основании априорной информации, а также по результатам наблюдений над
входными и выходными переменными системы должна быть построена оптимальная в
некотором смысле модель, т.~е. формализованное представление этой системы.

Начало идентификации систем, как предмета построения математических моделей на основе наблюдений,
связано с работами~\cite{gauss_1809, gauss_1810, gauss_1821} Гаусса,
в которых он разработал метод наименьших квадратов и использовал его для предсказания
траектории движения планет. Существенный вклад в развитие метода наименьших квадратов внесли
Лаплас~\cite{laplace_1812}, Чебышев~\cite{chebyshev_1859}, Марков~\cite{markov_1898},
Эйткен~\cite{aitken_1935}, Колмогоров~\cite{kolmogorov_1946} и Рао~\cite{rao_1946}.
Полное и системное изложение этого метода представлено в монографии Линника~\cite{linnik62}.




% Впоследствии этот метод нашёл применение во множестве других приложений,
% в том числе и для построения математических моделей управляемых объектов,
% используемых в автоматизации (двигатели, печи, различные исполнительные механизмы).
% Большая часть ранних работ по идентификации систем была сделана специалистами в области статистики,
% эконометрики и образовала область под названием статистическое оценивание.

% Приблизительно до 50-х годов XX века, большая часть процедур идентификации в автоматике, основывалась на наблюдении реакций управляемых объектов при наличии некоторых управляющих воздействий (чаще всего воздействий вида: ступенчатое ( {\displaystyle H\cdot 1(t)} {\displaystyle H\cdot 1(t)}), гармоническое ( {\displaystyle \sin(\alpha ),\exp(j\omega )} {\displaystyle \sin(\alpha ),\exp(j\omega )}), сгенерированный цветной либо белый шум) и в зависимости от того какой вид информации использовался об объекте, методы идентификации делились на частотные и временные. Проблема заключалась в том, что область приложений этих методов была ограничена чаще всего скалярными системами(SISO,Single-input,single-output). В 1960 году Рудольф Калман представил описание управляемой системы в виде пространства состояний, что позволяло работать и с многомерными (MIMO,Many-input,many-output) системами, и заложил основы для оптимальной фильтрации и оптимального управления, основывавшихся на данном типе описания.

% Конкретно для задач управления, методы идентификации систем были разработаны в в 1965 году в работах Хо и Калмана[2], Острёма и Болина[3]. Эти работы открыли путь разработке двух методов идентификации, популярных до сих пор: методу подпространства и методу ошибки предсказания. Первый основан на использовании проекций в евклидовом пространстве, а второй на минимизации критерия, зависящего от параметров модели.

% Работа Хо и Калмана посвящена поиску модели изучаемого объекта в пространстве состояний, имеющей наименьший порядок вектора состояний, на основе информации об импульсной переходной характеристике. Данная задача, но уже при наличии реализаций случайного процесса, где формируется марковская модель, была решена в 70-х годах в работах Форре[4] и Акайка[5]. Эти работы заложили создание метода подпространства в начале 90-х.

% Работа же Острёма и Болина представила для сообщества специалистов по идентификации метод максимального правдоподобия, который был разработан специалистами по временным рядам для оценивания параметров моделей в виде разностных уравнений[6][7]. Эти модели, которые известны в статистической литературе как ARMA (авторегрессионное скользящее среднее) и ARMAX (авторегрессионное скользящее среднее с входом), позднее, образовали основу для создания метода ошибки предсказания. В 1970, Бокс и Дженкинс опубликовали книгу [8], которая дала значительный импульс к применению методов идентификации во всех возможных для этого областях. Этот труд давал, проще говоря, полный рецепт для идентификации с момента начала сбора информации об объекте до получения и проверки модели. На протяжении 15 лет, эта книга оставалась главным источником по идентификации систем. Важной работой того времени также являлся обзор [9], посвящённый идентификации систем и анализу временных рядов, опубликованный в IEEE Transactions on Automatic Control в декабре 1974 года. Одним из открытых вопросов тогда был вопрос об идентификации замкнутых систем, для которых метод на основе взаимной корреляции приводит к неудовлетворительным результатам [10]. С середины 70-х годов, недавно изобретённый метод ошибки предсказания стал доминировать в теории и, что более важно, в приложениях идентификации. Большая часть исследовательской активности сфокусировалась на проблемах идентификации многомерных и замкнутых систем. Ключевой задачей для этих двух классов систем являлось найти условия для эксперимента и способы параметризации проблемы, при которых найденная модель приблизится к единственно точному описанию реальной системы. Обо всей активности того времени можно сказать, что это было время поиска "истинной модели", решения вопросов идентифицируемости, сходимости к точным параметрам, статистической эффективности оценок и асимптотической нормальности оцениваемых параметров. К 1976 году была сделана первая попытка рассмотреть идентификацию систем как теорию аппроксимации, в которой стоит задача наилучшей возможной аппроксимации реальной системы внутри данного класса моделей [11][12],[13]. Преобладающая точка зрения среди специалистов по идентификации, таким образом, сменилась с поиска описания для истинной системы на поиск описания наилучшей возможной аппроксимации. Важный прорыв также случился, когда Л. Льюнг ввел понятие смещения и ошибки дисперсии для оценивания передаточных функций объектов [14]. Работа со смещением и анализ дисперсии полученных моделей в течение 1980-х привела к перспективе рассмотрения идентификации как проблемы синтеза. На основе понимания влияния условий эксперимента, структуры модели и критерия идентификации, основанном на смещении и дисперсии ошибки, возможно так подобрать эти переменные синтеза к объекту, чтобы получить наилучшую модель в данном классе моделей [15][16]. Данной идеологией пропитана книга Леннарта Льюнга [17], имеющая большое влияние на сообщество специалистов по идентификации.

% Идея, что качество модели может быть изменено с помощью выбора переменных синтеза, привела к всплеску активности в 90-х годах XX века, который продолжается до сих пор. Главное применение новой парадигмы — это идентификация для MBC (управление на основе модели). Соответственно, идентификация для задач управления расцвела с небывалой силой со времени своего появления и применение к управлению методов идентификации вдохнуло вторую жизнь в такие уже известные области исследования, как планирование эксперимента, идентификация в замкнутом контуре, частотная идентификация, робастное управление при наличии неопределенности.



В зависимости от характера располагаемой априорной информации различают
задачи идентификации систем в узком и широком смысле.
Задача идентификации в узком смысле состоит в оценивании параметров и
состояния системы по результатам наблюдений над входными и выходными переменными,
полученными в условиях функционирования объекта.
При этом известна структура системы и задан класс, к которому эта модель относится.

В случае, когда априорная информация об объекте при идентификации в широком смысле отсутствует
или очень бедная, принято говорить о задаче идентификации в широком смысле.
При её решении приходится решать большое число дополнительных задач:
выбор структуры системы и задание класса моделей,
оценивание степени стационарности, линейности объекта и действующих переменных,
оценивание степени и формы влияния входных переменных на выходные,
выбор информативных переменных и др.

{\color{red}
  В данной работе рассматривается задача идентификации в узком смысле,
  поскольку она является неотъемлемой частью задачи идентификации в широком смысле,
  а также поддаётся полной формализации.
}

% Теория идентификации систем, рассматривающая данную задачу,
% получила свое наибольшее развитие во второй половине XX века.
% Её выводы находят свое применение в различных отраслях науки и техники:
% энергетике, машиностроении, авиации, химической промышленности,
% физике, экономике, биологии и медицине при решении таких прикладных задач, как
% диагностика, управление, автоматический контроль,
% автоматизация принятия решений, распознавание образов.
% Данная теория опирается на следующие дисциплины:
% метрологию, теории управления, систем, сигналов, информации,
% стохастическую аппроксимацию и математическую статистику~\cite{eikhoff_1975}.


Стохастическая система --- система, связь входа и выхода которой имеет недетерминированный
(стохастический) характер. В связи с тем, что при измерении всякой реальной системы имеет место
случайная ошибка измерения, данную систему можно считать стохастической.

Для идентификации стохастических систем можно использовать методы ??? % математической статистики
.

% Краткий исторический обзор

% Первое изложение элементов метода наименьших квадратов дано
% в 1806 г. А. М. Лежандром *) [311 в связи с вопросом о вычислениях
% кометных орбит. Ему же принадлежит название: яметод наименьших
% квадратови.
%  В 1809 г. К. Ф. Гаусс **) [8] дал первое вероятностное обоснование метода наименьших квадратов, а в 1810 г. он же [9] глубоко
% разработал вычислительную сторону вопроса и ввел символы и обозначения, сохранившиеся и поныне. Ряд новых важных результатов
% найден К. Ф. Гауссом в 1821 г. [10].
%  В 1812 г. П. С. Лаплас ***) в фундаментальном трактате по
% теории вероятностей [30] получил ряд важных результатов и применил
% их к методу наименьших квадратов. Дальнейшие важные результаты
% были получены в теории метода наименьших квадратов в 1859 г.
% П. Л. Чебышевым ****), разработавшим теорию интерполирования по
% методу наименьших квадратов с помощью ортогональных полиномов,
% носящих его имя (см. гл. XII).
%  А. А. Марков *****) в 1898 г. в работе I37] и в уже упоминавшемся в § 1 курсе теории вероятностей [38] внес в математическую
%  статистику ряд весьма важных идей, пояснивших суть метода наименьших
%  Много сделано для развития применений метода наименьших
% квадратов в астрономии и геодезии Ф. Гельмертом ******) в
% прошлого века. После работ А. А. Маркова с двадцатых годов
% нынешнего века метод наименьших квадратов включился в математическую статистику как важная и естественная часть теории оценивания параметров (см. гл. III). В этой связи ряд интересных и важных
% результатов получен Ю. Нейманом и Ф. Дэвид (см. об этом гл. VII),
% А. Эйткеном [57] и С. Рао [42].
%  В 1946 г. А. Н. Колмогоровым [23] дано изящное геометрическое
% изложение метода наименьших квадратов.
% За последнее время в метод наименьших квадратов и его применения все более и более проникает матричное изложение. Оно позволяет вести удобную и короткую запись выкладок и результатов и
% в дальнейшем будет систематически использоваться.


% История.
% Для идентификации стохастических систем разработан ряд методов математической статистики.




% Статистические методы, используемые для идентификации. Схожесть с задачей аппроксимации.

% История развития рассматриваемых статистических методов (Гаусс? Пирсон? Линник? Муха?).

% Решенные и нерешенные проблемы.



% По сути, это обзор литературы о проблеме идентификации.



% Обзор литературы.
% Постановка задачи исследования.
% Схожесть с задачей аппроксимации.

% ~\cite{eikhoff_1975}
% Eikhoff_1975:
% Задача идентификации формулируется следующим образом:
% по результатом наблюдений над входными и выходными переменными системы
% должна быть построена оптимальная в некотором смысле модель,
% т. е. формализованное представление этой системы.

% В зависимости от априорной информации об объекте управления различают
% задачи идентификации в узком и широком смысле.
% Задача идентификации в узком смысле состоит в оценивании параметров и
% состояния системы по результатам наблюдений над входными и выходными переменными,
% полученными в условиях функционирования объекта. При этом известна структура
% системы и задан класс моделей, к которому этот класс относится.

% Априорная информация об объекте при идентификации в широком смысле отсутствует
% или очень бедная, поэтому приходится решать большое число дополнительных задач.
% К этим задачам относятся:
% выбор структуры системы и задание класса моделей,
% оценивание степени стационарности, линейности объекта и действующих переменных,
% оценивание степени и формы влияния входных переменных на выходные,
% выбор информативных переменных и др.

% Построение модели сводится к следующим этапам:
% \begin{itemize}
% \item выбор структуры модели из физических соображений;
% \item подгонка параметров к имеющимся данным (оценивание);
% \item проверка и подтверждение модели (диагностическая проверка);
% \item испольование модели по назначению.
% \end{itemize}

% Структура модели выбирается на основе априорной информации о системе и
% преследуемых целях.

% В большинстве реальных ситуаций налюдения над системой искажены
% случайными воздествиями (возмущениями, ошибками).

% Под оцениванием параметров понимается экспериментальное определение
% значений параметров, характеризующих динамику поведения объекта,
% в предположении, что структура модели объекта известна.

% Как оценить точность идентификации -- по отклонениям параметров
% модели или её отклика? Если основная цель состоит в
% проектировании системы управления, то представляется логичным
% оценивать точность идентификации по результатам функционирования
% созданной на основе идентификации объекта системы управления.

% теория систем -- линейные и нелинейные модели
% теоретическая дисциплина -- стохастическая аппроксимация ?
% цели применения.

% Выбор между линейными и нелинейными моделями.
% Скорость изменения параметров?
% Выбор типа модели тесно связан с решаемой задачей.

% Два вида реализаций:
% - использующая явные математические выражения;
% - реализации по настраиваемой модели.

% Методы идентификации:
% - вне контура регулирования;
% - внутри замкнутого контура регулирования.

% Дополнительная априорная информация приводит к улучшению оценик,
% однако затраты на реализацию могут оказаться чрезмерными.

% Метод наименьших квадратов может быть получен из метода максимума правдоподобия.

% Источники ошибок:
% - ошибки, вызванные помехами;
% - ошибки усечения;
% - ошибки из-за неправильного определения состояния;
% - ошибки, связанные с упрощением при реализации;
% - ошибки выборочной аппроксимации.

% Есть ли связь между НМНК и марковскими оценками?

% В обзоре литературы, охватывающем не менее 30 источников за последние 10–15 лет (включая зарубежные публикации и электронные ресурсы), необходимо показать основные этапы в развитии знания по проблеме диссертации, критически осветив известные работы, необходимо назвать неразрешенные вопросы и таким образом определить свое место в решении проблемы. Желательно закончить этот раздел кратким резюме о той конкретной задаче, которую автор стремиться поставить и решить в диссертации.


\section[Идентификация стохастических систем как математическая проблема]{%
  Идентификация стохастических систем как \\
  математическая проблема}

\section{Классификация стохастических систем}

Скалярные/векторные, линейные/нелинейные.

\section{Критерии идентификации стохастических систем}

Критерием выбора оптимума должен быть функционал от выходных
сигналов или от математического ожидания ошибок оценок параметров.

\section{Постановка задачи исследования}

% https://ru.wikipedia.org/wiki/%D0%98%D0%B4%D0%B5%D0%BD%D1%82%D0%B8%D1%84%D0%B8%D0%BA%D0%B0%D1%86%D0%B8%D1%8F_%D1%81%D0%B8%D1%81%D1%82%D0%B5%D0%BC#CITEREFРастригин1977