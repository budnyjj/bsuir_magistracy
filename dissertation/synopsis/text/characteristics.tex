\chapter*{Общая характеристика работы}
\addcontentsline{toc}{chapter}{Общая характеристика работы}

\paragraph{Связь работы с научными исследованиями университета}
Диссертационная работа была выполнена в рамках госбюджетной
научно-исследовательской работы ГБ \textnumero 16-2016 кафедры
информационных технологий автоматизированных систем БГУИР
<<Модели и методы оптимизации решений в современных
информационно-аналитических и управляющих системах>>
на 2016--2020 годы.

\paragraph{Цель и задачи исследования}
Предметом исследования работы является идентификация стохастических систем.

Целью работы является формирование рекомендаций по предпочтительному использованию
некоторых методов аппроксимации статистических данных для идентификации
стохастических систем на основе результатов численного сравнительного
анализа их точности.

В задачи работы входят выполнение численного сравнительного анализа
точности методов идентификации стохастических систем второго типа,
описание условий предпочтительного использования сравниваемых методов
для оценивания параметров систем.

\paragraph{Новизна полученных результатов}

В работе получены следующие новые научные результаты:
\begin{enumerate}
\item предложена классификация стохастических объектов для целей их
  идентификации и сравнения;
\item разработан алгоритм метода симметричной аппроксимации;
\item получена эмпирическая зависимость для выбора более точного метода
  оценивания параметров линейных стохастических систем второго типа;
\item для идентификации нелинейных стохастических систем
  адаптированы нелинейный метод наименьших квадратов и метод рядов Тейлора;
\item описаны условия предпочтительного использования данных
  методов для идентификации нелинейных стохастических второго типа.
\end{enumerate}

\paragraph{Положения, выносимые на защиту}

\begin{enumerate}
\item Классификация стохастических объектов, выделяющая полустохастические
  объекты и стохастические объекты первого и второго типов,
  позволяющая выявить проблемный для исследования класс стохастических
  объектов второго типа.
\item Алгоритм метода симметричной аппроксимации,
  позволяющий применять данный метод для идентификации систем с помощью ЭВМ.
\item Эмпирическая зависимость, предназначенная для выбора метода
  оценивания параметров линейных стохастических систем второго типа.
\item Условия предпочтительного использования нелинейного метода наименьших квадратов и
  метода рядов Тейлора для идентификации нелинейных стохастических второго типа.
\end{enumerate}

\paragraph{Апробация результатов диссертации}
Результаты исследований были представлены на следующих конференциях:
\begin{enumerate}
\item <<51-ая научная конференции аспирантов, магистрантов и студентов>>
  учреждения образования
  <<Белорусский государственный университет информатики и радиоэлектроники>>,
  БГУИР, Минск, Беларусь, 13--17 апреля 2015 года.
\item <<53-ая научная конференции аспирантов, магистрантов и студентов>>
  учреждения образования
  <<Белорусский государственный университет информатики и радиоэлектроники>>,
  БГУИР, Минск, Беларусь, 2--6 мая 2017 года.
\item Международная научная конференция
  <<Информационные технологии и системы>>,
  БГУИР, Минск, Беларусь, 25 октября 2017 года.
\end{enumerate}

\paragraph{Опубликованность результатов исследования}
Результаты исследования были опубликованы в виде материалов
указанных выше трех конференций.

\paragraph{Структура и объем диссертации}

Работа 61 с., 18 рисунков (15 с.), 2 таблицы (2 с.), 4 приложения (10 с.),
34 источника.

Работа состоит из трех глав.
В первой главе работы приведен краткий исторический обзор развития
теории идентификации систем, а также математическая модель задачи идентификации
в общем виде. Предложена классификация стохастических систем для целей
их идентификации и сравнения. Рассмотрены классический и симметричный
критерии идентификации систем. Выполнена постановка задачи исследования.

Вторая глава работы посвящена выявлению условий предпочтительного использования
классического и симметричного критериев идентификации для
оценивания параметров линейных стохастических объектов второго типа,
а также прогнозирования наблюдений их выхода по наблюдениям входа.
Приведены алгоритмы методов, основанных на указанных критериях идентификации.
На основании результатов сравнительного численного анализа получена эмпирическая
зависимость для выбора более точного метода оценивания параметров
идентифицируемых систем.

В третьей главе работы рассмотрена задача идентификации нелинейных
стохастических систем второго типа.
Для решения данной задачи предложено использовать нелинейный метод
наименьших квадратов и метод рядов Тейлора.
Приведены алгоритмы данных методов.
На основании результатов сравнительного численного анализа описаны условия
предпочтительного использования методов для идентификации ряда нелинейных систем.
