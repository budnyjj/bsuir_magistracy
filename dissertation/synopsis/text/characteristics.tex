\chapter*{Общая характеристика работы}
\addcontentsline{toc}{chapter}{Общая характеристика работы}

{\color{red}
\paragraph{Связь работы с научными исследованиями университета}
Результаты диссертационного исследования были использованы при
проведении следующих научно исследовательских работ (указать
конкретное наименование научно исследовательских работ).

\paragraph{Цель и задачи исследования}
Предметом работы является задача идентификации стохастических систем.

Целью работы является формирование рекомендаций по предпочтительному использованию
некоторых методов аппроксимации статистических данных для идентификации
стохастических систем на основе результатов численного сравнительного
анализа их точности.

В задачи работы входят выполнение численного сравнительного анализа
точности методов идентификации стохастических систем второго типа,
описание условий предпочтительного использования сравниваемых методов
для оценивания параметров систем.

\paragraph{Новизна полученных результатов}

В работе получены следующие новые научные результаты:
\begin{enumerate}
\item предложена классификация стохастических объектов для целей их идентификации и сравнения;
\item разработан алгоритм метода симметричной аппроксимации;
\item получена эмпирическая зависимость для выбора более точного метода
  оценивания параметров линейных стохастических систем второго типа;
\item для идентификации нелинейных стохастических систем
  адаптированы нелинейный метод наименьших квадратов и метод рядов Тейлора;
\item описаны условия предпочтительного использования данных
  методов для идентификации нелинейных стохастических второго типа.
\end{enumerate}

\paragraph{Положения, выносимые на защиту}

Метод симметричной аппроксимации оценивает параметры линейных систем с
большим коэффициентом усиления более точно, чем классический метод
наименьших квадратов. Для оценивания параметров линейных стохастических
систем во всех остальных случаях, а также для прогнозирования наблюдений
выхода систем по наблюдениям входа рекомендуется использовать метод
наименьших квадратов.

Точность оценивания параметров нелинейных стохастических систем
нелинейным методом наименьших квадратов существенным образом
зависит от выбора опорной точки.
В аналогичных условиях метод рядов Тейлора позволяет получать оценки параметров
приемлемой точности без необходимости указания опорной точки.

\paragraph{Апробация результатов диссертации}
Результаты исследований были представлены на следующих конференциях:
\begin{enumerate}
\item "51-ая научная конференции аспирантов, магистрантов и студентов"
  учреждения образования
  "Белорусский государственный университет информатики и радиоэлектроники";
\item "53-ая научная конференции аспирантов, магистрантов и студентов"
  учреждения образования
  "Белорусский государственный университет информатики и радиоэлектроники".
\end{enumerate}

\paragraph{Опубликованность результатов исследования}
Результаты исследования были опубликованы в виде статей в
материалах и тезисах трех конференций.

\paragraph{Структура и объем диссертации}

Работа 61 с., 18 рисунков (15 с.), 2 таблицы (2 с.), 4 приложения (10 с.),
34 источника.

Работа состоит из трех глав.
Первая глава работы содержит краткий исторический обзор развития теории
идентификации систем, классификацию стохастических систем,
а также постановку задачи исследования.
Предметом второй и третьей главы является численный анализ точности идентификации
линейных и нелинейных стохастических систем.
}