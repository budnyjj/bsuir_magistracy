\addcontentsline{toc}{chapter}{Библиографический список}
\chapter*{БИБЛИОГРАФИЧЕСКИЙ СПИСОК}

\addcontentsline{toc}{section}{Список использованных источников}
\section*{Список использованных источников}

    {
\renewcommand{\chapter}[2]{}
\bibliography{bib/bibliography}
}

\addcontentsline{toc}{section}{Список публикаций соискателя}
\section*{Список публикаций соискателя}

\noindent [1"---A.] Будный Р. И.
Оценивание параметров нелинейной регрессионной
\hspace*{16.5mm} зависимости / Р. И. Будный // Материалы 51-ой научной конферен-
\hspace*{17mm} ции аспирантов, магистрантов и студентов --- Минск, 2015. --- C. 55.

\noindent [2"---A.] Будный Р. И.
Численный анализ методов линейной аппроксимации
\hspace*{15.5mm} статистических данных / Р. И. Будный // Материалы 53-ой научной
\hspace*{16mm} конференции аспирантов, магистрантов и студентов --- Минск, 2017.
\hspace*{16.5mm} --- C. 68.

\noindent [3"---A.] Муха В. С.
О линейной аппроксимации векторных статистических
\hspace*{15.5mm} данных / В. С. Муха, Р. И. Будный // Информационные технологии
\hspace*{16.5mm} и системы 2017 (ИТС 2017): материалы международной научной кон-
\hspace*{15.5mm} ференции, БГУИР, Минск, Беларусь, 25 октября 2017 г. --- Минск,
\hspace*{16.5mm} 2017. --- C. 288--289.
