\chapter[Идентификация нелинейных стохастических систем]{%
  Идентификация нелинейных \hspace{2cm}
  стохастических систем
}

\section{Математическая модель идентифицируемой системы}

Для обеспечения наглядности сравнения рассматривался случай идентификации скалярной системы:
\begin{equation}
  \label{eq:model_nonlinear_scalar}
  \begin{aligned}
  h &= \psi(\theta, \xi), \\
  x &= \xi + \varepsilon_x, \\
  y &= h + \varepsilon_y,
  \end{aligned}
\end{equation}
где \( \xi, h \) "--- фактические значения входной и выходной переменной, \par
\( \theta = (\theta_1, \theta_2, \dotsc, \theta_m) \) "--- вектор фактических значений параметров объекта, \par
\( x, y \) "--- измеренные значения входной и выходной переменной, \par
\( \varepsilon_x, \varepsilon_y \) "--- независимые ошибки измерений значений входной и
выходной переменной, распределенные по нормальному закону:
\(
\varepsilon_x = N(0, \sigma_{\varepsilon_x}),
\varepsilon_y = N(0, \sigma_{\varepsilon_y})
\).

Данная модель использовалась для генерации наблюдений входа и выхода системы,
на основании которых были получены оценки её параметров методами на основе
классического и симметричного критериев аппроксимации.
Значения \( \xi_i \) выбирались из равномерного в \( [0, 10] \) распределения.
Для получения каждой оценки \( \hat{\alpha}, \hat{\beta} \) использовались результаты
ста наблюдений \( ( x_i, y_i ), i = \overline{1, n}, n = 100 \).

\section{Алгоритмы методов идентификации}

\subsection{Нелинейный метод наименьших квадратов}

Один из подходов к оценке параметров системы~\eqref{eq:model_nonlinear_scalar} состоит в следующем.
Можно <<закрыть глаза>> на существование ошибок измерений
эндогенной переменной, то есть считать, что \( \varepsilon_x = 0 \),
и вместо данной модели рассматривать модель
\begin{equation}
  \label{eq:model_nonlinear_lse}
  \begin{aligned}
  x &= \psi(\theta, \xi), \\
  y &= x + \varepsilon_y.
  \end{aligned}
\end{equation}

Тогда оценка вектора параметров объекта определяется выражением~\cite{mukha_2009}
\begin{equation}
  \label{eq:nonlinear_lse}
  \hat{\Theta}_{\text{НМНК}} =
  \Theta_0 + (Q^T R^{-1}_{\Xi} Q)^{-1} Q^T R^{-1}_{\Xi} (y - \psi(\Theta_0, x)),
\end{equation}
где \( \Theta_0 \) --- опорная точка,
\( Q = \dfrac{\partial \psi(\Theta_0, x) }{ \partial \Theta_0 } \).

Эту оценку будем называть оценкой, полученной нелинейным
методом наименьших квадратов (НМНК).
В качестве опорной точки \( \Theta_0 \) можно использовать значение
\( \theta_1, \dotsc, \theta_m \),
полученное в результате численного решения системы уравнений
\begin{equation}
  \label{eq:basic}
  (y_j - \psi( \Theta, x_j )) = 0, \: j = \overline{1,m}.
\end{equation}
Для уточнения оценки, полученной по формуле~\eqref{eq:nonlinear_lse}, можно
организовать итерационную процедуру, заменяя опорную точку полученной оценкой.

\newpage
\subsection{Метод рядов Тейлора}

Применение метода рядов Тейлора (МРТ)~\cite{mukha_2000}
требует иной формулировки задачи, допускаемой формулировкой \eqref{eq:model_nonlinear_scalar}.
Следует предположить, что \( j \)-е наблюдение вектора параметров \( \Theta_j \)
определяется как векторная функция показаний приборов:
\begin{equation}
  \label{eq:mrt_phi}
  \Theta_j = \phi( \overline{z}_{j} ), \: j = \overline{1, n},
\end{equation}
где вектор
\( \overline{z}^{\text{T}}_{j} =
( \overline{x}^{\text{T}}_{j}, \overline{y}^{\text{T}}_{j}) \)
имеет нормальное распределение \( N(A_{z,j}, R_{z,j}) \)
и оцениваемый векторный параметр \( \Theta \) определяется как
\( \Theta = \Phi(A_{z,j}), \forall i = \overline{1, n} \).

В этом случае МРТ-оценка \( \hat{\Theta}_{\text{МРТ}} \) векторного параметра \( \Theta \)
определяется выражением
\begin{equation}
  \label{eq:nonlinear_mrt}
  \hat{\Theta}_{\text{МРТ}} =
  \Bigg( \sum^{n}_{i=1} R^{-1}_{\Theta,i} \Bigg)^{-1}
  \sum^{n}_{j=1} R^{-1}_{\Theta,j} \Theta_j,
\end{equation}
\hspace{1.6mm} где
\( R_{\Theta,i} = G_i R_{z,i} G^T_i \),
\( G_i =
\dfrac{\partial \phi( \overline{z}_{i} ) }{ \partial \overline{z}_{i} } \).

Численное значение наблюдения \( \Theta_j \)~\eqref{eq:mrt_phi} может определяться как
решение системы уравнений~\eqref{eq:basic}.


\section{Численный анализ точности оценивания параметров систем}

\subsection{Линейная функция регрессии}

\( \alpha = 1 \)

\subsection{Экспоненциальная функция регрессии}

\subsection{Синусоидальная функция регрессии}

Эмпирические выводы о точности.