\chapter*{ВВЕДЕНИЕ}
\addcontentsline{toc}{chapter}{Введение}

Многие проблемы, возникающие в различных отраслях науки и техники,
можно свести к задаче идентификации систем.
Диагностика, управление, автоматизация принятия решений, распознавание образов
"--- лишь некоторые практические цели решения данной задачи.
Получение новых научных результатов в этой области приводит к
развитию близких прикладных проблем,
таких как детектирование, прогнозирование и обучение.

В работе рассматривается задача идентификации стохастических систем.
Данный класс систем отличается случайным характером связи наблюдаемых
значений входа и выхода.
Для решения данной задачи применяются математические методы
аппроксимации статистических данных.

Целью работы является сравнительный анализ некоторых существующих методов аппроксимации,
формирование на этой основе рекомендаций по их предпочтительному использованию.

Работа имеет следующую структуру.
Первая глава содержит краткий исторический обзор развития теории идентификации систем,
классификацию стохастических систем, а также постановку задачи исследования.
Предметом второй и третьей главы является численный анализ точности идентификации
линейных и нелинейных стохастических систем второго типа соответственно.