\chapter*{Введение}
\addcontentsline{toc}{chapter}{Введение}

Многие проблемы, решаемые в различных отраслях науки и техники, сводятся к задаче идентификации систем.
Диагностика, управление, автоматизация принятия решений, распознавание образов "---
лишь некоторые практические цели решения данной задачи.
Получение новых научных результатов в этой области приводит к развитию близких прикладных проблем,
таких как детектирование, прогнозирование и обучение.

В работе рассматривается задача идентификации стохастических систем.
Данный класс систем характеризуется случайностью связи входных и выходных переменных.
Для решения данной задачи используются математические методы аппроксимации статистических данных.

Целью работы является сравнительный анализ некоторых существующих методов аппроксимации,
формирование на этой основе рекомендаций по их предпочтительному использованию.

Задачами исследования являются анализ методов линейной аппроксимации с целью определения их
места в задачах идентификации {\color{red} и разработка методов нелинейной аппроксимации}.

% Работа имеет следующую структуру.
% Структура работы.