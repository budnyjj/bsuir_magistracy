\section{Краткое введение}

Многие проблемы, решаемые в различных отраслях науки и техники,
можно свести к задаче идентификации систем.
Диагностика, управление, автоматизация принятия решений, распознавание образов
"--- лишь некоторые практические цели решения данной задачи.
Получение новых научных результатов в этой области приводит к
развитию близких прикладных задач,
таких как детектирование, прогнозирование и обучение.

Существующие исследования методов идентификации систем нельзя считать
достаточно завершенными: отсутствует обоснование используемых критериев,
не приводится сравнительный численный анализ различных алгоритмов идентификации,
отсутствуют рекомендации по применению того или иного алгоритма.
Представляется целесообразной попытка независимого решения этих задач.
