\section{Краткое содержание работы}

В первой главе работы сформулирована задача идентификации стохастических систем как
задача оценивания параметров и состояния системы по результатам наблюдений над
входными и выходными переменными, полученными в условиях её функционирования.

Предложена следующая классификация стохастических систем.
\begin{enumerate}
\item \emph{Полустохастический объект} --- объект с детерминированным входом и
  случайным выходом.
\item \emph{Стохастический объект первого типа} --- объект со случайным входом и выходом.
  Возможно наличие ошибок в измерениях входных и выходных переменных.
\item \emph{Стохастический объектом второго типа} --- детерминированный объект с ошибками
  в измерениях входных и выходных переменных.
\end{enumerate}

Рассмотрены классический и симметричный критерии идентификации стохастических систем,
минимизирующие суммы квадратов вертикальных и перпендикулярных расстояний от наблюдений
входных и выходных переменных до аппроксимирующей прямой или плоскости.

Выявлено, что области предпочтительного использования классического и симметричного
критериев для идентификации линейных стохастических систем второго типа
определены недостаточно четко, а также обоснована целесообразность
описания условий предпочтительного использования методов идентификации
нелинейных стохастических систем.
Исходя из этого определены задачи исследования.

Вторая глава работы посвящена выявлению условий предпочтительного использования
классического и симметричного критериев идентификации
для оценивания параметров стохастических объектов второго типа,
а также прогнозирования наблюдений их выхода по наблюдениям входа.
Разработаны алгоритмы классического метода наименьших квадратов и метода симметричной
аппроксимации, использующие классический и симметричный критерии идентификации.

Для сравнения точности оценивания параметров использовалась разность средних
Евклидовых расстояний между точными значениями параметров модели и их оценками.
Исследована зависимость данной величины от коэффициента усиления модели и
с.~к.~о. ошибок наблюдений.
Показано, что метод симметричной аппроксимации оценивает
параметры линейных систем с большим коэффициентом усиления более точно,
чем классический метод наименьших квадратов.
Для принятия решения о том, какой метод использовать более предпочтительно,
предложено использовать следующую эмпирическую зависимость:
<<Если условие
\begin{equation*}
  \sigma_{\varepsilon_y} > (0{,}7 + |\beta|) \sigma_{\varepsilon_x}
\end{equation*}
выполняется, то метод наименьших квадратов оценивает параметры линейной
стохастической системы второго типа более точно, чем метод симметричной аппроксимации.
В противном случае метод симметричной аппроксимации позволяет получить
оценки более высокой точности, чем метод наименьших квадратов>>.

Для сравнения точности прогнозирвания наблюдений выхода по наблюдениям входа
использовалась разность средних Евклидовых расстояний между наблюдениями выхода
модели и их оценками.
Исследована зависимость данной величины от коэффициента усиления модели и
с.~к.~о. ошибок наблюдений.
На основании результатов исследования для прогнозирования наблюдений выхода
систем по наблюдениям входа рекомендовано использовать метод наименьших квадратов.

Предметом третьей главы является проблема идентификации нелинейных стохастических систем.
Для решения задачи идентификации нелинейных стохастических систем второго типа
предложено использовать нелинейный метод наименьших квадратов и метод рядов Тейлора.
Приведены алгоритмы и примеры использования программных реализаций данных методов.
На примере ряда систем с нелинейными функциями регрессии исследована зависимость разности
средних Евклидовых расстояний между точными значениями параметров модели и их оценками
от с.~к.~о. ошибок наблюдений.

Показано, что точность нелинейного метода наименьших квадратов существенным образом
зависит от того, насколько <<удачной>> является используемая им опорная точка.
В аналогичных условиях метод рядов Тейлора позволяет получать оценки параметров
приемлемой точности без необходимости указания опорной точки.

В заключении перечислены новые научные результаты, полученные в работе.

% По результатам анализа исследований в области идентификации
% Известно, что оптимальным решением задач идентификации линейного полустохастического объекта,
% а также линейного стохастического объекта первого типа является классическая линейная регрессия.
% Области предпочтительного классического и симметричного критериев для идентификации
% стохастических объектов второго типа определены недостаточно четко.

% Проблема идентификации нелинейных стохастических систем разработана
% в недостаточной степени.
% Существующие исследования нельзя считать достаточно завершенными:
% отсутствует обоснование используемых критериев,
% не приводится сравнительный численный анализ различных алгоритмов,
% отсутствуют рекомендации по применению того или иного алгоритма.


% Вторая глава работы посвящена выявлению условий предпочтительного использования этих критериев
% для оценивания параметров стохастических объектов второго типа,
% а также прогнозирования наблюдений их выхода по наблюдениям входа.


% математическая модель задачи идентификации
% классификация стохастических систем для целей их идентификации и сравнения
% классический и симметричный критерии идентификации систем
% постановка задачи исследования

% В первой главе работы приведен краткий исторический обзор развития
% теории идентификации систем, а также математическая модель задачи идентификации
% в общем виде. Предложена классификация стохастических систем для целей
% их идентификации и сравнения. Рассмотрены классический и симметричный
% критерии идентификации систем. Выполнена постановка задачи исследования.

% Вторая глава работы посвящена выявлению условий предпочтительного использования
% классического и симметричного критериев идентификации для
% оценивания параметров линейных стохастических объектов второго типа,
% а также прогнозирования наблюдений их выхода по наблюдениям входа.
% В главе приведены алгоритмы методов, основанных на указанных критериях идентификации.
% На основании результатов сравнительного численного анализа получена эмпирическая
% зависимость для выбора более точного метода оценивания параметров
% идентифицируемых систем.

% В третьей главе работы рассмотрена задача идентификации нелинейных
% стохастических систем второго типа.
% Для решения данной задачи предложено использовать нелинейный метод
% наименьших квадратов и метод рядов Тейлора.
% Приведены алгоритмы данных методов.
% На основании результатов сравнительного численного анализа описаны условия
% предпочтительного использования методов для идентификации ряда нелинейных систем.


% Моделирование показало, что метод симметричной аппроксимации оценивает
% параметры линейных систем с большим коэффициентом усиления более точно,
% чем классический метод наименьших квадратов.
% Для оценивания параметров линейных стохастических систем во всех остальных случаях,
% а также для прогнозирования наблюдений выхода систем по наблюдениям входа
% рекомендуется использовать метод наименьших квадратов.

% На примере ряда нелинейных стохастических систем второго типа было показано,
% что точность нелинейного метода наименьших квадратов существенным образом
% зависит от того, насколько <<удачной>> является используемая им опорная точка.
% В аналогичных условиях метод рядов Тейлора позволяет получать оценки параметров
% приемлемой точности без необходимости указания опорной точки.
% В связи с этим представляется целесообразным применение метода рядов Тейлора
% для определения опорной точки, используемой для оценивания параметров методом
% наименьших квадратов.
% Данный подход позволит получать точные оценки параметров идентифицируемой системы
% независимо от вида её функции регрессии.

% В заключении перечислены новые научные результаты, полученные в работе.