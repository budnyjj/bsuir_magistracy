\section{Общая характеристика работы}
\addcontentsline{toc}{section}{Общая характеристика работы}

\paragraph{Связь работы с научными исследованиями университета}
Результаты диссертационного исследования были использованы при
проведении следующих научно исследовательских работ (указать
конкретное наименование научно исследовательских работ).

\paragraph{Цель и задачи исследования}
В пункте «Цель и задачи исследования» формулируется цель работы и задачи,
которые необходимо решить для достижения поставленной цели.
Не рекомендуется формулировать цель как
«Исследование...», «Изучение...», так как эти слова указывают на средство
достижения цели, а не на саму цель. Здесь же указывается объект и
предмет исследования и обосновывается их выбор.

\paragraph{Новизна полученных результатов}
В пункте «Новизна полученных результатов» необходимо показать отличие полученных
результатов от
известных,
описать
степень
новизны
(впервые
получено,
усовершенствовано, дано дальнейшее развитие и т.п.).

\paragraph{Положения, выносимые на защиту}
В пункте «Положения,
выносимые на защиту» в сжатой форме отражается сущность полученных
научных результатов. В формулировке положений, выносимых на защиту,
должны содержаться отличительные признаки новых научных
результатов, характеризующие вклад соискателя в область науки, к
которой относится тема диссертации. Они должны содержать не только
краткое изложение сущности полученных новых результатов, но и
сравнительную оценку их научной и практической значимости.

\paragraph{Апробация результатов диссертации} В пункте «Апробация
результатов диссертации» указывается, на каких конференциях, семинарах
и т.п. были представлены результаты исследований, включенные в
магистерскую диссертацию.

\paragraph{Опубликованность результатов исследования}
В пункте «Опубликованность результатов исследования» указывается в скольких
статьях в научных журналах, сборниках, тезисах конференций, патентах
опубликованы результаты работы.

\paragraph{Структура и объем диссертации}
В пункте «Структура и объем диссертации» кратко излагается структура работы и
поясняется логика ее построения.
Приводится полный объем диссертации в страницах, объем,
занимаемый иллюстрациями, таблицами, приложениями (с указанием их
количества), а также количество использованных библиографических
источников (включая собственные публикации соискателя).