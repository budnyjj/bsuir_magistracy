\section{Заключение}

В данной работе:
\begin{enumerate}
\item предложена классификация стохастических объектов для целей
  идентификации и сравнения;
\item разработаны алгоритмы классического метода наименьших квадратов и
  метода симметричной аппроксимации;
\item получена эмпирическая зависимость для выбора более точного метода
  оценивания параметров линейных стохастических систем второго типа;
\item для идентификации нелинейных стохастических
  систем адаптированы нелинейный метод наименьших квадратов и метод рядов Тейлора;
\item описаны условия предпочтительного использования
  методов идентификации нелинейных стохастических второго типа;
\item разработаны программные реализации упомянутых методов.
\end{enumerate}

Моделирование показало, что метод симметричной аппроксимации оценивает
параметры линейных систем с большим коэффициентом усиления более точно,
чем классический метод наименьших квадратов.
Для оценивания параметров линейных стохастических систем во всех остальных случаях,
а также для прогнозирования наблюдений выхода систем по наблюдениям входа
рекомендуется использовать метод наименьших квадратов.

На примере ряда нелинейных стохастических систем второго типа было показано,
что точность нелинейного метода наименьших квадратов существенным образом
зависит от того, насколько <<удачной>> является используемая им опорная точка.
В аналогичных условиях метод рядов Тейлора позволяет получать оценки параметров
приемлемой точности без необходимости указания опорной точки.
В связи с этим представляется целесообразным применение метода рядов Тейлора
для определения опорной точки, используемой для оценивания параметров методом
наименьших квадратов.
Данный подход позволит получать точные оценки параметров идентифицируемой системы
независимо от вида её функции регрессии.
