\section{Заключение}

Моделирование показало, что метод симметричной аппроксимации оценивает
параметры линейных систем с большим коэффициентом усиления более точно,
чем классический метод наименьших квадратов.
Для оценивания параметров линейных стохастических систем во всех остальных случаях,
а также для прогнозирования наблюдений выхода систем по наблюдениям входа
рекомендуется использовать метод наименьших квадратов.

На примере ряда нелинейных стохастических систем второго типа было показано,
что точность нелинейного метода наименьших квадратов существенным образом
зависит от того, насколько <<удачной>> является используемая им опорная точка.
В аналогичных условиях метод рядов Тейлора позволяет получать оценки параметров
приемлемой точности без необходимости указания опорной точки.
В связи с этим представляется целесообразным применение метода рядов Тейлора
для определения опорной точки, используемой для оценивания параметров методом
наименьших квадратов.
Данный подход позволит получать точные оценки параметров идентифицируемой системы
независимо от вида её функции регрессии.
