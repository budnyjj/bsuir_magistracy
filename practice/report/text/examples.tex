\chapter{Примеры задач аппроксимации статистических данных}
% 1. Выполнить поиск практических примеров постановки задач аппроксимации
% статистических данных в литературных источниках;
% 2. Описать исходные данные для сравнения точности рассматриваемых методов
% линейной и нелинейной аппроксимации;
% 3. Определить место результатов исследований в учебной дисциплине
% «Статистические методы обработки данных».

% \chapter{Обзор литературы}

% \section{Раздел}

% \subsection{Подраздел}

% В обзоре литературы, охватывающем не менее 30 источников за последние 10–15 лет (включая зарубежные публикации и электронные ресурсы), необходимо показать основные этапы в развитии знания по проблеме диссертации, критически осветив известные работы, необходимо назвать неразрешенные вопросы и таким образом определить свое место в решении проблемы. Желательно закончить этот раздел кратким резюме о той конкретной задаче, которую автор стремиться поставить и решить в диссертации.

% \chapter{Теоретический раздел}

% В последующих главах с исчерпывающей полнотой  необходимо изложить собственное исследование с выявлением того нового и оригинального, что вносится в разработку проблемы. Все идеи и положения автора должны быть обстоятельно обоснованы на базе принятой методики, вытекающей из сущности предмета исследования.

% \chapter{Экспериментальный раздел}

% Весь порядок изложения в диссертации должен быть подчинен  руководящей идее, четко сформулированной в теоретическом разделе диссертации. Логичность построения и целеустремленность изложения  глав достигается в случае, если каждая из глав имеет определенное целевое назначение и является базой для последующей.
