\chapter*{Введение}
\addcontentsline{toc}{chapter}{Введение}

Аппроксимация --- научный метод, состоящий в замене одних объектов другими,
в каком-то смысле близкими к исходным, но более простыми.
Аппроксимация позволяет исследовать числовые характеристики и
качественные свойства объекта, сводя задачу к изучению более простых или
более удобных объектов (например, таких, характеристики которых легко
вычисляются или свойства которых уже известны)~\cite{wiki_approximation}.
Данный метод находит широкое применение в различных областях науки и техники.

Целью данной работы является поиск и систематизация материала,
связанного с темой аппроксимации функций скалярного аргумента.
В работе приведены примеры постановки задач аппроксимации статистических данных,
описаны исходные данные для сравнения точности различных методов
линейной и нелинейной аппроксимации.