\chapter{Исходные данные для сравнения методов аппроксимации}

\paragraph*{Линейная аппроксимация}
В работе~\cite{budny17} было выполнено сравнение точности оценок параметров
линейной зависимости по измерениям с ошибками,
полученных классической линейной регрессией и методом симметричной аппроксимации.
Рассматривался объект с математической моделью
\begin{equation}
  \label{eq:linear_model}
  h = \alpha + \beta \xi,
  x = \xi + \varepsilon,
  y = h + \delta
\end{equation}
где
\( x, y \) --- фактические значения входных и выходных переменных соответственно,
\( \varepsilon, \delta \) --- фактические значения входных и выходных переменных соответственно,
...

Для линейного объекта со скалярным входом \( x \), выходом \( y \) и параметрами
\( \alpha, beta \) исследовалась зависимость точности оценок его параметров
\( \hat{\alpha}, \hat{beta} \) от значений \( alpha, \beta \),
а также от с.~к.~о. ошибок измерений \( \sigma_{\delta}, \sigma_{\xi} \).

\paragraph*{Нелинейная аппроксимация}
В работе~\cite{budny15} было выполнено сравнение точности оценивания параметров
нелинейных зависимостей по измерениям с ошибками,
полученных нелинейным методом наименьших квадратов и методом рядов Тейлора.



% 1. Выполнить поиск практических примеров постановки задач аппроксимации
% статистических данных в литературных источниках;
% 2. Описать исходные данные для сравнения точности рассматриваемых методов
% линейной и нелинейной аппроксимации;
% 3. Определить место результатов исследований в учебной дисциплине
% «Статистические методы обработки данных».

% \chapter{Обзор литературы}

% \section{Раздел}

% \subsection{Подраздел}

% В обзоре литературы, охватывающем не менее 30 источников за последние 10–15 лет (включая зарубежные публикации и электронные ресурсы), необходимо показать основные этапы в развитии знания по проблеме диссертации, критически осветив известные работы, необходимо назвать неразрешенные вопросы и таким образом определить свое место в решении проблемы. Желательно закончить этот раздел кратким резюме о той конкретной задаче, которую автор стремиться поставить и решить в диссертации.

% \chapter{Теоретический раздел}

% В последующих главах с исчерпывающей полнотой  необходимо изложить собственное исследование с выявлением того нового и оригинального, что вносится в разработку проблемы. Все идеи и положения автора должны быть обстоятельно обоснованы на базе принятой методики, вытекающей из сущности предмета исследования.

% \chapter{Экспериментальный раздел}

% Весь порядок изложения в диссертации должен быть подчинен  руководящей идее, четко сформулированной в теоретическом разделе диссертации. Логичность построения и целеустремленность изложения  глав достигается в случае, если каждая из глав имеет определенное целевое назначение и является базой для последующей.
