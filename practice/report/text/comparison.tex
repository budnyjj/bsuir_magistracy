\chapter{Исходные данные для сравнения методов аппроксимации}

В работах~\cite{budny17,budny15} было выполнено сравнение точности оценивания параметров
зависимостей по измерениям с ошибками, полученных различными методами.

Рассматривался объект вида
\begin{equation}
  \label{eq:model}
  \begin{aligned}
  h &= \psi (\alpha, \beta, \xi), \\
  x &= \xi + \varepsilon, \\
  y &= h + \delta
  \end{aligned}
\end{equation}
где
\( \xi, h \) --- фактические значения входной и выходной переменной,
\( \alpha, \beta \) --- фактические значения параметров объекта,
\( \psi \) --- скалярная функция регрессии,
\( x, y \) --- измеренные значения входной и выходной переменной,
\( \varepsilon, \delta \) --- независимые ошибки измерений значений входной и выходной переменной,
распределенные по нормальному закону:
\( \varepsilon = N(0, \sigma_{\varepsilon}), \delta = N(0, \sigma_{\delta}) \).

Исследовалась зависимость точности оценок параметров от с.~к.~о.
ошибок измерений \( \sigma_{\varepsilon}, \sigma_{\delta} \).

В качестве величины, характеризующей точность оценивания,
использовалось среднее Евклидово расстояние в пространстве параметров:
\begin{equation}
  \label{eq:quality}
  d = \frac{1}{k} \sum_{j=1}^k \sqrt{(\hat{\alpha}_j - \alpha)^2 + (\hat{\beta}_j - \beta)^2}.
\end{equation}

В работе~\cite{budny17} было выполнено сравнение точности оценок параметров
\( \hat{\alpha}, \hat{\beta} \) линейной зависимости \( \psi = \alpha + \beta \xi \),
полученных классической линейной регрессией и методом симметричной аппроксимации.

Значения \( \xi_i \) выбирались из равномерного в \( [0, 10] \) распределения.
Для получения каждой оценки \( ( \alpha, \beta ) \) использовались результаты
ста наблюдений \( ( x_i, y_i ), i = \overline{1, n}, n = 100 \).

Расчеты расстояний~\eqref{eq:quality} производились в узлах сетки значений
\( \sigma_{\varepsilon}, \sigma_{\delta} \) в прямоугольнике
\( [0, 2] \times [0, 2] \) с шагом 0{,}1.
В каждом узле сетки вычислялось \( k = 100 \) оценок.

В работе~\cite{budny15} было выполнено сравнение точности оценивания параметров,
полученных нелинейным методом наименьших квадратов и методом рядов Тейлора,
для следующих зависимостей:
\begin{itemize}
\item линейной --- \( \psi = \alpha + \beta \xi, \alpha = 31, \beta = 0{,}5 \);
\item экспоненциальной --- \( \psi = \alpha e^{\beta \xi}, \alpha = 31, \beta = 0{,}5 \);
\item синусоидальной --- \( \psi = \alpha + \beta \sin \xi, \alpha = 31, \beta = 100 \).
\end{itemize}

Значения \( \xi_i \) выбирались из равномерного в \( [0, 10] \) распределения.
Для получения каждой оценки \( ( \alpha, \beta ) \) использовались результаты
двадцати наблюдений \( ( x_i, y_i ), i = \overline{1, n}, n = 20 \).

Расчеты расстояний~\eqref{eq:quality} производились в узлах сетки значений
\( \sigma_{\varepsilon}, \sigma_{\delta} \) в прямоугольнике
\( [0, 0{,}1] \times [0{,}1, 10] \) с шагом 0{,}1.
В каждом узле сетки вычислялось \( k = 100 \) оценок.
