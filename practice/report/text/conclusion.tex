\chapter*{Заключение}
\addcontentsline{toc}{chapter}{Заключение}

В данной работе был приведен ряд примеров постановки задач
аппроксимации статистических данных,
описаны исходные данные для сравнения точности оценивания параметров
линейных и нелинейных зависимостей по измерениям с ошибками.

В связи с родственностью решаемых задач автору работы представляется логичным
изложение методов рядов Тейлора и симметричной аппроксимации в
рамках раздела учебной дисциплины <<Статистические методы обработки данных>>,
посвященного методам нахождения точечных оценок параметров распределений.
Поскольку различия между целевыми функциями методов наименьших квадратов и
симметричной аппроксимации понятны интуитивно, а также простой их реализации,
предлагается включить их применение в лабораторный практикум по данной дисциплине.